\chapter{Υλοποιήσεις Αλγορίθμων}
\label{implementations}

Το Nidhugg είναι εργαλείο εντοπισμού σφαλμάτων που στοχεύει στην ανεύρεση σφαλμάτων που προκύπτουν από το μη ντετερμινισμό του δρομολογιτή
και σε λάθη που προκύπτουν από τα μοντέλα χαλαρής μνήμης.
Δουλεύει στο επίπεδο της ενδιαμέσης αναπαράστασης του LLVM, που σημαίνει ότι μπορεί να χρησιμοποιηθεί για προγράμματα που είναι γραμμένα
σε C και C++ ή γλώσσες που μπορούν να γίνουν compile σε LLVM και υλοποιούν τον ταυτοχρονισμό χρησιμοποιώντας τη βιβλιοθήκη pthreads.

Τη στιγμή που γραφόταν η παρούσα διπλωματική το Nidhugg υποστήριζε τα μοντέλα SC, TSO, PSO and POWER. 
Από την άλλη το Nidhugg δεν μπορεί να χειριστεί τον μη ντετερμινισμό των δεδομένων γι αυτό τα threads θα πρέπει να είναι ντετερμινιστικά 
όταν τρέχουν σε απομόνωση.

\section{Η Ροή Προγράμματος του Nidhugg}

Το Nidhugg δουλεέυι στο επίπεδο της ενδιάμεσης αναπαράστασης του LLVM (IR). Προκειμένου το Nidhugg να βρέι ένα σφάλμα δημιουργεί
ένα διερμηνέα LLVM assembly.
Στη συνέχεια δρομολογεί και εκτελεί διαφορετικα traces μέχρι ένα σφάλμα να βρεθεί όπως η παραβίαση ενός assertion. 
Τα traces παίζουν ένα σημαντικό ρόλο στο Nidhugg καθώς αναπαριστούν τα διαφορετικά schedulings.
Αυτά τα traces αντιπροσωπεύονται ως vectors από Events objects. Το Event object συντηρεί όλη την χρήσιμή πληροφορία σχετικά με
ένα event όπως το pid του thread που εκτελέστηκε. Οι διακλαδώσεις που προκαλούν την εξερεύνηση διαφορετικών δρομολογήσεων
αποθηκεύονται επίσης στο Event object. Οι δρομολογήσεις ρυθμίζονται από το Tracebuilder object με τη σειρά του καθορίζεται από
το memory model που χρησιμοποιείται. Ο Tracebuilder είναι επίσης υπεύθυνος για τον ανεύρεση races μεταξύ των διαφόρων threads και τις προσβάσεις
στη μνήμη.

Η εκτέλεση ακολουθεί γενικά τη ροή που παρουσιάζεται στο Figure \ref{Nidhugg's Flow Chart}. Όπως βλέπουμε και στο διάγραμμα
ροή ο TraceBuilder προσπαθεί να δρομολογήσει καινούρια γεγονότα με τη χρήση της schedule(). 
Μετά τη δρομολόγηση τα γεγονότα εκτελούνται και ενημερώνονται τα διανυσματικά ρολόγια. Στη συνέχεια ελέγχεται 
αν κάποιο γεγονός είναι εξαρτόμενο από κάποιο άλλο, δηλαδή, προσπελαύνει κάποια τοποθεσία της μνήμης που ανήκει και στο
άλλο. Μετά από αυτό το Niddhugg προσπαθεί να προσθέσει διακλαδώσεις και να ελέγξει αν υπάρχει κάποιο σφάλμα.
Στην περίπτωση σφάλματος ο χρήστης ενημερώνεται και η διαδικασία τερματίζει.  
Αξίζει να σημειωθεί ότι υπάρχει επιλογή η εξερεύνηση να συνεχίσει ώστε να βρεθούν περισσότερα λάϑη. 
Αν στο τέλος μιας δρομολόγησης δεν έχει βρεθεί κάποιο λάθος ο TraceBuilder κάνει reset στην πιο κοντινή 
διακλάδωση και συνεχίζει την εξερεύνηση. Αν δεν υπάρχουν άλλες διαθέσιμες διακλαδώσεις, η εξερεύνηση τερματίζει.

Σε ό,τι αφορά του αλγορίθμους που υλοποιήθηκαν είναι ξεκάθαραο ότι το πιο σημαντικό κομμάτι του διαγράμματος ροής είναι ο εντοπισμός εξαρτήσεων.

\mediumGraph{flowchartv2.pdf}{Nidhugg's Flow Chart}

\section{Προσθήκη διακλαδώσεων στο Nidhugg}

Μόλις δρομολογηθεί μια εντολή που πιθανώς να προκαλέσει σφάλμα το διάνυσμα see\_accesses δημιουργείται και περιλαμβάνει όλες τις προσπελάσεις 
που λαμϐανουν χώρα στη ίδια θέση στη μνήμη και καλείται η διαδικασία \verb|see_events()|.

\begin{algorithm}
    \caption{see\_events()}
    \Fn{$see\_events(seen\_access)$}{
        $branches := seen\_access - \{ a \in seen\_access \mid a  \rightarrow last(E)$ or $\exists a' \in seen\_access$ which happens after $ a \}$ \;
        $update\_clocks()$ \;
        \ForEach{$b \in  branches $}{
            $add\_branch(b)$ \;
        }
     }
\end{algorithm}

Είναι σαφές από τον αλγόριθμο ότι ο σκοπός της συνάρτησης είναι εξαιρέσει όλες εκείνες τις προσβάσεις στη μνήμη που δεν είναι σε
race με την συγκεκριμένη πρόσβαση δηλαδή εξαρτήσεις που δεν μπορούν να αναπαρασταθούν
σαν traces στα οποία δεν υπάρχουν άλλα γεγονότα που να συμβαίνουν μεταξύ τους.
Είναι σημαντικό να ξεκαθαρίσουμε ότι αυτό δε σημαίνει ότι τα γεγονότα δεν μπορούν να είναι ταυτόχρονα σε κάποια άλλη δρομολόγηση.
Τα γεγονότα που επιβίωσαν της διαδικασίς προστίθενται σε ένα πίνακα ο οποίος χρησιμοποιείται από την συνάρτηση \verb|add_branch()|.

Μία άλλη λειτουργία της \verb|see_events()| είναι η ενημέρωση των vector clocks. Μετά το τέλος της ρουτίνας για δύο γεγονότα που είναι ταυτόχρονα
τα διανυσματικά ρολόγια θα έχουν ενημερωθεί ώστε να φαίνεται η σχέση-πριν (happens-before relation).

H συνάρτηση \verb|add_branch()| φαίνεται στον αλγόριθμο \ref{add_branch} είναι η πιο σημαντική στην υποδομή του Nidhugg.

\begin{algorithm}
    \caption{add\_branch()}
    \label{add_branch}
    \Fn{add\_branch($b$)}{
        $candidates$ = $\emptyset$ \;
        $lc := null$ \;
        $E' := E \text{ starting from } next(b)$ \; 
        \ForEach{$e \in dom(E')$}{
            \If{$b \rightarrow e$ or $\exists c \in candidates : c \rightarrow e $}{
                continue \;
            }

            $lc := e.pid$\;
            \If{$lc \in candidates$}{
                continue \;
            }

            \If{$e.pid \in backtrack(b)$ or $e.pid \in sleep\_set$}{
                    return \;
            }
            $candidates := candidates \cup lc$ \;
        }
        $backtrack(b) := backtrack(b) \cup lc $ \;
    }
\end{algorithm}

Διαισθητικά, η add\_branch κάνει τα ακόλουθα: Ξεκινώντας από ένα γεγονός που έχει conflict με το πιο πρόσφατα δρομολογημένο γεγονός,
ξεκινά να διανύει το διάνυσμα που αντιπροσωπεύει το $E$ μέχρι το κοντινότερο στο τέλος του trace, thread βρεθεί (αν
αυτό είναι δυνατό). Στην πραγματικότητα αυτό που συμβαίνει είναι ο υπολογισμός των $I$ (initials). Όπως φαίνεται και από τον
αλγόριθμο αν έχει ήδη τοποθετηθεί μια κατάλληλη διακλάδωση ή μια κατάλληλη διακλάδωση είναι στο sleep set η διαδικασία 
τερματίζει.

Το αποτέλεσμα της \verb|add_branch()| είναι ο υπολογισμός ενός source set.

\section{Υλοποίηση του Nidhugg-DPOR}

Η υλοποίηση των persistent sets βασίζεται στην ήδη υλοποιημένη υποδομή των vector clocks.
Συγκεκριμένα η \verb|include()| συνάρτηση των vector clocks μας επιτρέπει να καθορίσουμε αν υπάρχουν σχέσεις $i \rightarrow p$. 
    Ο αλγόριθμος που αυτή υλοποιεί \verb|includes()| δίνεται στον Αλγόριθμο \ref{include routine}. Ακολουθώντας τον συμβολισμό του Source-DPOR το $\langle e,
i \rangle$ είναι ένα ήδη δρομολογημένο γεγονός και είναι το  i-οστό βήμα της διεργασίας $e$ και το $p$ είναι το πιο πρόσφατα backtracked
γεγονότ της διεργασίας $p$. 

\begin{algorithm}
    \caption{includes() routine}
    \label{include routine}
    \Fn{includes($ \langle e, i \rangle$, p)}{
        return $i \leq p.clock[e]$
    }
\end{algorithm}

Για να υπολογίσουμε το $CI$ χρειάζεται απλλά να προλάβουμε την \verb|add\_branch()| από το να απορρίψει διακλαδώσει λόγω thread  που ανήκουν στο 
$I$ και όχι στο $CI$. Η αλλαγές που έγιναν δίνονται παρακάτω στον Αλγόριθμο \ref{addbranch
routine for persistent sets}.

\begin{algorithm}
    \caption{add\_branch() for persistent sets}
    \label{addbranch routine for persistent sets}
    \Fn{add\_branch($b$)}{
        $candidates = \emptyset$ \;
        $lc = null$
        $E'$ := $E text{starting from} next(b)$ \; 
        \ForEach{$e in dom(E')$}{
            \If{$b \rightarrow e \text{ or } \exists c \in candidates \mid c \rightarrow e $}{
                continue \;
            }

            \If{$e \rightarrow last(E)$}{
                $lpc := e.pid$\;
            }

            $lc := e.pid$\;

            \If{$e.pid \in backtrack(b) or e.pid \in sleep\_set(b)$}{
                \If{$lc \rightarrow last(E)$}{
                    return \;
                }
            }
        }
        \If{$lpc$}{
            $backtrack(b) := backtrack(b) \cup lpc $
        }
        \Else{$backtrack(b) := backtrack(b) \cup lc $}
    }
\end{algorithm}

Στην περίπτωση που το E είναι κενό μπορούμε να χρησιμοποιήσουμε έναν υποψήφιο που υποδείχτηκε από το $I$.

\section{Υλοποίηση του Naive-BPOR}

Το πρώτο που πρέπει να υλοποιήσουμε για κάθε bounding technique είναι ένα μετρητής του bound count. Στην περίπτωση 
μας ένα μετρητής που θα μας δίνει πόσα preemptive switces έχουν συμβεί στο trace.

Πρέπει να είμαστε σε θέση να καταλάβουμε πότε ένα thread είναι ενεργό. Αυτήν την πληροφορία μπορούμε να την πάρουμε
από παλαιότερες τιμές του ίδιου του bound count.
Δοθέντων δύο διαδοχικών γεγονότων $a,b$, αν $a.bound\_count < b.bound\_count$ τότε $a$ ήταν διαθέσιμο.

Ο ψευδοκώδικας δίνεται στον Αλγόριθμο \ref{Should we increase the bound count?}.

\begin{algorithm}
    \caption{Should we increase the bound count?}
    \label{Should we increase the bound count?}
    \Let{$i =\text{the most recent branching point} $}
    $bound\_count := prefix[i].bound\_count$  \;
    \If{$i>0$}{
        \If{$prefix[i].id == prefix[i-1].id$}{
            $prefix[i].bound\_count = ++bound\_count$\;
        }
        \Else{
            $prefix[i].bound\_count = bound\_count$ \;
        }
    }

\end{algorithm}


Προκειμένου να μπορέσουμε να επιβεβαιώσουμε την επιβεβαίωση της σωστής μέτρηση του bound count, το debug print του Nidhugg τροποποιήθηκε κατάλληλα.
Έτσι ο μετρητής πρέπει να δουλεύει όπως φαίνεται στο Listing \ref{Example of bound counter}.

\Output{./code/bound_count.out}{Example of bound counter}

Το Nidhugg αναμένει τα traces να μπλοκάρονται μόνο λόγω sleep sets. Έτσι έπρεπε να γίνουν οι κατάλληλες αλλαγές ώστε να γνωρίζουμε 
το λόγο που δρομολόγηση σταμάτησε. Συγκεκριμένα προστέθηκαν flags στην συνάρτηση \verb|schedule()| γι αυτό το σκοπό.

Το αποτέλεσμα μετά την εκτέλεση του προγράμματος φαίνεται στο Listing \ref{Naive-BPOR output}.

\Output{./code/vanilla_output.out}{Naive-BPOR output}

Μια αλλαγή που πρέπει να κάνουμε είναι να αλλάξουμε τη προτεραιότητα με την οποία δρομολογούνται τα threads. Το Nidhugg
δρομολογεί τα threads δίνοντας προτεραιότητα στο παλαίοτερο. Έτσι μόλις ένα παλιό thread ξυπνήσει θα δρομολογηθεί αμέσως.
Με αποτέλεσμα να αυξηθεί το preemption count. Επομένως πρεπει να αλλάξουμε την προτεραίοτητα δρομολόγησης ώστε να 
προηγείται πάντα το thread που έτρεχε προηγούμενως ώστε να αποφύγουμε αυξήσεις του bound count.

\trace{w2rscheduling.pdf}{Execution without the scheduling optimization}

Στο Figure \ref{Execution without the scheduling optimization} δίνεται ένα παράδειγμα με δύο threads να διαβάζουν μια μεταβλητή και 
ενα να τη γράφει. Όπως φαίνεται και στην εικόνα όταν το $p$ ξυπνά λόγω του conflict τότε δρομολογείται αμέσως αυξάνοντας το bound count.

\section{Υλοποίηση του Nidhugg-BPOR}

Όπως είδαμε και προηγούμενως οι βασικές τροποποιήσεις πρέπει να γίνουν στις συναρτήσεις see\_events και add\_branch. 
Ο ψευδοκώδικας για το see\_events φαίνεται στον Αλγόριθμο
\ref{seeevents routine for BPOR}.

Κατα τη διάρκει του see\_events προσθέτουμε διακλαδώσεις και στην αρχή του block όπου αυτό είναι δυνατό. Για να το κάνουμε 
αυτό πρέπει να ελέγξουμε αν το thread που προσθέτουμε ήταν ενεργό στην αρχή του block.

\begin{algorithm}[H]
    \caption{see\_events() for BPOR}
    \label{seeevents routine for BPOR}
    Explore($\langle \rangle$,$\emptyset$)\;
    \Fn{see\_events($seen\_access$)}{
        $branches := seen\_access - \{ a \in seen\_access \mid a$ happens before $last(E) or \exists a' \in seen\_access$ which happens after $a \}$ \;
        $update\_clocks()$ \;
        \ForEach{$b \in branches$}{
            $add\_branch(b)$ \;
            \If{$last(E).id \in enabled($ at the beggining of block $b)$}{
                $add\_branch($ at the beginning of block $b)$ \;
            }
        }
     }
\end{algorithm}

Στην περίπτωση που η \verb|add_branch()| εκτελείται απευθείας χρησιμοποιύμε την \verb|add_conservative_branch()| που λειτουργεί ώς το ``συντηρητικό''
κομμάτι της \verb|see_events()|. Ο ψευδοκώδικας της συνάρτησης δίνεται στον Αλγόριθμο \ref{try to add conservative branch}.
Η \verb|add_branch()| έχει ήδη περιγραφεί στον Αλγόριθμο \ref{addbranch routine for persistent sets}.

\begin{algorithm}[H]
    \caption{try\_to\_add\_conservative\_branches()}
    \label{try to add conservative branch}
    Explore($\langle \rangle$,$\emptyset$)\;
    \Fn{try\_to\_add\_conservative\_branches($b$)}{
        \If{$last(E).id \in enabled($ at the beggining of block $b)$}{
            $add\_branch($ at the beginning of block $b)$ \;
        }
    }
\end{algorithm}

Κατά την \verb|add_branch()| προσθέτουμε διακλαδώσεις στα κατάλληλα σημεία
χρησιμοποιώντας την λογική των persistent sets όπως είδαμε σε προηγούμεν ενόητα.
Όπως αναφέραμε προστίθενται δύο ειδών διακλαδώσεις. Κατά την αναζήτηση μας στο sleep set θα πρέπει να αναζητούμε 
μόνο τα μη συντηρητικά threads. Στην περίπτωση που μια διακλάδωση του ίδιου thread προστίθεται και σαν συντηρητική και σαν
μη συντηρητική τότε η συντηρητική υπερισχύει.

\section{Υλοποίηση του Source-BPOR}

Η υλοποίηση ξανά βασίζεται σε τροποποιησεις στις συναρτήσεις \verb|add_branch()| καθώς και στην
\verb|see_events()|.  Μπορούμε να παρατηρήσουμε ότι ο αλγόριθμος θα διαλέξει τον ίδιο υποψήφιο που θα διάλεγε ο Source-DPOR 
όταν θα προσέθετε μη συντηρητικές διακλαδώσεις και τον ίδιο υποψήφιο με τον Nidhugg-BPOR όταν θα διαλέξει συντηρητικές διακλαδώσεις.
Ο ψευδοκόωδικάς για την 
\verb|add_branch()| δίνεται στον Αλγόριθμο \ref{addbranch routine for Source-BPOR}.

\begin{algorithm}[H]
    \caption{add\_branch() routine for Source-BPOR}
    \label{addbranch routine for Source-BPOR}
    \Fn{add\_branch($b,is\_conservative$)}{
        $candidates = \emptyset$ \;
        $lc = null$ \;
        $lpc = null$ \;
        $E'$ := $E$ starting from $next_{E}(b)$ \; 
        \ForEach{$e \in dom(E')$}{
            \If{$b \rightarrow e$ or $\exists c \in candidates \mid c \rightarrow e $}{
                continue \;
            }

            \If{$e \rightarrow last(E)$}{
                $lpc := e.pid$\;
            }

            $lc := e.pid$\;

            \If{$e.pid \in backtrack(b)$ or $e.pid \in sleep\_set(b)$}{
                \If{not is\_conservative or $ lc \rightarrow last(E)$}{
                    return \;
                }
            }
        }
        \If{$lpc$ and $is\_conservative$}{
            $backtrack(b) := backtrack(b) \cup lpc $
        }
        \Else{$backtrack(b) := backtrack(b) \cup lc $}
    }
\end{algorithm}
