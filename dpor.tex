\chapter{Dynamic Partial Order Reduction}
\label{dpor}

\section{Basic DPOR Concepts}

Generally, DPOR algorithms use a depth-first backtrack search to explore the state-space of a concurrent system.
This exploration is driven by two basic concepts: \textit{persistent sets} and \textit{sleep sets}, which make sure to explore a
sufficient portion (at least one interleaving from different Mazurkiewicz traces) of the state-space,
while trying to minimize any unnecessary exploration.

Intuitively, a persistent set at a state $s$ is (specific) subset of $enabled(s)$ whose exploration guarantees that all
non-equivalent interleavings (from different Mazurkiewicz traces) will be explored. This is vital in proving the correctness of Classic 
DPOR algorithms \cite{FlanaganDPOR}, on the assumption 
(that is taken into account by our abstraction) that our state space is acyclic and finite.
The way that such sets are constructed differs from paper to paper \cite{FlanaganDPOR, Lei:2006:RTC:1248722.1248743, 10.1007/3-540-53863-1_36},
and those variations can lead to different degrees of state-space reduction.

The sleep set technique, being complimentary to persistent sets (it does not contribute to the soundness of algorithm),
aims to further reduce the number of the explored interleavings.
A sleep set at an execution sequence $E$ contains processes, whose exploration would be redundant,
preventing equivalent interleavings from being explored.

Specifically, after $E.p$ has been explored, process $p$ is added to the sleep set
at $E$. From this point on, $p$ will exist in any sleep set of an execution sequence
of the form $E.w$, provided that $E.w$ is also an execution sequence and $E\models p \diamondsuit w$. 
The processes in the sleep set are not going to be executed from this point on, unless a dependency gets detected. 
For instance, $p$ will be removed from the sleep set at $E.w.q$ if a dependency gets detected between the
next steps of $q$ and $p$. 

It can be proved \cite{Godefroid1996} that sleep sets will eventually block all the redundant interleavings and thus 
the only interleavings that are going to be explored fully will belong to different Mazurkiewicz traces.
However, this does not mean sleep sets avoid all redundant explorations. To elaborate, sleep sets 
make it possible for the exploration of an interleaving, which belongs to the same Mazurkiewicz trace as another
interleaving that has already been explored, to eventually block, by having all of its enabled processes
appear in the sleep set. This is called \textit{sleep-set blocking} and it means that all possible traces, from this point on,
are redundant and therefore, need not be explored further. Sleep sets do not guarantee optimality 
for a DPOR algorithm, since redundant traces get explored, albeit not completely.

Source-DPOR \cite{AbdullaAronisJohnssonSagonasDPOR2014} replaces persistent sets with \textit{source sets} in order to achieve a 
significantly better reduction in the amount of the explored interleavings. However, source-DPOR still suffers from sleep-set 
blocking. Optimal-DPOR \cite{AbdullaAronisJohnssonSagonasDPOR2014} 
combines the concept of source sets with \textit{wakeup trees} to fully avoid sleep-set blocking and lead to the exploration
of an optimal subset of interleavings.

\section{Source Sets}

Before defining source sets formally, we need to define the concepts 
of possible initial steps in an execution sequence \cite{AbdullaAronisJohnssonSagonasDPOR2014}:

\begin{definition}{(Initials after an execution sequence $E.w$, $I_{[E]}(w)$)}\\
For an execution sequence $E.w$, let $I_{[E]}(w)$ denote the set of
processes that perform events $e$ in $dom_{[E]}(w)$ that have no
“happens-before” predecessors in $dom_{[E]}(w)$. More formally,
$p \in I_{[E]}(w)$ if $p \in w$ and there is no other event $e \in dom_{[E]}(w)$ with
$e \rightarrow_{E.w} next_{[E]}(p)$.
\end{definition}

By relaxing this definition, we can get the definition of Weak Initials, $WI$:

\begin{definition}{(Weak Initials after an execution sequence $E.w$, $WI_{[E]}(w)$)}\\
For an execution sequence $E.w$, let $WI_{[E]}(w)$ denote the union of $I_{[E]}(w)$ and the set of
processes that perform events $p$ such that $p \in enabled(s_{[E]}) $.
\end{definition}

To clarify these notations, for an execution sequence $E.w$:
\begin{itemize}
    \item  $p \in I_{[E]}(w)$ iff there is a sequence $w'$ such that $E.w \simeq E.p.w'$, and
    \item  $p \in WI_{[E]}(w)$ iff there are sequences $w'$ and $v$ such that $E.w.v \simeq E.p.w'$.
\end{itemize}

\begin{definition}{(Source Sets)}\label{def:Source Sets}\\
Let $E$ be an execution sequence,
and let $W$ be a set of sequences, such that $E.w$ is an execution
sequence for each $w \in W$. A set $T$ of processes is a source set for
$W$ after $E$ if for each $w \in W$ we have $WI_{[E]}(w) \cap P  = \emptyset$.
\end{definition}

A direct consequence of this definition is that every 
set of processes that can cover the complete state space after an execution sequence $E$
can be considered a source set of $E$.

\section{Source-DPOR}
Here we present the source-DPOR algorithm \cite{AbdullaAronisJohnssonSagonasDPOR2014}.

\SetKwProg{Fn}{Function}{}{}
\SetKwHangingKw{Let}{let}
\begin{algorithm}
    \caption{Source-DPOR}
    \label{Source}
    \Fn{Explore($E$,$Sleep$)}{
        \If{$\exists p \in (enabled(s_{[E]}) \backslash Sleep)$}{
            $backtrack(E) :={p}$\;
            \While{$\exists p \in (backtrack(E) \backslash Sleep)$}{
                \ForEach{$e \in dom(E)$ such that $e \lesssim_{E.p} next_{[E]}(p)$}{
                    \Let{$E' = pre(E,e)$}
                    \Let{$u = notdep(e,E).p$}
                    \If{$I_{[E']}(u) \cap backtrack(E') = \emptyset$}{
                        add some $q' \in I_{[E']}(u) \text{ to } backtrack(E')$\;
                    }
                }
                \Let{$Sleep' := \{q \in Sleep \mid E \models p \diamondsuit q \} $}
                $Explore(E.p, Sleep')$\;
                add $p$ to $Sleep$\;

            }
        }
    }
\end{algorithm}

An execution step $Explore(E, Sleep)$ is responsible for the explorations of all Mazurkiewicz traces 
that begin with the prefix $E$. These explorations start by initializing $backtrack(E)$ with an arbitrary enabled process
that is not in the sleep set ($Sleep$). From this point forward, for each process $p$ that exists in $backtrack(E)$
source-DPOR will perform two main phases.

During the first phase (race detection), we find all events $e$ that occur in $E$ (i.e., $e \in dom(E)$) which are racing
with the next event of $p$, and that race can be reversed ($e \lesssim_{E.p} next_{[E]}(p)$). For each such event $e$, we 
are trying to reverse that race by ensuring that the next event of $p$ gets performed before $e$, or using the symbolism
of the algorithm, that a sequence equivalent to $E.notdep(e,E).p.proc(e).z$ ($z$ is any continuation of the execution sequence)
is explored.
Such a trace could be explored by taking the next available step from any process in $I_{[E']}(notdep(e,E).p)$ (where $E'=pre(E,e)$)
at $E'$. Therefore, a process from $I_{[E']}(notdep(e,E).p)$ is added to the backtrack set  at $E'$, provided that it is not already there.

During the last phase (exploration), we recursively explore $E.p$. The sleep set at $E.p$ is initialized appropriately 
by taking the sleep set at $E$ and removing all processes whose next step is dependent with the next step of $p$. This ensures
that at $E.p$ we will not consider the races of processes that have already been considered, unless they race with the new
scheduled process $p$. After $E.p$ finished its exploration, $p$ is added to the sleep set at $E$,
because we want to refrain from executing an equivalent trace.

Practically, on Concuerror the algorithm is structured differently. The main difference is that the algorithm goes into the race
detection phase when an interleaving has reached its end. At this point all the races that occurred in the interleaving
are detected and backtrack points are inserted in the appropriate prefixes of the complete execution sequence. Then the exploration
continues by exploring the backtrack set of the longest prefix first. This does not affect the soundness of the algorithm 
since the only thing that changes is the order in which new interleavings are explored.

We should note here that $Explore(E, Sleep)$ does not need any additional information from the various prefixes of $E$ that
has not already been established. However, $Explore(E, Sleep)$ can add backtrack points to the various prefixes of $E$. This
is vital and must be taken into consideration while trying to parallelize source-DPOR.

\section{Wakeup Trees}

In order to achieve optimality, we must completely avoid having sleep-set blocked interleavings. This is achieved by
combining a mechanism called wakeup trees \cite{AbdullaAronisJohnssonSagonasDPOR2014} with source sets.

Notice that in source-DPOR a sequence of the form $E'.notdep(e,E).p.proc(e).z$ needs to be explored, but only
a single process from the Initials set of $notdep(e,E).p$ is potentially added to the backtrack set.
Therefore, a piece 
of information on how to reverse the race gets lost. This may lead to sleep-set blocking, since an alternative sequence could be explored 
instead. Intuitively, wakeup trees hold in the form of a tree the fragments that need to be explored in order
to explore the necessary interleavings, while avoid sleep-set blocking.

In order to define wakeup trees, we first present the generalizations of 
the concepts of Initials and Weak Initials so they can contain sequences of processes instead of just processes:

\begin{itemize}
    \item $v \sqsubseteq_{[E]} w $ denotes that exists a sequence $v'$ such that $E.v.v'$ and $E.w$ are execution
    sequences with the relation $E.v.v' \simeq E.w$. What this means is that after $E$, $v$ is a possible
    way to start an execution that is equivalent to $w$. To connect this to the concept of Initials we have
    $p \in I_{[E]}(w)$ iff $p \sqsubseteq_{[E]} w $.
    \item $v \sim_{[E]} w $ denotes that exist sequences $v'$ and $w'$ such
    that $E.v.v'$ and $E.w.w'$ are execution sequences with the relation $E.v.v' \simeq E.w.w'$.
    What this means is that after $E$, $v$ is a possible way to start an execution that is equivalent to $E.w.w'$.
    To connect this to the concept of Weak Initials we have $p \in WI_{[E]}(w)$ iff $p \sim_{[E]} w $.
\end{itemize}

\begin{definition}{(Ordered Tree)}\label{def:Ordered}\\
An $ordered$ $tree$ is a pair $\langle B , \prec \rangle$, where B (the set of nodes) is a finite prefix-closed
    set of sequences of processes with the empty sequence $\langle\rangle$ being the root.
    The children of a node $w$, of form $w.p$ for some set of processes $p$, are ordered by $\prec$. 
    In $\langle B , \prec \rangle$, such an ordering between children has been extended to the total 
    order $\prec$ on $B$ by letting $\prec$ be the induced post-order relation between the nodes in $B$.
    This means that if the children $w.p_1$ and $w.p_2$ are ordered as $w.p_1 \prec w.p_2$,
    then $w.p_1 \prec w.p_2 \prec w $ in the induced post-order.
\end{definition}

\begin{definition}{(Wakeup Tree)}\\
    Let $E$ be an execution sequence and $P$ a set of processes. a $wakeup$ $tree$ after $\langle E , P \rangle$
    is an ordered tree $\langle B , \prec \rangle$, for which the following properties hold:
    \begin{itemize}
        \item $WI_{[E]}(w) \cap P = \emptyset$ for every leaf $w$ of $B$.
        \item For every node in $B$ of the form $u.p$ and $u.w$ such that $u.p \prec u.w$ and $u.w$ is a leaf
        the $p \not \in WI_{[E.u]}(w)$ property must hold true.
    \end{itemize}
\end{definition}

\subsection{Operations on Wakeup Trees}



An important operation used by the optimal-DPOR algorithm is the insertion of new initial fragments of interleavings,
which need to be explored, into the wakeup tree.

Considering a wakeup tree $\langle B , \prec \rangle$ after $\langle E , P \rangle$ and some sequence $w$ with
$E.w$ being an execution sequence such that $WI_{[E]}(w) \cap P = \emptyset$, the following properties are used to 
define the $insert_{[E]}(w,\langle B , \prec \rangle)$:

\begin{itemize}
    \item $insert_{[E]}(w,\langle B , \prec \rangle)$ is also a wakeup tree after $\langle E , P \rangle$.
    \item Any leaf of $\langle B , \prec \rangle$ remains a leaf of $insert_{[E]}(w,\langle B , \prec \rangle)$.
    \item $insert_{[E]}(w,\langle B , \prec \rangle)$ contains a  leaf $u$ with $u \sim_{[E]} w$.
\end{itemize}

Let $v$ be the smallest (according to the $\prec$ order) in $B$ with $v \sim_{[E]} w$. The operation 
$insert_{[E]}(w,\langle B , \prec \rangle)$ can either be taken as $\langle B , \prec \rangle$, provided that
$v$ is a leaf, or by adding $v.w'$ as a leaf and ordering it after all existing nodes in $B$ of form $v.w''$,
where $w'$ is the shortest sequence with $w \sqsubseteq_{[E]} v.w'$.

Let us also describe the $subtree(\langle B , \prec \rangle, p)$ operation. For a wakeup tree $\langle B , \prec \rangle$
and a process  $p \in B$, $subtree(\langle B , \prec \rangle), p)$ is used to denote the
subtree of $\langle B , \prec \rangle$ rooted at $p$. More formally, $subtree(\langle B , \prec \rangle), p) 
= \langle B' , \prec' \rangle$ where $B' = \{w \mid p.w \in B \}$ and $\prec'$ is the extension of $\prec$ to $B'$.

\section{Optimal-DPOR} 

Here the optimal-DPOR algorithm is presented as described in its original paper \cite{AbdullaAronisJohnssonSagonasDPOR2014}.

\SetKwProg{Fn}{Function}{}{}
\SetKwHangingKw{Let}{let}
\begin{algorithm}
    \caption{Optimal-DPOR}
    \label{optimal}
    \Fn{Explore($E$,$Sleep$,$WuT$)}{
        \uIf{$enabled(s_{[E]}) = \emptyset$}{
            \ForEach{$e,e' \in dom(E)$ such that ($e \lesssim_{E} e'$)}{
                \Let{$E' = pre(E,e)$}
                \Let{$v = notdep(e,E).proc(e')$}
                \If{$sleep(E') \cap WI_{[E']}(v)= \emptyset$}{
                    $insert_{[E']}(v,wut(E'))$\;
                }
            }
        }
    \Else {
        \uIf{$WuT \not = \langle \{ \langle \rangle \}, \emptyset \rangle$}{
            $wut(E) := WuT$\;
        }
        \Else {
            choose $p \in enabled(s_{[E]})$\;
            $wut(E) := \langle \{ p \}, \emptyset \rangle $\;
        }
        $sleep(E) := Sleep$\;
        \While{$\exists p \in wut(E)$}{
            \Let{ $p = min_{\prec}\{ p \in wut(E)\}$}
            \Let{$Sleep' := \{q \in sleep(E) \mid E \models p \diamondsuit q \} $}
            \Let{$WuT' = subtree(wut(E), p)$}
            $Explore(E.p, Sleep', WuT')$\;
            add $p$ to $sleep(E)$\;
            remove all sequences of form $p.w$ from $wut(E)$\;
        }
    }
    }
\end{algorithm}

Similarly, with the other algorithms, optimal-DPOR has two different phases: race detection and state exploration. 
However, the algorithm is structured differently. In the same way that source-DPOR is structured at Concuerror,
optimal-DPOR only detects the races when a maximal execution sequence has been reached (i.e., there exist no
enabled processes). This is necessary because the condition for inserting new wakeup trees is only
valid when the fragment that is going to be inserted contains all the events in the complete executions that 
do not happen after $e$ and those that occur after $e$.

The race detection phase works mostly similarly with source-DPOR. The main differences have to do with the fact that
we require the knowledge of the sleep set for every prefix $E'$ and that the concepts of Weak Initials is used instead of
the Initials to determine whether a fragment is going to be inserted at the wakeup tree, which is rooted at the prefix $E'$.

In the exploration phase of a non-maximal execution sequence, the wakeup tree of that sequence is initialized to the 
given $WuT$. If $WuT$ is empty, then an arbitrary enabled process is chosen, in the same way that it would for the non-optimal algorithms.
Afterwards, for every process that exists in $WuT$ the explore function is going to be called recursively, with the appropriate
subtree of $wut(E)$. This guarantees that the complete fragment gets explored. After the recursive call finishes,
the sequences that were explored are removed from the wakeup tree. Sleep sets are handled in a similar way with previous 
algorithms.

As the name suggests, optimal-DPOR is optimal in the sense that it never explores two maximal execution sequences
that belong to the same Mazurkiewicz trace, since it can be proven that no interleaving is sleep-set blocked
\cite{AbdullaAronisJohnssonSagonasDPOR2014}.