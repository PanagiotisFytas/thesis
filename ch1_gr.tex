\chapter {Εισαγωγή}


Ο νόμος του Moore, που πήρε το όνομά τους από τον συνιδρυτή της Intel, Gordon Moore, αναφέρει ότι ο αριθμός των τρανζίστορ που
μπορούν να τοποθετηθούν σε ένα ολοκληρωμένο κύκλωμα διπλασιάζεται περίπου κάθε δυο χρόνιοα. Για δεκαετίες οι κατασκευαστές 
πετύχαιναν την συρρίκνωση των διαστάσεων των chip, επιτρέποντας στο νόμο του Moore να συνεχίζει να επαληθεύεται ενώ οι 
τελικοί καταναλωτές συνέχιζαν να απολαμβάνουν ακόμα πιο ισχυρούς φορητούς υπολογιστές, tablets και έξυπνα τηλέφωνα. Από την
άλλη οι μηχανικοί λογισμικού μπορούσαν απλώς να περιμένουν το νόμο του Moore να συνεχίζει να ισχύει ώστε να μπορούν να κατασκευάσουν
ακόμα πιο απαιτητικά προγράμματα. Παρολ' αυτά περιορισμοί όπως η αύξηση της θερμοκρασία και η συχνότητα των ρολογιών εμποδίζουν
την περεταίρω βελτίωση στην επίδοση του λογισμικού. Προκειμένου οι προγραμματιστές να ικανοποιήσουν την απαίτηση για αποδοτικό
λογισμικό, πρότυπα προγραμματισμού όπως αυτό του ταυτοχρονισμού έχουν γίνει αναγκαία. Η χρήση αυτού του προτύπου όμως δημιουργεί νέες 
προκλήσεις καθώς ο προγραμματισμός γίνεται πιο δύσκολος και πιο επιρρεπής σε λάθη από το ακολουθιακό πρότυπο.
 
Συγκεκριμένα, συχνά προβλήμα που εμφανίζονται στο μοντέλο του ταυτοχρονισμού είναι:
\begin{description}
\item[Race conditions] Όπου μία δρομολόγηση έχει μη αναμενόμενο αποτέλεσμα. 
\item[Deadlocks] Δύο οι περισσότερες διεργασίες σταματούν και περιμένουν η μία την άλλη.
\item[Livelocks] Δύο οι περισσότερες διεργασίες συνεχίζουν να εκτελούνται χωρίς να σημειώνουν πρόοδο.
\item[Resource starvations] Δύο οι περισσότερες διεργασίες σταματούν να εκετελούνται περιμένοντας για πόρους.
\end{description}
Το χειρότερο όμως είναι ότι αυτάτ τα προβλήματα χαρακτηρίζονται συνήθως ως Heisenbugs
Δηλαδή, μπορεί να αλλάξουν συμπεριφορά ή να εξαφανιστούν τελείως όταν κάποιος προσπαθήσει να τα απομονώσει καθώς 
σχετίζονται άμεσα με τη σειρά που οι διεργασίες εκτελούνται.

\section{Δοκιμή,Έλεγχος και Επαλήθευση Ταυτόχρονων Προγραμμάτων}

Η δοκιμή και η επαλήθευση ταυτόχρονων προγραμμάτων είναι μια απαιτητική διαδικασία. Μια τεχνική που χρησιμοποιείται για
την εξερεύνηση του χώρου καταστάσεων είναι ο έλεγχος του μοντέλου (model checking)~\cite{WikipediaModelChecking}. To
model checking είναι μια μέθοδος για την τυπική επιβεβαίωση ταυτόχρονων προγραμμάτων και συστημάτων μέσω απαιτήσεων για
το σύστημα εκφρασμένων σε λογική φόρμουλα και την χρήση αποδοτικών αλγορίθμων που μπορούν να ελέγξουν το ορισθέν μοντέλο
όπως έχει οριστεί από το σύστημα ώστε να ελεγχθεί αν οι απαιτήσεις εξακολουθούν να ισχύουν. Το μεγάλο πρόβλημα με τα
εργαλία που κάνουν model checking είναι ότι πρέπει να αντιμετωπίσουν την εκθετική αύξηση του χώρου καταστάσεων καθώς
ένας μεγάλος αριθμός καταστάσεων πρέπει να αποθηκευτεί. Πολλές τεχνικές έχουν προταθεί προκειμένου να αντιμετωπιστεί
αυτό το πρόβλημα. To Stateless Model Checking, για παράδειγμα, αποφεύγει την αποθήκευση καθολικών καταστάσεων. Αυτή η
τεχνική για παράδειγμα έχει υλοποιηθεί σε εργαλεία όπως το Verisoft \cite{Godefroid:2005:SMC:1084665.1084674}.
Ακόμα, αυτή η τεχνική πάσχει από συνδυαστική έκρηξη. Ωστόσο, διαφορετικές παρεμβολές που μπορούν να ληφθούν μεταξύ τους από την
 εναλλαγή γειτονικών και ανεξάρτητων βημάτων εκτέλεσης μπορεί να θεωρηθεί ισοδύναμη.
Οι αλγόριθμοι \cite{Godefroid1996, POR, 10.1007/3-540-53863-1_36} partial order reduction (POR) χρησιμοποιούν αυτή την παρατήρηση για να μειώσουν με επιτυχία
το μέγεθος του διερευνηθέντος χώρου κατάστασης. Οι Dynamic Partial Order Reduction
(DPOR) αλγόριθμοι \cite{FlanaganDPOR, AbdullaAronisJohnssonSagonasDPOR2014} καταφέρνουν να επιτύχουν μια ακόμα αυξημένη μείωση,
με την ανίχνευση των εξαρτήσεων με μεγαλύτερη ακρίβεια. Οι τεχνικές DPOR έχουν εφαρμοστεί με επιτυχία σε εργαλεία όπως το Concuerror
\cite{6569727, Gotovos:2011:TDC:2034654.2034664}, Nidhugg \cite{Abdulla:2015:SMC:2945565.2945622}, Inspect \cite{Yang:2007:DDP:1770532.1770541}
και Eta \cite{simsa2011efficient}.

Ο παραλληλισμός των αλγορίθμων DPOR είναι απαραίτητος προκειμένου να καταστούν κλιμακωτές στις σύγχρονες αρχιτεκτονικές υπολογιστών, αλλά και δυνητικά
να επιτύγχουν σημαντικές επιταχύνσεις που θα ανακουφίσουν την επίδραση της εκθετικής έκρηξης χώρου κατάστασης. Η παραλληλισμός των αλγορίθμων DPOR που χρησιμοποιούν σταθερά σύνολα \cite{FlanaganDPOR, Lei:2006:RTC:1248722.1248743, 10.1007/3-540-53863-1_36} έχει ήδη γίνει,
έχουν εξεταστεί και έχουν εκτελεστεί παράλληλες εκδόσεις για το Inspect \cite{yang2008inspect} και Eta
\cite{Simsa2012ScalableDP}.

\section{Σκοπός της παρούσας εργασίας}

Σε αυτή τη διατριβή, θα επικεντρωθούμε στον παραλληλισμό του Concuerror, ενός stateless model checker
που χρησιμοποιείται για τη δοκιμή ταυτόχρονων προγραμμάτων Erlang. Συγκεκριμένα, πρόκειται να:

\begin{itemize}
\item Αναπτύξουμε παράλληλες εκδόσεις για δύο αλγορίθμους DPOR: source-DPOR \cite{AbdullaAronisJohnssonSagonasDPOR2014} και optimal-DPOR \cite{AbdullaAronisJohnssonSagonasDPOR2014}.
\item Εφαρμόσουμε αυτούς τους παράλληλους αλγορίθμους στο Concuerror.
\item Αξιολογήσουμε την απόδοση της εφαρμογής μας.

\end{itemize}

\section{Overview}

Στο Κεφάλαιο \ref{sec:background} παρέχουμε βασικές πληροφορίες για την Erlang, τον Concuerror και την αφαίρεση που χρησιμοποιούνται για το μοντέλο των ταυτόχρονων συστημάτων. Στο Κεφάλαιο \ref{dpor} περιγράφουμε τους αλγόριθμους source-DPOR και optimal-DPOR. Στα Κεφάλαια \ref{paradpor} και \ref{paradpor_opt} παρουσιάζουμε την παράλληλη έκδοση που έχουμε αναπτύξει για τον source-DPOR και optimal-DPOR, αντίστοιχα. Στο κεφάλαιο
\ref{conc_mods} περιγράφουμε τα βασικά προβλήματα που είχαμε με την εφαρμογή των αλγορίθμων στον Concuerror.
Στο Κεφάλαιο \ref{perfresults} παρουσιάζουμε και αξιολογούμε την απόδοση που επιτεύχθηκε με την εφαρμογή μας. Τέλος, στο κεφάλαιο \ref{conclusion}
συνοψίζουμε τα προηγούμενα κεφάλαια και εξετάζουμε πιθανές επεκτάσεις του έργου μας.