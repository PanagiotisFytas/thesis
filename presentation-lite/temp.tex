\begin{frame} {Stateless Model Checking}


\tracelong{../img/initstateless.png}{State space of a program.}

\end{frame}

\begin{frame} {Partial Order Reduction}

Partial Order Reduction tries to avoid exploring equivalent interleavings through race detection.

\trace{../img/initpor.png}{Interleavings explored by POR}

\end{frame}

\begin{frame} {Partial Order Reduction}
\begin{itemize}[<+->]
    \item Static Partial Order Reduction: Dependencies are tracked before execution, by statically analyzing the code.
    \item Dynamic Partial Order Reduction (DPOR): Actual dependencies are observed during runtime.
\end{itemize}

\end{frame}


\begin{frame} {Terminology}

\begin{itemize}[<+->]
    \item The complete execution of a process $p$ splits into different execution steps, which are
    to be executed atomically. Those steps are referred to as $events$. Each event must be deterministic.
    \item An execution sequence $E$ of a system is a finite sequence of execution steps
    of its processes that is performed from a unique initial state.
    \item An execution sequence $E$ is uniquely characterized by the sequence of processes
    that perform steps in $E$. For instance, $p.p.q$ denotes the execution
    sequence where first $p$ performs two steps, followed by a step of $q$.
    \item The sequence of processes that perform steps in $E$ also uniquely
    determines the global state of the system after $E$.
\end{itemize}

\end{frame}


\begin{frame} {Erlang}

Erlang is a programming language that has build-in support for:

\begin{itemize}[<+->]
  \item Concurrency
  \item Distribution
\end{itemize}

\end{frame}
\begin{frame} {Concurrent Erlang}

In the core of concurrent Erlang are its lightweight processes which:

\begin{itemize}[<+->]
  \item are implemented by the Erlang VM's (BEAM) runtime system, not by operating system threads.
  \item are uniquely identified by their Pid (Process Identifier).
  \item communicate with each other mainly though message passing (actor model).

\end{itemize}
\pause

Erlang also has support for shared memory through the build-in ETS (Erlang Term Storage) module.

\end{frame}

\begin{frame} {Distributed Erlang}

\begin{itemize}[<+->]
\item An Erlang node is an Erlang runtime system containing a complete virtual machine which
contains its own address space and set of processes. 
\item Erlang nodes can communicate over the network, remotely or through the localhost. 
\item Pids continue to be unique over different
nodes (globally).
\item Inside two different nodes, two different processes can have the
same local Pid.
\item All primitives operate over the
network similarly as they would on the same node.

\end{itemize}

\end{frame}


\begin{frame} {Concuerror}

Concuerror is a tool that uses various stateless model checking techniques
in order to systematically 
test an Erlang program, with the aim of detecting and reporting concurrency-related runtime errors.
Its main components are:

\begin{itemize}[<+->]
\item Instrumenter:
  \begin{itemize}
  \item adds preemption points to various points in the code of a tested program.
  \item makes it possible to produce specific interleaving.
  \end{itemize}

\item Scheduler:
  \begin{itemize}
  \item uses mainly source-DPOR or optimal-DPOR, to determine which interleavings need to be checked.
  \item controls the execution of the processes to produce those interleavings.
  \end{itemize}
\end{itemize}

 
\end{frame}

\begin{frame}{Aim of Thesis}

\begin{itemize}[<+->]
    \item Develop parallel version for source-DPOR algorithm.
    \item Develop parallel version for optimal-DPOR algorithm.
    \item Implement those parallel algorithms in Concuerror.
    \item Evaluate the performance of our implementation.
\end{itemize}

\end{frame}

\section{Sequential DPOR Algorithms}


\begin{frame} {General DPOR Concepts}

DPOR: performs a DFS using a backtrack set. Two main techniques:
\begin{itemize}[<+->]
    \item Persistent sets: only a provably sufficient subset of the enabled processes gets explored.
    \item Sleep sets: prevents redundant exploration.
\end{itemize}

\end{frame}

\begin{frame} {Optimality in DPOR}

\begin{itemize}[<+->]
\item A DPOR algorithm is optimal if, for every maximal execution sequence $E$, it explores exactly one interleaving from 
$[E]_{\simeq}$.
\item Persistent-set based DPOR is not optimal.
\end{itemize}
\end{frame}


\begin{frame}{Source Sets}

%%%\begin{definition}[Initials after an execution sequence $E.w$, $I_{[E]}(w)$]
%%%    $p \in I_{[E]}(w)$ if and only if there is a sequence $w'$ such that $E.w \simeq E.p.w'$.
%%%\end{definition}


\begin{definition}[Weak Initials after an execution sequence $E.w$, $WI_{[E]}(w)$]

$p \in WI_{[E]}(w)$ if and only if there are sequences $w'$ and $v$ such
that $E.w.v \simeq E.p.w'$.

\end{definition}

\begin{definition}[Source Sets]
Let $E$ be an execution sequence,
and let $W$ be a set of sequences, such that $E.w$ is an execution
sequence for each $w \in W$. A set $T$ of processes is a source set for
$W$ after $E$ if for each $w \in W$ we have $WI_{[E]}(w) \cap T  \neq \emptyset$.
\end{definition}
\\

\item A source set of a sequence $w$ at an execution sequence $E$, contains the process that can perform the  ``first steps" 
after $E$ that can reproduce sequences equivalent to $w$.

\end{frame}


\begin{frame}{Source-DPOR}

\begin{figure}
    
\scalebox{1}{
\SetKwProg{Fn}{Function}{}{}
\SetKwHangingKw{Let}{let}
\begin{algorithm}[H]
    \caption{Source-DPOR}
    \label{Source}
    \Fn{Explore($E$,$Sleep$)}{
        \If{$\exists p \in (enabled(s_{[E]}) \backslash Sleep)$}{
            $backtrack(E) :={p}$\;
            \While{$\exists p \in (backtrack(E) \backslash Sleep)$}{
                \ForEach{$e \in dom(E)$ such that $e \lesssim_{E.p} next_{[E]}(p)$}{
                    \Let{$E' = pre(E,e)$}
                    \Let{$u = notdep(e,E).p$}
                    \If{$I_{[E']}(u) \cap backtrack(E') = \emptyset$}{
                        add some $q' \in I_{[E']}(u) \text{ to } backtrack(E')$\;
                    }
                }
                \Let{$Sleep' := \{q \in Sleep \mid E \models p \diamondsuit q \} $}
                $Explore(E.p, Sleep')$\;
                add $p$ to $Sleep$\;

            }
        }
    }
\end{algorithm}
}
\end{figure}

\end{frame}


\begin{frame}{Source-DPOR}

Source-DPOR in action:

\tracelong{../img/source1.png}{Interleavings explored by the source-DPOR.}

\end{frame}


\begin{frame}{Motivation for Wakeup Trees}

\begin{columns}
\column{0.75\textwidth}
\begin{figure}
    
\scalebox{0.7}{
\SetKwProg{Fn}{Function}{}{}
\SetKwHangingKw{Let}{let}
\begin{algorithm}[H]
    \label{Source}
    \Fn{Explore($E$,$Sleep$)}{
        \If{$\exists p \in (enabled(s_{[E]}) \backslash Sleep)$}{
            $backtrack(E) :={p}$\;
            \While{$\exists p \in (backtrack(E) \backslash Sleep)$}{
                \ForEach{$e \in dom(E)$ such that $e \lesssim_{E.p} next_{[E]}(p)$}{
                    \Let{$E' = pre(E,e)$}
                    \HiLi\Let{$u = notdep(e,E).p$}
                    \If{$I_{[E']}(u) \cap backtrack(E') = \emptyset$}{
                        add some $q' \in I_{[E']}(u) \text{ to } backtrack(E')$\;
                    }
                }
                \Let{$Sleep' := \{q \in Sleep \mid E \models p \diamondsuit q \} $}
                $Explore(E.p, Sleep')$\;
                add $p$ to $Sleep$\;

            }
        }
    }
\end{algorithm}
}
\end{figure}
\column{0.25\textwidth}

\begin{itemize}

\item $E'.u$ needs to be explored.
\item But only a single process gets added to the backtrack.
\item This may lead to sleep-set blocking.
\end{itemize}

\end{columns}


\end{frame}


\begin{frame} {Wakeup Trees}


\begin{definition}{(Ordered Tree)}\label{def:Ordered}\\
An $ordered$ $tree$ is a pair $\langle B , \prec \rangle$, where B (the set of nodes) is a finite prefix-closed
    set of sequences of processes with the empty sequence $\langle\rangle$ being the root.
    The children of a node $w$, of form $w.p$ for some set of processes $p$, are ordered by $\prec$. 
    In $\langle B , \prec \rangle$, such an ordering between children has been extended to the total 
    order $\prec$ on $B$ by letting $\prec$ be the induced post-order relation between the nodes in $B$.
    This means that if the children $w.p_1$ and $w.p_2$ are ordered as $w.p_1 \prec w.p_2$,
    then $w.p_1 \prec w.p_2 \prec w $ in the induced post-order.
\end{definition}

\begin{definition}{(Wakeup Tree)}\\
    Let $E$ be an execution sequence and $P$ a set of processes. a $wakeup$ $tree$ after $\langle E , P \rangle$
    is an ordered tree $\langle B , \prec \rangle$, for which the following properties hold:
    \begin{itemize}
        \item $WI_{[E]}(w) \cap P = \emptyset$ for every leaf $w$ of $B$.
        \item For every node in $B$ of the form $u.p$ and $u.w$ such that $u.p \prec u.w$ and $u.w$ is a leaf
        the $p \not \in WI_{[E.u]}(w)$ property must hold true.
    \end{itemize}
\end{definition}

\end{frame}


\begin{frame} {Wakeup Trees}

Intuitively, wakeup trees hold sequences of processes that
need to be explored. They can visualized in the following way:

\pause

\tracelonglong{WuTExample.png}{Visualizing wakeup trees.}

\end{frame}

\begin{frame} {Wakeup Trees}

Inserting new sequences ($insert_{[E]}(w,\langle B , \prec \rangle)$) in the wakeup tree has the following properties:

\begin{itemize}[<+->]

    \item $insert_{[E]}(w,\langle B , \prec \rangle)$ is also a wakeup tree after $\langle E , P \rangle$.
    \item Any leaf of $\langle B , \prec \rangle$ remains a leaf of $insert_{[E]}(w,\langle B , \prec \rangle)$.
    \item $insert_{[E]}(w,\langle B , \prec \rangle)$ , while it may not contain $w$ as a leaf, it contains a  leaf $u$ with $u \sim_{[E]} w$. 

\end{itemize}


\end{frame}


\begin{frame} [shrink=28]{Optimal-DPOR}

\SetKwProg{Fn}{Function}{}{}
\SetKwHangingKw{Let}{let}
\begin{algorithm}[H]
    \caption{Optimal-DPOR}
    \label{optimal}
    \Fn{Explore($E$,$Sleep$,$WuT$)}{
        \uIf{$enabled(s_{[E]}) = \emptyset$}{
            \ForEach{$e,e' \in dom(E)$ such that ($e \lesssim_{E} e'$)}{
                \Let{$E' = pre(E,e)$}
                \Let{$v = notdep(e,E).proc(e')$}
                \If{$sleep(E') \cap WI_{[E']}(v)= \emptyset$}{
                    $insert_{[E']}(v,wut(E'))$\;
                }
            }
        }
    \Else {
        \uIf{$WuT \not = \langle \{ \langle \rangle \}, \emptyset \rangle$}{
            $wut(E) := WuT$\;
        }
        \Else {
            choose $p \in enabled(s_{[E]})$\;
            $wut(E) := \langle \{ p \}, \emptyset \rangle $\;
        }
        $sleep(E) := Sleep$\;
        \While{$\exists p \in wut(E)$}{
            \Let{ $p = min_{\prec}\{ p \in wut(E)\}$}
            \Let{$Sleep' := \{q \in sleep(E) \mid E \models p \diamondsuit q \} $}
            \Let{$WuT' = subtree(wut(E), p)$}
            $Explore(E.p, Sleep', WuT')$\;
            add $p$ to $sleep(E)$\;
            remove all sequences of form $p.w$ from $wut(E)$\;
        }
    }
}
\end{algorithm}
\end{frame}


\begin{frame} {Optimal-DPOR example}
\tracelong{../img/opt_seq/11.png}{Optimal-DPOR exlporation.}
\end{frame}

\begin{frame} {Optimal-DPOR example}
\tracelong{../img/opt_seq/55.png}{Optimal-DPOR exlporation.}
\end{frame}

\section{Parallelizing source-DPOR and optimal-DPOR}

\begin{frame} {Parallel source-DPOR}

Lets assume that we have an execution sequence $E$ and that $p$ and $q$ are processes in $backtrack(E)$. We could:

\begin{itemize}[<+->]
  \item Assign the exploration of $E.p$ and $E.q$ to different workers-schedulers.
  \item The exploration frontier gets updated in a non-local manner.
  \item Who would be responsible for exploring $E.r$?
  \item How do we now that the subtrees are balanced?
\end{itemize}
 
\end{frame}

\begin{frame} {Basic Idea}

We are going to use a centralized Controller who will be responsible for:

\begin{itemize}[<+->]
  \item assigning work to different schedulers, by partitioning Execution sequences.
  \item resolving conflicts.
\end{itemize}
\end{frame}

\begin{frame} {Basic Idea}
The Controller will keep track of:

\begin{itemize}[<+->]
  \item the Frontier of our search: the set of the execution sequences assigned to different schedulers.
  \item the Execution Tree: the subtree of the state space that is currently being explored i.e., the Frontier combined in a tree form.
  A path from the root of tree to a leaf uniquely determines an execution sequence.
\end{itemize}

\end{frame}

\begin{frame} {Basic Idea}

Also, lets introduce the concept of Execution Tree node ownership:

\begin{itemize}[<+->]
  \item A scheduler exclusively \textbf{owns} a node of the state space if: 
    \begin{itemize}
    \item it is an assigned backtrack entry
    \item it is a descendant of a node that the scheduler owns.
    \end{itemize}
  \item All other nodes, are considered to have a \textbf{disputed} ownership.
\end{itemize}

\end{frame}

\begin{frame} {Controller Logic}

\SetKwProg{Fn}{Function}{}{}
\SetKwHangingKw{Let}{let}
\begin{algorithm}[H]
    \label{controllerloop}
    \Fn{controller\_loop($N$, $Budget$, $Schedulers$)}{
        $E_0 \leftarrow$ an arbitrary initial execution sequence\;
        $Frontier \leftarrow[E_0]$\;
        $T \leftarrow$ an execution tree rooted at $E_0$\;
        \While{$Frontier \neq \emptyset$} {
            $Frontier \leftarrow partition(Frontier, N)$\;
            \While{exists an idle scheduler $S$ and an unassigned execution sequence $E$ in $Frontier$}{
                $E_c \leftarrow$ a copy of $E$\;
                mark $E$ as assigned in $Frontier$\;
                $spawn(S, explore\_loop(E_c, Budget))$\;
            }
            $Frontier,T \leftarrow wait\_scheduler\_response(Frontier, T)$\;
        }
           
    }

\end{algorithm}

\end{frame}

\begin{frame} {Frontier Partitioning}

\SetKwProg{Fn}{Function}{}{}
\SetKwHangingKw{Let}{let}
\begin{algorithm}[H]
    \label{partition}
    \Fn{partition($Frontier$, $N$)}{
        \For{all E $\in$ Frontier}{
            \While{$total\_backtrack\_entries(E) > 1$ \textbf{and} $size(Frontier) < N$}{
                $E' \leftarrow $ \textnormal{the smallest prefix of $E$
                that has a backtrack entry} \;
                $p \leftarrow \textnormal{ a process} \in backtrack(E') $\;
                $E_c' \leftarrow \textnormal{ a copy of } E'$\;
                \textnormal{remove $p$ from $ backtrack(E')$}\;
                \textnormal{add $p$ to $ sleep(E')$}\;
                \textnormal{add $ backtrack(E') $ to $ sleep(E_c')$}\;
                \textnormal{add $E_c'$ to $ Frontier$}\;
            }
        }
        \Return $Frontier$\;
    }
\end{algorithm}

\end{frame}



\begin{frame} {Scheduler Exploration Loop}

\SetKwProg{Fn}{Function}{}{}
\SetKwHangingKw{Let}{let}
\begin{algorithm}[H]
    \label{explore_loop}
    \Fn{explore\_loop($E_0$, $Budget$)}{
        $StartTime \leftarrow get\_time()$\;
        $ E \leftarrow E_0$\;
        \Repeat{$CurrentTime - StartTime > Budget \textbf{ or } size(E) \leq size(E_0)$}{
            $ E' \leftarrow explore(E)$\;
            $ E' \leftarrow plan\_more\_interleavings(E') $\;
            $ E \leftarrow get\_next\_execution\_sequence(E')$\;
            $CurrentTime \leftarrow get\_time()$\;
        }
        \textnormal{\textbf{send}  $E$ to Controller} \;
    }
\end{algorithm}

\end{frame}

\begin{frame} {Parallel source-DPOR}

\SetKwProg{Fn}{Function}{}{}
\SetKwHangingKw{Let}{let}
\begin{algorithm}[H]
    \caption{Handling Scheduler Response}
    \label{response}
    \Fn{wait\_scheduler\_response(Frontier, T)}{
        \textnormal{\textbf{receive} $E$ from a scheduler}\;
        remove $E$ from $Frontier$\;
        $E',T \leftarrow update\_execution\_tree(E, T)$\;
        \If{$E'$ has at least one backtrack entry}{
                \textnormal{add $E'$ to $Frontier$}\;
        }
        \Return $Frontier,T$;
    }
\end{algorithm}


\end{frame}
\begin{frame}{Resolving conflicts}

$update\_execution\_tree(E, T)$:

\begin{itemize}
\item iterates
over the execution sequence and the execution tree simultaneously .
\item backtrack entries of E that are not found in the execution tree, are added to it.
\item ownership is claimed over them.


\end{itemize}

\end{frame}