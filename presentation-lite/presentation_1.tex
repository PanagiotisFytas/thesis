%%%%%%%%%%%%%%%%%%%%%%%%%%%%%%%%%%%%%%%%%
% Beamer Presentation
% LaTeX Template
% Version 1.0 (10/11/12)
%
% This template has been downloaded from:
% http://www.LaTeXTemplates.com
%
% License:
% CC BY-NC-SA 3.0 (http://creativecommons.org/licenses/by-nc-sa/3.0/)
%
%%%%%%%%%%%%%%%%%%%%%%%%%%%%%%%%%%%%%%%%%

%----------------------------------------------------------------------------------------
%	PACKAGES AND THEMES
%----------------------------------------------------------------------------------------

\documentclass[9pt]{beamer}

\mode<presentation> {

% The Beamer class comes with a number of default slide themes
% which change the colors and layouts of slides. Below this is a list
% of all the themes, uncomment each in turn to see what they look like.

%\usetheme{default}
%\usetheme{AnnArbor}
%\usetheme{Antibes}
%\usetheme{Bergen}
%\usetheme{Berkeley}
%\usetheme{Berlin}
%\usetheme{Boadilla}
%\usetheme{CambridgeUS}
%\usetheme{Copenhagen}
%\usetheme{Darmstadt}
%\usetheme{Dresden}
%\usetheme{Frankfurt}
%\usetheme{Goettingen}
%\usetheme{Hannover}
%\usetheme{Ilmenau}
%\usetheme{JuanLesPins}
%\usetheme{Luebeck}
%\usetheme{Madrid}
%\usetheme{Malmoe}
%\usetheme{Marburg}
%\usetheme{Montpellier}
%\usetheme{PaloAlto}
%\usetheme{Pittsburgh}
%\usetheme{Rochester}
\usetheme{Singapore}
%\usetheme{Szeged}
%\usetheme{Warsaw}

% As well as themes, the Beamer class has a number of color themes
% for any slide theme. Uncomment each of these in turn to see how it
% changes the colors of your current slide theme.

%\usecolortheme{albatross}
%\usecolortheme{beaver}
%\usecolortheme{beetle}
%\usecolortheme{crane}
%\usecolortheme{dolphin}
%\usecolortheme{dove}
%\usecolortheme{fly}
%\usecolortheme{lily}
%\usecolortheme{orchid}
%\usecolortheme{rose}
%\usecolortheme{seagull}
%\usecolortheme{seahorse}
%\usecolortheme{whale}
%\usecolortheme{wolverine}

%\setbeamertemplate{footline} % To remove the footer line in all slides uncomment this line
%\setbeamertemplate{footline}[page number] % To replace the footer line in all slides with a simple slide count uncomment this line

%\setbeamertemplate{navigation symbols}{} % To remove the navigation symbols from the bottom of all slides uncomment this line
}

\usepackage{graphicx} % Allows including images
\usepackage{booktabs} % Allows the use of \toprule, \midrule and \bottomrule in tables
\usepackage[cm-default]{fontspec}
\usepackage{unicode-math}
\usepackage{mathtools}
\usepackage{amsmath}
\usepackage{amssymb}
\usepackage{fancyvrb}
\usepackage{float}
\usepackage{color}
%\usepackage[table,xcdraw]{xcolor}
\usepackage{siunitx}
\usepackage{multirow}
\usepackage{colortbl}
\usepackage{subcaption}
\usepackage{booktabs}
\usepackage{pdflscape}
\usepackage{listings}
\usepackage{algorithm2e}
\usepackage{xcolor}

\def\HiLi{\leavevmode\rlap{\hbox to \hsize{\color{yellow!50}\leaders\hrule height .8\baselineskip depth .5ex\hfill}}}


\usepackage{graphicx}



\usepackage{fontspec}
\usepackage{xunicode}
\usepackage{xltxtra}
%\setromanfont{Comic Sans MS}
\setromanfont{CMU Serif}
\setsansfont{CMU Sans Serif}
\setmonofont{FreeMono}
%\usepackage[english,greek{babel}
%\usepackage[iso-8859-7]{inputenc}
%\setmainfont{Minion Pro} % substitute with any font that exists on your system
%\setsansfont{Myriad Pro} % substitute with any font that exists on your system
%\setmonofont{Consolas} % substitute with any font that exists on your system
\AtBeginSection[]{
  \begin{frame}
  \vfill
  \centering
  \begin{beamercolorbox}[sep=8pt,center,shadow=true,rounded=true]{title}
    \usebeamerfont{title}\insertsectionhead\par%
  \end{beamercolorbox}
  \vfill
  \end{frame}
}



\setbeamertemplate{theorems}[numbered]

\newcommand{\img}[2]{
    \begin{center}
\includegraphics[scale=#1]{#2}
\end{center}
}


\graphicspath{ {img/} }
\newcommand{\trace}[2]{
\begin{figure}[H]
\centering
\includegraphics[scale=0.3]{#1}
\caption{#2}
\label{#2}
\end{figure}
}

\newcommand{\tracelong}[2]{
\begin{figure}[H]
\centering
\includegraphics[scale=0.2]{#1}
\caption{#2}
\label{#2}
\end{figure}
}

\newcommand{\graph}[2]{
\begin{figure}[H]
\centering
\includegraphics[scale=0.5]{#1}
\caption{#2}
\label{#2}
\end{figure}
}

\newcommand{\mediumGraph}[2]{
  \begin{figure}[H]
    \centering
    \includegraphics[scale=0.2]{#1}
    \caption{#2}
    \label{#2}
    \end{figure}
    }

\lstset{basicstyle=\ttfamily,columns=fullflexible}

\newcommand{\Code}[2]{
  \begin{minipage}{\linewidth}
  \lstinputlisting[basicstyle=\ttfamily\scriptsize,caption=#2,captionpos=b]{#1}
  \label{#2}
  \end{minipage}
 }

\newcommand{\Output}[2]{
  %%\BVerbatimInput[fontsize=\tiny]{#1}
  \begin{minipage}{0.85\textwidth}
  \lstinputlisting[label={#2},numbers=none,frame=none,caption=#2]{#1}
  \end{minipage}
  %%\caption{#2}
 }

\newcommand{\Side}[5]{
  %\begin{figure}
    \begin{minipage}{0.5\textwidth}
      \lstinputlisting[frame=none, numbers=none,caption={[#2]}]{#1}
    \end{minipage}
    %%\begin{minipage}{0.1\textwidth}
    %%  \includegraphics[scale=0.1]{arrow.pdf}
    %%\end{minipage}
    \begin{minipage}{0.5\textwidth}
      %%\VerbatimInput{#2}
      \lstinputlisting[frame=none, numbers=none,caption={[#4]}]{#3}
    \end{minipage}
    \captionof{figure}{#5}
    \label{#5}
    
     % \caption{#5}
     % \label{#5}
  %\end{figure}
} 

\newcommand{\bigtabular}[2]{
 \begin{table} 
   \resizebox{\linewidth}{!}{
      \input{#1}
    }
    \caption{#2}
    \label{#2}
 \end{table}
}

\newcommand{\landscapetabular}[2]{
\begin{landscape}
 \begin{table} 
   \resizebox{\linewidth}{!}{
      \input{#1}
    }
    \caption{#2}
    \label{#2}
 \end{table}
\end{landscape}
}

\newcommand{\smalltabular}[2]{
  \begin{table} 
    \resizebox{0.8\textwidth}{!}{\begin{minipage}{\textwidth}
     \input{#1}
     \caption{#2}
     \label{#2}
    \end{minipage}}
  \end{table} 
}

\newtheorem{thm}{Θεώρημα}[section]
\newtheorem{lem}[thm]{Λήμμα}
\newtheorem{por}[thm]{Πόρισμα}
\newtheorem{defn}[thm]{Ορισμός}
%----------------------------------------------------------------------------------------
%	TITLE PAGE
%----------------------------------------------------------------------------------------

\title[Short title]{Parallelizing Concuerror: A Dynamic Partial Order Reduction Testing Tool for Erlang Programs} % The short title appears at the bottom of every slide, the full title is only on the title page

\author{Φυτάς Παναγιώτης} % Your name
\institute[NTUA] % Your institution as it will appear on the bottom of every slide, may be shorthand to save space
{
ΣΗΜΜΥ - ΕΜΠ \\ % Your institution for the title page
\medskip
\textit{03112113} % Your email address
}
\date{} % Date, can be changed to a custom date
\setcounter{subsection}{1}


\begin{document}

\begin{frame}
\titlepage % Print the title page as the first slide
\end{frame}

\begin{frame}
\frametitle{Summary} % Table of contents slide, comment this block out to remove it
\tableofcontents % Throughout your presentation, if you choose to use \section{} and \subsection{} commands, these will automatically be printed on this slide as an overview of your presentation
\end{frame}

%----------------------------------------------------------------------------------------
%	PRESENTATION SLIDES
%----------------------------------------------------------------------------------------

%------------------------------------------------
%------------------------------------------------

%\subsection{} % A subsection can be created just before a set of slides with a common theme to further break down your presentation into chunks

\begin{frame}{Aim of Thesis}

\begin{itemize}[<+->]
    \item Develop parallel version for source-DPOR algorithm.
    \item Develop parallel version for optimal-DPOR algorithm.
    \item Implement those parallel algorithms at Concuerror.
    \item Evaluate the performance of our implementation.
\end{itemize}

\end{frame}

\section{Background}
%% \frame{\sectionpage}

\begin{frame}{Concurrent Computing}
\emph{Concurrent Computing} is a form of computing in which several computations are executed during
overlapping time periods concurrently, instead of sequentially (one completing before the next starts).
\\

Essential because:
\pause
\begin{itemize}[<+->]
    \item Multithreaded computer architectures 
    \item Availability of services
    \item Controllability 
\end{itemize}

\end{frame}

\begin{frame}{Concurrent Computing}

However, concurrency is difficult to get right:
\pause
\begin{itemize}[<+->]
    \item Deadlocks
    \item Race conditions
    \item Resource starvation 
    \item Scheduling non-determinism
    \begin{itemize}[<+->]
        \item Interleaving non-determinism
        \item Timing non-determinism
    \end{itemize}
\end{itemize}
\pause
Errors can occur only on specific rare interleavings. 
Detecting and reproducing bugs becomes extremely hard (Heisenbugs).

\end{frame}

\begin{frame}{Modeling our Problem}

\begin{itemize}[<+->]
    \item An interleaving represents a scheduling of the concurrent program.
    \item The state-space is the set of all possible interleavings.
    \item In order to verify a program, the complete state-space must be explored.
\end{itemize}
    

\end{frame}

\begin{frame} {Stateless Model Checking}

\begin{itemize}[<+->]
    \item Stateless Model Checking systematically explores all possible interleavings.
    \item Combinatorial state-space explosion.
    \item Different interleavings can be equivalent.
\end{itemize}
\pause
\tracelong{../img/initstateless.png}{Stateless Model Checking Example}

\end{frame}

\begin{frame} {Partial Order Reduction}

Partial Order Reduction tries to avoid exploring equivalent interleavings through race detection.

\trace{../img/initpor.png}{Partial Order Reduction Example}

\end{frame}

\begin{frame} {Partial Order Reduction}
\begin{itemize}[<+->]
    \item Static Partial Order Reduction: Dependencies are tracked before execution, by statically analazing the code.
    \item Dynamic Partial Order Reduction (DPOR): Actual dependencies are observed during runtime.
\end{itemize}

\end{frame}

\begin{frame} {General DPOR}

DPOR: performs a DFS using a backtrack set. Exploration is based on two techniques:
\begin{itemize}[<+->]
    \item Persistent sets: only a provably sufficient subset of the enabled processes gets explored.
    \item Sleep sets: contain processes, whose exploration would be redundant, 
    preventing equivalent interleavings from being fully explored.
\end{itemize}

\end{frame}

%% \subsection{Dynamic Partial Order Reduction}

\begin{frame} {Important Concepts}

\begin{itemize}[<+->]
    \item The complete execution of a process $p$ splits into different execution steps, which are
    to be executed atomically. Those steps are referred to as $events$. Each event must be deterministic.
    \item An execution sequence $E$ of a system is a finite sequence of execution steps
    of its processes that is performed from a unique initial state.
    \item We use $E \simeq E'$ to denote that $E$ and $E'$ are
    equivalent, and $[E]_{\simeq}$ to denote the equivalence class of E.
\end{itemize}

\end{frame}

\begin{frame}{Source Sets}

\begin{definition}[Initials after an execution sequence $E.w$, $I_{[E]}(w)$]
    $p \in I_{[E]}(w)$ if and only if there is a sequence $w'$ such that $E.w \simeq E.p.w'$.
\end{definition}


\begin{definition}[Weak Initials after an execution sequence $E.w$, $WI_{[E]}(w)$]

$p \in WI_{[E]}(w)$ if and only if there are sequences $w'$ and $v$ such
that $E.w.v \simeq E.p.w'$.

\end{definition}
    
\end{frame}


\begin{frame}{Source Sets}

\begin{definition}[Source Sets]
Let $E$ be an execution sequence,
and let $W$ be a set of sequences, such that $E.w$ is an execution
sequence for each $w \in W$. A set $T$ of processes is a source set for
$W$ after $E$ if for each $w \in W$ we have $WI_{[E]}(w) \cap T  \neq \emptyset$.
\end{definition}

\end{frame}

\begin{frame}{Source-DPOR}

\begin{figure}
    
\scalebox{0.7}{
\SetKwProg{Fn}{Function}{}{}
\SetKwHangingKw{Let}{let}
\begin{algorithm}[H]
    \caption{Source-DPOR}
    \label{Source}
    \Fn{Explore($E$,$Sleep$)}{
        \If{$\exists p \in (enabled(s_{[E]}) \backslash Sleep)$}{
            $backtrack(E) :={p}$\;
            \While{$\exists p \in (backtrack(E) \backslash Sleep)$}{
                \ForEach{$e \in dom(E)$ such that $e \lesssim_{E.p} next_{[E]}(p)$}{
                    \Let{$E' = pre(E,e)$}
                    \Let{$u = notdep(e,E).p$}
                    \If{$I_{[E']}(u) \cap backtrack(E') = \emptyset$}{
                        add some $q' \in I_{[E']}(u) \text{ to } backtrack(E')$\;
                    }
                }
                \Let{$Sleep' := \{q \in Sleep \mid E \models p \diamondsuit q \} $}
                $Explore(E.p, Sleep')$\;
                add $p$ to $Sleep$\;

            }
        }
    }
\end{algorithm}
}
\end{figure}

\end{frame}

\end{document} 
