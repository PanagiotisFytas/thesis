%%%%%%%%%%%%%%%%%%%%%%%%%%%%%%%%%%%%%%%%%
% Beamer Presentation
% LaTeX Template
% Version 1.0 (10/11/12)
%
% This template has been downloaded from:
% http://www.LaTeXTemplates.com
%
% License:
% CC BY-NC-SA 3.0 (http://creativecommons.org/licenses/by-nc-sa/3.0/)
%
%%%%%%%%%%%%%%%%%%%%%%%%%%%%%%%%%%%%%%%%%

%----------------------------------------------------------------------------------------
%	PACKAGES AND THEMES
%----------------------------------------------------------------------------------------

\documentclass[9pt]{beamer}

\mode<presentation> {

% The Beamer class comes with a number of default slide themes
% which change the colors and layouts of slides. Below this is a list
% of all the themes, uncomment each in turn to see what they look like.

%\usetheme{default}
%\usetheme{AnnArbor}
%\usetheme{Antibes}
%\usetheme{Bergen}
%\usetheme{Berkeley}
%\usetheme{Berlin}
%\usetheme{Boadilla}
%\usetheme{CambridgeUS}
%\usetheme{Copenhagen}
%\usetheme{Darmstadt}
%\usetheme{Dresden}
%\usetheme{Frankfurt}
%\usetheme{Goettingen}
%\usetheme{Hannover}
%\usetheme{Ilmenau}
%\usetheme{JuanLesPins}
%\usetheme{Luebeck}
%\usetheme{Madrid}
%\usetheme{Malmoe}
%\usetheme{Marburg}
%\usetheme{Montpellier}
%\usetheme{PaloAlto}
%\usetheme{Pittsburgh}
%\usetheme{Rochester}
\usetheme{Singapore}
%\usetheme{Szeged}
%\usetheme{Warsaw}

% As well as themes, the Beamer class has a number of color themes
% for any slide theme. Uncomment each of these in turn to see how it
% changes the colors of your current slide theme.

%\usecolortheme{albatross}
%\usecolortheme{beaver}
%\usecolortheme{beetle}
%\usecolortheme{crane}
%\usecolortheme{dolphin}
%\usecolortheme{dove}
%\usecolortheme{fly}
%\usecolortheme{lily}
%\usecolortheme{orchid}
%\usecolortheme{rose}
%\usecolortheme{seagull}
%\usecolortheme{seahorse}
%\usecolortheme{whale}
%\usecolortheme{wolverine}

%\setbeamertemplate{footline} % To remove the footer line in all slides uncomment this line
%\setbeamertemplate{footline}[page number] % To replace the footer line in all slides with a simple slide count uncomment this line

%\setbeamertemplate{navigation symbols}{} % To remove the navigation symbols from the bottom of all slides uncomment this line
}

\usepackage{graphicx} % Allows including images
\usepackage{booktabs} % Allows the use of \toprule, \midrule and \bottomrule in tables
\usepackage[cm-default]{fontspec}
\usepackage{unicode-math}
\usepackage{mathtools}
\usepackage{amsmath}
\usepackage{amssymb}
\usepackage{fancyvrb}
\usepackage{float}
\usepackage{color}
%\usepackage[table,xcdraw]{xcolor}
\usepackage{siunitx}
\usepackage{multirow}
\usepackage{colortbl}
\usepackage{subcaption}
\usepackage{booktabs}
\usepackage{pdflscape}
\usepackage{listings}
\usepackage{algorithm2e}
\usepackage{xcolor}

\def\HiLi{\leavevmode\rlap{\hbox to \hsize{\color{yellow!50}\leaders\hrule height .8\baselineskip depth .5ex\hfill}}}


\usepackage{graphicx}


\usepackage{listings}
\usepackage{lstlinebgrd}
\lstset{
    numbers=left,
    numberstyle=\tiny\color{black},
    basicstyle=\ttfamily\footnotesize,
    basewidth=0.59em,
    keywordstyle=[3]{},
    commentstyle=\itshape\footnotesize,
    tabsize=8,
    frame=single,
    frameround=tttt,
    showstringspaces=false,
    breaklines=false,
    captionpos=b,
    aboveskip=\bigskipamount,
    belowskip=\bigskipamount,
    escapechar=^
}
\lstdefinestyle{custom}{
    numbers=left,
    numberstyle=\tiny\color{black},
    basicstyle=\ttfamily\footnotesize,
    basewidth=0.59em,
    keywordstyle=[3]{},
    commentstyle=\itshape\footnotesize,
    tabsize=8,
    frame=single,
    frameround=tttt,
    showstringspaces=false,
    rulecolor=\color{black},
    breaklines=false,
    captionpos=b,
    aboveskip=\bigskipamount,
    belowskip=\bigskipamount,
    escapechar=^,
    moredelim=**[is][\color{blue}]{@}{@}
}

\lstnewenvironment{code}[2]{
  \nopagebreak
  \lstset{language=Erlang, label={#1}, caption={#2}, style=custom}
}{}

\lstnewenvironment{code_appendix}{
  \nopagebreak
  \lstset{language=Erlang, style=custom, numbers=none,rulecolor=\color{black}}
}{}

\lstnewenvironment{console}[2]{
  \nopagebreak
  \lstset{label={#1}, caption={#2}, numbers=none, style=custom}
}{}
\usepackage{adjustbox}
\newsavebox\listingboxname

\usepackage{fontspec}
\usepackage{xunicode}
\usepackage{xltxtra}
%\setromanfont{Comic Sans MS}
\setromanfont{CMU Serif}
\setsansfont{CMU Sans Serif}
\setmonofont{FreeMono}
%\usepackage[english,greek{babel}
%\usepackage[iso-8859-7]{inputenc}
%\setmainfont{Minion Pro} % substitute with any font that exists on your system
%\setsansfont{Myriad Pro} % substitute with any font that exists on your system
%\setmonofont{Consolas} % substitute with any font that exists on your system
\AtBeginSection[]{
  \begin{frame}
  \vfill
  \centering
  \begin{beamercolorbox}[sep=8pt,center,shadow=true,rounded=true]{title}
    \usebeamerfont{title}\insertsectionhead\par%
  \end{beamercolorbox}
  \vfill
  \end{frame}
}



\setbeamertemplate{theorems}[numbered]

\newcommand{\img}[2]{
    \begin{center}
\includegraphics[scale=#1]{#2}
\end{center}
}


\graphicspath{ {img/} }
\newcommand{\trace}[2]{
\begin{figure}[H]
\centering
\includegraphics[scale=0.3]{#1}
\caption{#2}
\label{#2}
\end{figure}
}

\newcommand{\tracelong}[2]{
\begin{figure}[H]
\centering
\includegraphics[scale=0.2]{#1}
\caption{#2}
\label{#2}
\end{figure}
}

\newcommand{\tracelonglong}[2]{
\begin{figure}[H]
\centering
\includegraphics[scale=0.3]{#1}
\caption{#2}
\label{#2}
\end{figure}
}

\newcommand{\repeatcaption}[2]{%
  \renewcommand{\thefigure}{\ref{#1}}%
  \captionsetup{list=no}%
  \caption{#2 (repeated from page \pageref{#1})}%
}

\newcommand{\graph}[2]{
\begin{figure}[H]
\centering
\includegraphics[scale=0.5]{#1}
\caption{#2}
\label{#2}
\end{figure}
}

\newcommand{\customGraph}[2]{
\begin{figure}[H]
\centering
\includegraphics[scale=0.35]{#1}
\caption{#2}
\label{#2}
\end{figure}
}


\newcommand{\mediumGraph}[2]{
  \begin{figure}[H]
    \centering
    \includegraphics[scale=0.2]{#1}
    \caption{#2}
    \label{#2}
    \end{figure}
    }

\lstset{basicstyle=\ttfamily,columns=fullflexible}

\newcommand{\Code}[2]{
  \begin{minipage}{\linewidth}
  \lstinputlisting[basicstyle=\ttfamily\scriptsize,caption=#2,captionpos=b]{#1}
  \label{#2}
  \end{minipage}
 }

\newcommand{\Output}[2]{
  %%\BVerbatimInput[fontsize=\tiny]{#1}
  \begin{minipage}{0.85\textwidth}
  \lstinputlisting[label={#2},numbers=none,frame=none,caption=#2]{#1}
  \end{minipage}
  %%\caption{#2}
 }

\newcommand{\Side}[5]{
  %\begin{figure}
    \begin{minipage}{0.5\textwidth}
      \lstinputlisting[frame=none, numbers=none,caption={[#2]}]{#1}
    \end{minipage}
    %%\begin{minipage}{0.1\textwidth}
    %%  \includegraphics[scale=0.1]{arrow.pdf}
    %%\end{minipage}
    \begin{minipage}{0.5\textwidth}
      %%\VerbatimInput{#2}
      \lstinputlisting[frame=none, numbers=none,caption={[#4]}]{#3}
    \end{minipage}
    \captionof{figure}{#5}
    \label{#5}
    
     % \caption{#5}
     % \label{#5}
  %\end{figure}
} 

\newcommand{\bigtabular}[2]{
 \begin{table} 
   \resizebox{\linewidth}{!}{
      \input{#1}
    }
    \caption{#2}
    \label{#2}
 \end{table}
}

\newcommand{\landscapetabular}[2]{
\begin{landscape}
 \begin{table} 
   \resizebox{\linewidth}{!}{
      \input{#1}
    }
    \caption{#2}
    \label{#2}
 \end{table}
\end{landscape}
}

\def\HiLi{\leavevmode\rlap{\hbox to \hsize{\color{yellow!50}\leaders\hrule height .8\baselineskip depth .5ex\hfill}}}


\newcommand{\smalltabular}[2]{
  \begin{table} 
    \resizebox{0.8\textwidth}{!}{\begin{minipage}{\textwidth}
     \input{#1}
     \caption{#2}
     \label{#2}
    \end{minipage}}
  \end{table} 
}

\newtheorem{thm}{Θεώρημα}[section]
\newtheorem{lem}[thm]{Λήμμα}
\newtheorem{por}[thm]{Πόρισμα}
\newtheorem{defn}[thm]{Ορισμός}
%----------------------------------------------------------------------------------------
%	TITLE PAGE
%----------------------------------------------------------------------------------------

\title[Short title]{Parallelizing Dynamic Partial Order Reduction Algorithms in Concuerror} % The short title appears at the bottom of every slide, the full title is only on the title page

\author{Παναγιώτης Φυτάς} % Your name
\institute[NTUA] % Your institution as it will appear on the bottom of every slide, may be shorthand to save space
{
ΣΗΜΜΥ - ΕΜΠ \\ % Your institution for the title page
\medskip
\textit{03112113} % Your email address
}
\date{} % Date, can be changed to a custom date
\setcounter{subsection}{1}


\begin{document}

\begin{frame}
\titlepage % Print the title page as the first slide
\end{frame}

\begin{frame}
\frametitle{Summary} % Table of contents slide, comment this block out to remove it
\tableofcontents % Throughout your presentation, if you choose to use \section{} and \subsection{} commands, these will automatically be printed on this slide as an overview of your presentation
\end{frame}

%----------------------------------------------------------------------------------------
%	PRESENTATION SLIDES
%----------------------------------------------------------------------------------------

%------------------------------------------------
%------------------------------------------------

%\subsection{} % A subsection can be created just before a set of slides with a common theme to further break down your presentation into chunks



\AtBeginSection{\frame{\sectionpage}}
\section{Background}

\begin{frame}{Concurrent Computing}

Concurrency is difficult to get right:
\pause
\begin{itemize}[<+->]
    \item Deadlocks
    \item Race conditions
    \item Resource starvation 
    \item Scheduling non-determinism
    \begin{itemize}[<+->]
        \item Interleaving non-determinism
        \item Timing non-determinism
    \end{itemize}
\end{itemize}
\pause
Errors can occur only on specific rare interleavings. 
Detecting and reproducing bugs becomes extremely hard.

\end{frame}

\begin{frame}[fragile]{Modeling our Problem}

\begin{itemize}[<+->]
    \item An interleaving (or a trace) represents a scheduling of the concurrent program.
    \item The state space is the set of all possible interleavings.
    \item In order to verify a program, the complete state space must be explored.
\end{itemize}

\end{frame}

\begin{frame}{Stateless Model Checking}

\begin{itemize}[<+->]
    \item Stateless Model Checking systematically explores all possible interleavings.
    \item Combinatorial state space explosion.
    \item Different interleavings can be equivalent (Mazurkiewicz trace).
\end{itemize}

\end{frame}

\begin{frame}[fragile]{A Simple Program}


\begin{columns}[onlytextwidth]
    \begin{column}{0.4\textwidth}
        \begin{lstlisting}[frame=none, numbers=none]
            p:
            read(x)
            write(x)
        \end{lstlisting}
    \end{column}
    \begin{column}{0.4\textwidth}
        \begin{lstlisting}[frame=none, numbers=none]
            q:
            read(y)
            read(x)
        \end{lstlisting}
    \end{column}
\end{columns}


\end{frame}



\begin{frame} {Stateless Model Checking}


\tracelong{../img/initstateless.png}{State space of a program.}

\end{frame}

\begin{frame} {Partial Order Reduction}

Partial Order Reduction tries to avoid exploring equivalent interleavings through race detection.

\trace{../img/initpor.png}{Interleavings explored by POR}

\end{frame}

\begin{frame} {Partial Order Reduction}
\begin{itemize}[<+->]
    \item Static Partial Order Reduction: Dependencies are tracked before execution, by statically analyzing the code.
    \item Dynamic Partial Order Reduction (DPOR): Actual dependencies are observed during runtime.
\end{itemize}

\end{frame}


\begin{frame} {Terminology}

\begin{itemize}[<+->]
    \item The complete execution of a process $p$ splits into different execution steps, which are
    to be executed atomically. Those steps are referred to as $events$. Each event must be deterministic.
    \item An execution sequence $E$ of a system is a finite sequence of execution steps
    of its processes that is performed from a unique initial state.
    \item An execution sequence $E$ is uniquely characterized by the sequence of processes
    that perform steps in $E$. For instance, $p.p.q$ denotes the execution
    sequence where first $p$ performs two steps, followed by a step of $q$.
    \item The sequence of processes that perform steps in $E$ also uniquely
    determines the global state of the system after $E$.
\end{itemize}

\end{frame}


\begin{frame} {Erlang}

Erlang is a programming language that has build-in support for:

\begin{itemize}[<+->]
  \item Concurrency
  \item Distribution
\end{itemize}

\end{frame}
\begin{frame} {Concurrent Erlang}

In the core of concurrent Erlang are its lightweight processes which:

\begin{itemize}[<+->]
  \item are implemented by the Erlang VM's (BEAM) runtime system, not by operating system threads.
  \item are uniquely identified by their Pid (Process Identifier).
  \item communicate with each other mainly though message passing (actor model).

\end{itemize}
\pause

Erlang also has support for shared memory through the build-in ETS (Erlang Term Storage) module.

\end{frame}

\begin{frame} {Distributed Erlang}

\begin{itemize}[<+->]
\item An Erlang node is an Erlang runtime system containing a complete virtual machine which
contains its own address space and set of processes. 
\item Erlang nodes can communicate over the network, remotely or through the localhost. 
\item Pids continue to be unique over different
nodes (globally).
\item Inside two different nodes, two different processes can have the
same local Pid.
\item All primitives operate over the
network similarly as they would on the same node.

\end{itemize}

\end{frame}


\begin{frame} {Concuerror}

Concuerror is a tool that uses various stateless model checking techniques
in order to systematically 
test an Erlang program, with the aim of detecting and reporting concurrency-related runtime errors.
Its main components are:

\begin{itemize}[<+->]
\item Instrumenter:
  \begin{itemize}
  \item adds preemption points to various points in the code of a tested program.
  \item makes it possible to produce specific interleaving.
  \end{itemize}

\item Scheduler:
  \begin{itemize}
  \item uses mainly source-DPOR or optimal-DPOR, to determine which interleavings need to be checked.
  \item controls the execution of the processes to produce those interleavings.
  \end{itemize}
\end{itemize}

 
\end{frame}

\begin{frame}{Aim of Thesis}

\begin{itemize}[<+->]
    \item Develop parallel version for source-DPOR algorithm.
    \item Develop parallel version for optimal-DPOR algorithm.
    \item Implement those parallel algorithms in Concuerror.
    \item Evaluate the performance of our implementation.
\end{itemize}

\end{frame}

\section{Sequential DPOR Algorithms}


\begin{frame} {General DPOR Concepts}

DPOR: performs a DFS using a backtrack set. Two main techniques:
\begin{itemize}[<+->]
    \item Persistent sets: only a provably sufficient subset of the enabled processes gets explored.
    \item Sleep sets: prevents redundant exploration.
\end{itemize}

\end{frame}

\begin{frame} {Optimality in DPOR}

\begin{itemize}[<+->]
\item A DPOR algorithm is optimal if, for every maximal execution sequence $E$, it explores exactly one interleaving from 
$[E]_{\simeq}$.
\item Persistent-set based DPOR is not optimal.
\end{itemize}
\end{frame}


\begin{frame}{Source Sets}

%%%\begin{definition}[Initials after an execution sequence $E.w$, $I_{[E]}(w)$]
%%%    $p \in I_{[E]}(w)$ if and only if there is a sequence $w'$ such that $E.w \simeq E.p.w'$.
%%%\end{definition}


\begin{definition}[Weak Initials after an execution sequence $E.w$, $WI_{[E]}(w)$]

$p \in WI_{[E]}(w)$ if and only if there are sequences $w'$ and $v$ such
that $E.w.v \simeq E.p.w'$.

\end{definition}

\begin{definition}[Source Sets]
Let $E$ be an execution sequence,
and let $W$ be a set of sequences, such that $E.w$ is an execution
sequence for each $w \in W$. A set $T$ of processes is a source set for
$W$ after $E$ if for each $w \in W$ we have $WI_{[E]}(w) \cap T  \neq \emptyset$.
\end{definition}
\\

\item A source set of a sequence $w$ at an execution sequence $E$, contains the process that can perform the  ``first steps" 
after $E$ that can reproduce sequences equivalent to $w$.

\end{frame}


\begin{frame}{Source-DPOR}

\begin{figure}
    
\scalebox{1}{
\SetKwProg{Fn}{Function}{}{}
\SetKwHangingKw{Let}{let}
\begin{algorithm}[H]
    \caption{Source-DPOR}
    \label{Source}
    \Fn{Explore($E$,$Sleep$)}{
        \If{$\exists p \in (enabled(s_{[E]}) \backslash Sleep)$}{
            $backtrack(E) :={p}$\;
            \While{$\exists p \in (backtrack(E) \backslash Sleep)$}{
                \ForEach{$e \in dom(E)$ such that $e \lesssim_{E.p} next_{[E]}(p)$}{
                    \Let{$E' = pre(E,e)$}
                    \Let{$u = notdep(e,E).p$}
                    \If{$I_{[E']}(u) \cap backtrack(E') = \emptyset$}{
                        add some $q' \in I_{[E']}(u) \text{ to } backtrack(E')$\;
                    }
                }
                \Let{$Sleep' := \{q \in Sleep \mid E \models p \diamondsuit q \} $}
                $Explore(E.p, Sleep')$\;
                add $p$ to $Sleep$\;

            }
        }
    }
\end{algorithm}
}
\end{figure}

\end{frame}


\begin{frame}{Source-DPOR}

Source-DPOR in action:

\tracelong{../img/source1.png}{Interleavings explored by the source-DPOR.}

\end{frame}


\begin{frame}{Motivation for Wakeup Trees}

\begin{columns}
\column{0.75\textwidth}
\begin{figure}
    
\scalebox{0.7}{
\SetKwProg{Fn}{Function}{}{}
\SetKwHangingKw{Let}{let}
\begin{algorithm}[H]
    \label{Source}
    \Fn{Explore($E$,$Sleep$)}{
        \If{$\exists p \in (enabled(s_{[E]}) \backslash Sleep)$}{
            $backtrack(E) :={p}$\;
            \While{$\exists p \in (backtrack(E) \backslash Sleep)$}{
                \ForEach{$e \in dom(E)$ such that $e \lesssim_{E.p} next_{[E]}(p)$}{
                    \Let{$E' = pre(E,e)$}
                    \HiLi\Let{$u = notdep(e,E).p$}
                    \If{$I_{[E']}(u) \cap backtrack(E') = \emptyset$}{
                        add some $q' \in I_{[E']}(u) \text{ to } backtrack(E')$\;
                    }
                }
                \Let{$Sleep' := \{q \in Sleep \mid E \models p \diamondsuit q \} $}
                $Explore(E.p, Sleep')$\;
                add $p$ to $Sleep$\;

            }
        }
    }
\end{algorithm}
}
\end{figure}
\column{0.25\textwidth}

\begin{itemize}

\item $E'.u$ needs to be explored.
\item But only a single process gets added to the backtrack.
\item This may lead to sleep-set blocking.
\end{itemize}

\end{columns}


\end{frame}


\begin{frame} {Wakeup Trees}


\begin{definition}{(Ordered Tree)}\label{def:Ordered}\\
An $ordered$ $tree$ is a pair $\langle B , \prec \rangle$, where B (the set of nodes) is a finite prefix-closed
    set of sequences of processes with the empty sequence $\langle\rangle$ being the root.
    The children of a node $w$, of form $w.p$ for some set of processes $p$, are ordered by $\prec$. 
    In $\langle B , \prec \rangle$, such an ordering between children has been extended to the total 
    order $\prec$ on $B$ by letting $\prec$ be the induced post-order relation between the nodes in $B$.
    This means that if the children $w.p_1$ and $w.p_2$ are ordered as $w.p_1 \prec w.p_2$,
    then $w.p_1 \prec w.p_2 \prec w $ in the induced post-order.
\end{definition}

\begin{definition}{(Wakeup Tree)}\\
    Let $E$ be an execution sequence and $P$ a set of processes. a $wakeup$ $tree$ after $\langle E , P \rangle$
    is an ordered tree $\langle B , \prec \rangle$, for which the following properties hold:
    \begin{itemize}
        \item $WI_{[E]}(w) \cap P = \emptyset$ for every leaf $w$ of $B$.
        \item For every node in $B$ of the form $u.p$ and $u.w$ such that $u.p \prec u.w$ and $u.w$ is a leaf
        the $p \not \in WI_{[E.u]}(w)$ property must hold true.
    \end{itemize}
\end{definition}

\end{frame}


\begin{frame} {Wakeup Trees}

Intuitively, wakeup trees hold sequences of processes that
need to be explored. They can visualized in the following way:

\pause

\tracelonglong{WuTExample.png}{Visualizing wakeup trees.}

\end{frame}

\begin{frame} {Wakeup Trees}

Inserting new sequences ($insert_{[E]}(w,\langle B , \prec \rangle)$) in the wakeup tree has the following properties:

\begin{itemize}[<+->]

    \item $insert_{[E]}(w,\langle B , \prec \rangle)$ is also a wakeup tree after $\langle E , P \rangle$.
    \item Any leaf of $\langle B , \prec \rangle$ remains a leaf of $insert_{[E]}(w,\langle B , \prec \rangle)$.
    \item $insert_{[E]}(w,\langle B , \prec \rangle)$ , while it may not contain $w$ as a leaf, it contains a  leaf $u$ with $u \sim_{[E]} w$. 

\end{itemize}


\end{frame}


\begin{frame} [shrink=28]{Optimal-DPOR}

\SetKwProg{Fn}{Function}{}{}
\SetKwHangingKw{Let}{let}
\begin{algorithm}[H]
    \caption{Optimal-DPOR}
    \label{optimal}
    \Fn{Explore($E$,$Sleep$,$WuT$)}{
        \uIf{$enabled(s_{[E]}) = \emptyset$}{
            \ForEach{$e,e' \in dom(E)$ such that ($e \lesssim_{E} e'$)}{
                \Let{$E' = pre(E,e)$}
                \Let{$v = notdep(e,E).proc(e')$}
                \If{$sleep(E') \cap WI_{[E']}(v)= \emptyset$}{
                    $insert_{[E']}(v,wut(E'))$\;
                }
            }
        }
    \Else {
        \uIf{$WuT \not = \langle \{ \langle \rangle \}, \emptyset \rangle$}{
            $wut(E) := WuT$\;
        }
        \Else {
            choose $p \in enabled(s_{[E]})$\;
            $wut(E) := \langle \{ p \}, \emptyset \rangle $\;
        }
        $sleep(E) := Sleep$\;
        \While{$\exists p \in wut(E)$}{
            \Let{ $p = min_{\prec}\{ p \in wut(E)\}$}
            \Let{$Sleep' := \{q \in sleep(E) \mid E \models p \diamondsuit q \} $}
            \Let{$WuT' = subtree(wut(E), p)$}
            $Explore(E.p, Sleep', WuT')$\;
            add $p$ to $sleep(E)$\;
            remove all sequences of form $p.w$ from $wut(E)$\;
        }
    }
}
\end{algorithm}
\end{frame}


\begin{frame} {Optimal-DPOR example}
\tracelong{../img/opt_seq/11.png}{Optimal-DPOR exlporation.}
\end{frame}

\begin{frame} {Optimal-DPOR example}
\tracelong{../img/opt_seq/55.png}{Optimal-DPOR exlporation.}
\end{frame}

\section{Parallelizing source-DPOR and optimal-DPOR}

\begin{frame} {Parallel source-DPOR}

Lets assume that we have an execution sequence $E$ and that $p$ and $q$ are processes in $backtrack(E)$. We could:

\begin{itemize}[<+->]
  \item Assign the exploration of $E.p$ and $E.q$ to different workers-schedulers.
  \item The exploration frontier gets updated in a non-local manner.
  \item Who would be responsible for exploring $E.r$?
  \item How do we now that the subtrees are balanced?
\end{itemize}
 
\end{frame}

\begin{frame} {Basic Idea}

We are going to use a centralized Controller who will be responsible for:

\begin{itemize}[<+->]
  \item assigning work to different schedulers, by partitioning Execution sequences.
  \item resolving conflicts.
\end{itemize}
\end{frame}

\begin{frame} {Basic Idea}
The Controller will keep track of:

\begin{itemize}[<+->]
  \item the Frontier of our search: the set of the execution sequences assigned to different schedulers.
  \item the Execution Tree: the subtree of the state space that is currently being explored i.e., the Frontier combined in a tree form.
  A path from the root of tree to a leaf uniquely determines an execution sequence.
\end{itemize}

\end{frame}

\begin{frame} {Basic Idea}

Also, lets introduce the concept of Execution Tree node ownership:

\begin{itemize}[<+->]
  \item A scheduler exclusively \textbf{owns} a node of the state space if: 
    \begin{itemize}
    \item it is an assigned backtrack entry
    \item it is a descendant of a node that the scheduler owns.
    \end{itemize}
  \item All other nodes, are considered to have a \textbf{disputed} ownership.
\end{itemize}

\end{frame}

\begin{frame} {Controller Logic}

\SetKwProg{Fn}{Function}{}{}
\SetKwHangingKw{Let}{let}
\begin{algorithm}[H]
    \label{controllerloop}
    \Fn{controller\_loop($N$, $Budget$, $Schedulers$)}{
        $E_0 \leftarrow$ an arbitrary initial execution sequence\;
        $Frontier \leftarrow[E_0]$\;
        $T \leftarrow$ an execution tree rooted at $E_0$\;
        \While{$Frontier \neq \emptyset$} {
            $Frontier \leftarrow partition(Frontier, N)$\;
            \While{exists an idle scheduler $S$ and an unassigned execution sequence $E$ in $Frontier$}{
                $E_c \leftarrow$ a copy of $E$\;
                mark $E$ as assigned in $Frontier$\;
                $spawn(S, explore\_loop(E_c, Budget))$\;
            }
            $Frontier,T \leftarrow wait\_scheduler\_response(Frontier, T)$\;
        }
           
    }

\end{algorithm}

\end{frame}

\begin{frame} {Frontier Partitioning}

\SetKwProg{Fn}{Function}{}{}
\SetKwHangingKw{Let}{let}
\begin{algorithm}[H]
    \label{partition}
    \Fn{partition($Frontier$, $N$)}{
        \For{all E $\in$ Frontier}{
            \While{$total\_backtrack\_entries(E) > 1$ \textbf{and} $size(Frontier) < N$}{
                $E' \leftarrow $ \textnormal{the smallest prefix of $E$
                that has a backtrack entry} \;
                $p \leftarrow \textnormal{ a process} \in backtrack(E') $\;
                $E_c' \leftarrow \textnormal{ a copy of } E'$\;
                \textnormal{remove $p$ from $ backtrack(E')$}\;
                \textnormal{add $p$ to $ sleep(E')$}\;
                \textnormal{add $ backtrack(E') $ to $ sleep(E_c')$}\;
                \textnormal{add $E_c'$ to $ Frontier$}\;
            }
        }
        \Return $Frontier$\;
    }
\end{algorithm}

\end{frame}



\begin{frame} {Scheduler Exploration Loop}

\SetKwProg{Fn}{Function}{}{}
\SetKwHangingKw{Let}{let}
\begin{algorithm}[H]
    \label{explore_loop}
    \Fn{explore\_loop($E_0$, $Budget$)}{
        $StartTime \leftarrow get\_time()$\;
        $ E \leftarrow E_0$\;
        \Repeat{$CurrentTime - StartTime > Budget \textbf{ or } size(E) \leq size(E_0)$}{
            $ E' \leftarrow explore(E)$\;
            $ E' \leftarrow plan\_more\_interleavings(E') $\;
            $ E \leftarrow get\_next\_execution\_sequence(E')$\;
            $CurrentTime \leftarrow get\_time()$\;
        }
        \textnormal{\textbf{send}  $E$ to Controller} \;
    }
\end{algorithm}

\end{frame}

\begin{frame} {Parallel source-DPOR}

\SetKwProg{Fn}{Function}{}{}
\SetKwHangingKw{Let}{let}
\begin{algorithm}[H]
    \caption{Handling Scheduler Response}
    \label{response}
    \Fn{wait\_scheduler\_response(Frontier, T)}{
        \textnormal{\textbf{receive} $E$ from a scheduler}\;
        remove $E$ from $Frontier$\;
        $E',T \leftarrow update\_execution\_tree(E, T)$\;
        \If{$E'$ has at least one backtrack entry}{
                \textnormal{add $E'$ to $Frontier$}\;
        }
        \Return $Frontier,T$;
    }
\end{algorithm}


\end{frame}
\begin{frame}{Resolving conflicts}

$update\_execution\_tree(E, T)$:

\begin{itemize}
\item iterates
over the execution sequence and the execution tree simultaneously .
\item backtrack entries of E that are not found in the execution tree, are added to it.
\item ownership is claimed over them.


\end{itemize}

\end{frame}


\begin{frame} {Load Balancing}

\begin{itemize}[<+->]

\item Load balancing through time-slicing.

\item Schedulers report back to Controller after their running time exceeds their budget.

\item Smaller budget leads to better load balance, but also a larger communication overhead.

\item We change the value of budget depending on the number of idle schedulers.


\end{itemize}

\end{frame}

\begin{frame}  {A simple example}
\tracelong{../img/src_example/1_2.png}{Initial execution sequences.}
\end{frame}


\begin{frame}  {A simple example}
\tracelong{../img/src_example/3.png}{Scheduler 1 exploration.}
\end{frame}

\begin{frame}  {A simple example}
\tracelong{../img/src_example/5.png}{Execution Tree after Scheduler 1 reports to Controller.}
\end{frame}



\begin{frame}  {A simple example}
\tracelong{../img/src_example/4.png}{Scheduler 2 exploration.}
\end{frame}


\iffalse
\begin{frame} {Parallel optimal-DPOR - A first attempt}

While trying to parallelize optimal-DPOR following the same technique we encounter two main issues:

\begin{itemize}[<+->]
\item In the sequential optimal-DPOR calls to $Explore(E, Sleep, WuT)$ guarantee that the complete subtree rooted
at $E$ will be explored. However, for this to hold true,
all sequences in $wut(E)$ that were ordered before WuT must have been explored. This means that the concept of ownership cannot be applied to complete wakeup trees.
\item $insert_{[E]}(v,wut(E))$ may end up inserting at $wut(E)$ an execution sequence different than $v$ (but one that will lead to equivalent interleavings).
\end{itemize}

\end{frame}
\fi

\begin{frame} {Parallel optimal-DPOR - A first attempt}

Any leaf of $\langle B , \prec \rangle$ remains a leaf of $insert_{[E]}(w,\langle B , \prec \rangle)$.
\\
We use this to develop a parallel algorithm that uses:

\begin{itemize}
  \item a single planner that is responsible for race detecting interleavings sequentially.
  \item multiple schedulers that simply explore in parallel the interleavings
  that generated by the planner
  \item a Controller who is responsible for managing the queue of the planner and for assigning interleavings to schedulers for exploration
\end{itemize}

\end{frame}

\begin{frame} {Parallel optimal-DPOR - A first attempt}

This attempt fails to provide any speedup:


\smalltabular{"../tables/opttab.tex"}{Parallel optimal-DPOR performance.}

\end{frame}

\begin{frame} {Parallel optimal-DPOR - A first attempt}

Our attempt fails because:

\begin{itemize}[<+->]
\item We have parallelized the exploration phase of our algorithm, but have kept the most time consuming phase sequential.
\item We have noticed that the planner, during most of the execution of our program, does not generate enough interleavings to keep the schedulers busy.
\end{itemize}
\end{frame}

\begin{frame} {Parallel optimal-DPOR - A first attempt}

This behavior can be observed from the following graphs:

\graph{../img/bottleneck_lastzero_green.png}{Execution time for readers 10 by optimal-DPOR.}


\end{frame}


\begin{frame} {Scalable parallel optimal-DPOR}


Lets assume that we have an execution sequence $E$ and that $w$, $v$ are leaf sequences of $wut(E)$:
\begin{itemize}[<+->]
\item Assign the exploration of $E.w$ and $E.v$ to different workers-schedulers.
\item Those schedulers will explore the subtrees rooted at the assigned sequences.
\end{itemize}


\end{frame}

\begin{frame} {Changing the concept of ownership}
We need to modify the concept of ownership as follows:

\begin{itemize}[<+->]
\item A leaf sequence assigned to a scheduler is \textbf{owned} by that scheduler. 
\item This scheduler also owns every node in the subtree rooted at that leaf. If $v$ is an owned leaf sequence in $wut(E)$ a scheduler owns every execution sequence that has a prefix of $E.v$.
\item Leaf sequence assigned to different schedulers are marked as \textbf{not owned}
\item All other nodes are considered \textbf{disputed}.
\end{itemize}
\end{frame}

\begin{frame} {Resolving conflicts}

Conflicts are harder to resolve:

\begin{itemize}[<+->]
\item Conflicts occur between sequences, instead of processes.
\item $insert_{[E]}(v,wut(E))$ may end up inserting at $wut(E)$ an execution sequence different than $v$.
\item different sequences may be equivalent.

\end{itemize}

\end{frame}

\begin{frame} {Resolving conflicts}

To resolve them:

\begin{itemize}[<+->]
\item We modify $insert_{[E]}(v,wut(E))$ to insert sequences in the Execution Trees.
\item If a disputed sequence can be inserted into the execution tree, ownership is claimed it.
\item Otherwise, it is marked as not owned.

\end{itemize}

\end{frame}


\begin{frame} {Scalable parallel optimal-DPOR}

This approach is similar to the parallel source-DPOR except that:

\begin{itemize}[<+->]

\item $update\_execution\_tree(E, T)$ $inserts$ sequences in the Execution Tree, instead of processes.
\item Schedulers are assigned the state space under a sequence, instead of a single process.

\end{itemize}

\end{frame}

\begin{frame} {Comparing the two algorithms}

Therefore:

\begin{itemize}[<+->]

\item the Controller has a higher computational complexity.
\item the schedulers communicate with the Controller more frequently.

\end{itemize}



\end{frame}


\begin{frame} {Modifying Concuerror}

\begin{itemize}[<+->]
\item Sequential Concuerror has one process for every process of tested program.
\item Different schedulers must be able to explore different interleavings concurrently.
\item Each scheduler must have its own set of processes. 
\item Create mappings between the different copies of a process.
\end{itemize}

\end{frame}


\begin{frame} {The issue with Pids}


\begin{itemize}[<+->]
\item Different Pids per process.
\item Erlang allows comparison between Pids.
\item The result of those comparisons must remain the same on different schedulers. 
\item $pid\_to\_list$ BIF makes things worse.
\end{itemize}

\end{frame}

\begin{frame} {The issue with Pids}


\begin{itemize}[<+->]
\item We want different process with the same Pid.
\item Inside an Erlang node, process from that node are reference by their local Pid.
\item Processes from different nodes can have the same local Pid.
\item Pids are generated sequentially within a node.
\end{itemize}

\end{frame}

\begin{frame} {Different processes with the same Pid}


\begin{itemize}[<+->]
\item Each scheduler runs in its own node.
\item Reach a consensus between the different schedulers as to the initial local Pid.
\item Spawn processes after the consensus been reached.
\item The $i_{th}$ processes spawned this way on different nodes, will all have the same local Pid.
\end{itemize}

\end{frame}

\begin{frame} {Modifying Concuerror}

Concuerror now runs distributed.

\end{frame}

\section{Performance Evaluation}



\begin{frame}{Benchmarks}

\begin{itemize}
\item The benchmarks were performed on a multiprocessor with 64 AMD Opteron 6276(2.3 GHz) cores, 126 GB of memory, running
Linux 4.9.0-8amd64 and running the later Erlang version (Erlang/OTP 21.1). 
\item While running our tests, we are using the
--$keep\_going$ flag to continue exploring our state space, even after an error is found. We do this so we can evaluate
how fast the complete state space gets explored, regardless of whether errors exist.
\end{itemize}

\end{frame}


\begin{frame} {Benchmarks}
Lets give a brief overview of our benchmarks:

\begin{itemize}
    \item $indexer$ $N$: This test uses a Compare and Swap (CAS) primitive instruction to check if a specific element of
    a matrix is set to 0 and if so, set it to a new value. This is implemented in Erlang by using ETS tables and specifically
    the $insert\_new/2$ function. This function returns false if the key of the inserted tuple exists (the entry is set to 0)
    or it inserts the tuple if the key is not found. $N$ refers to the number of threads that are performing this function.
    \item $readers$ $N$: This benchmark uses a writer process that writes a variable and $N$ reader processes that read that variable.
    \item $lastzero$ $N$: In this test we have $N+1$ processes that read and write on an array of $N+1$ size, which has all its 
    values initialized with zero. The first process reads the array in order to find the zero element with the highest
    index. The other $N$ processes read an array element and update the next one.
    \item $rush$ $hour$: a program that uses processes and
    ETS tables to solve the Rush Hour puzzle in parallel, using A\textasteriskcentered  search. Rush hour is a complex but self-contained (917 lines of code) program.
\end{itemize}

\end{frame}

\begin{frame}{Results for sequential algorithms}

Lets present here number of interleavings for our benchmarks, as well as their performance for sequential algorithm and the parallel algorithm with one scheduler:

\bigtabular{"../tables/synthetic_unbounded.tex"}{Sequential performance of source-DPOR and optimal-DPOR on four benchmarks.}

We can notice that the overhead for our parallel optimal-DPOR appears to be larger than the overhead of the parallel source-DPOR.
This is to be expected since updating the execution tree in the Controller should be more expensive for the optimal algorithm.


\end{frame}

\begin{frame}{Evaluation on readers 15}

\customGraph{../scripts/readers_15_10000_combo_time.png}{Performance of readers 15 with Budget of 10000.}

\end{frame}


\begin{frame}{Evaluation on readers 15}

Trying to figure out why our algorithm fails scale for readers 15:

\customGraph{../scripts/readers1510000_combined_stack.png}{Number of times schedulers stopped their execution with a Budget of 10000.}

\end{frame}


\begin{frame}{Evaluation on readers 15}

Increase budget to 30000ms:

\customGraph{../scripts/readers_15_30000_combo_time.png}{Performance of readers 15 with budget of 30000.}

\end{frame}

\begin{frame}{Evaluation on readers 15}


\customGraph{../scripts/readers1530000_combined_stack.png}{Number of times schedulers stopped their execution with a Budget of 30000.}

\end{frame}


\begin{frame}{Evaluation on readers 15}

Increasing the Budget:

\begin{itemize}[<+->]
\item Reduces the performance of source-DPOR, since it causes load imbalance.
\item Increases the performance of optimal-DPOR, since finding disputed entries also leads to load balancing and therefore,
optimal-DPOR does not need as frequent load balance.
\end{itemize}

\end{frame}

\iffalse
\begin{frame}{Evaluation on rush hour}


\tracelong{../img/rushhour.png}{Input of rush hour}


\end{frame}
\fi

\begin{frame}{Evaluation on lastzero 11}

\customGraph{../scripts/lastzero_11_10000_combo_time.png}{Performance of lastzero 11 with Budget of 10000 for source and 30000 for optimal.}


\end{frame}

\begin{frame}{Evaluation on rush hour}

\customGraph{../scripts/rush_hour_10000_combo_time.png}{Performance of rush hour with Budget of 10000 for source and 30000 for optimal.}


\end{frame}

\begin{frame}{Evaluation on indexer 17}

\customGraph{../scripts/indexer_17_10000_combo_time.png}{Performance of indexer 17 with Budget of 10000 for source and 30000 for optimal.}


\end{frame}

\begin{frame}{Evaluation on rush hour}

Both of our algorithms provide high speed and decent scalability on this test case.
\\
However, after 24 schedulers we notice that the scalability of optimal-DPOR starts to break. This is because:

\begin{itemize}[<+->]
\item the communication between the Controller and the schedulers in the case of optimal-DPOR is substantially more frequent.
\item the $update\_execution\_tree$ function in optimal-DPOR is more ``expensive".
\end{itemize}

\end{frame}

\begin{frame}{Recap}

Depending on the benchmark, Source-DPOR:
\begin{itemize}[<+->]
\item Can provide a speedup by a factor of 20.
\item Retain scalability for 32 schedulers.
\end{itemize}
\pause
Optimal-DPOR:
\begin{itemize}[<+->]
\item Can provide a speedup by a factor of 17.
\item Retain scalability for up to 24 schedulers.
\end{itemize}


\end{frame}

\begin{frame}{Conclusion}

During this thesis we:
\begin{itemize}[<+->]
\item Parallelized source-DPOR and optimal-DPOR.
\item Implemented our algorithms in Concuerror.
\item Achieved significant speedups.
\item Achieved scalability up to 32 schedulers on specific benchmarks.
\end{itemize}

\end{frame}




\begin{frame}{}

Thank you for your attention!

\end{frame}

\end{document} 
