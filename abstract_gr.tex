Η επαλήθευση και ο έλεγχος ταυτόχρονων προγραμμάτων είναι μία πολύπλοκη εργασία. Λόγω του μη ντετερμινιστικού τρόπου με τον οποίο
δουλεύει ο scheduler ενός συστήματος, μπορεί μόνο μερικά interleavings από processes να είναι ελαττωματικά και συνεπώς, 
για να ελεγχθεί η ορθότητα ενός προγράμματος, πρέπει να εξεταστούν όλα τα πιθανά schedulings. Ο αριθμός, όμως, αυτών των 
schedulings είναι υψηλής τάξης εκθετικός σε σχέση με το μέγεθος του προγράμματος. Η πιο αποτελεσματική μέθοδος για να λυθεί αυτό
το πρόβλημα είναι το Stateless model checking με τη χρήση αναγωγής σε δυναμικές
σχέσεις μερική διάταξης (Dynamic Partial Order Reduction ή DPOR). Καθώς τα παράλληλα συστήματα επεξεργασίας έχουν
γίνει το κυρίαρχο πρότυπο της σύγχρονης αρχιτεκτονικής υπολογιστών, η παραλληλοποίηση των διαφόρων DPOR αλγορίθμων είναι 
αναγκαία για την κλιμάκωση αυτών των αλγορίθμων στους σύγχρονους υπολογιστές.

Αυτή η διπλωματική εργασία ασχολείται με τη παραλληλοποίηση του Concuerror, ένα stateless model checking εργαλείο που χρησιμοποιεί
διάφορους DPOR αλγόριθμους για να ελέγξει ταυτόχρονα προγράμματα γραμμένα σε Erlang. Συγκεκριμένα, θα εστιάσουμε στο
να καταστήσουμε εφικτή την παράλληλη εξερεύνηση διαφορετικών interleavings απο τον Concuerror και στο να σχεδιάσουμε 
παράλληλες εκδοχές για τα τους δύο πιο αποδοτικούς αλγορίθμους του Concuerror: τον source-DPOR και τον optimal-DPOR.
Επίσης, θα αξιολογήσουμε την επιτάχυνση και την κλιμακωσιμότητα των υλοποιήσεων μας σε διάφορα benchmarks που 
χρησιμοποιούνται ευρέως για την αξιολόγηση DPOR αλγορίθμων. 

