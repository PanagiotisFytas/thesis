\chapter{Τεχνικές περιορισμού για DPOR}
\label{bounded}

Σε αυτό το κεφάλαιο Τεχνικές Περιοσμού (Bounding Techniques) για τον DPOR συζητόνται καθώς και προκλήσεις που δημιουργούνται από αυτές τις τεχνικές.
Οι προκλήσεις που πρέπει να αντιμετωπίσουμε για να περιορίσουμε τον DPOR έχουν ήδη συζητηθεί.

%% Challenges discussion can not be placed before the presentation of the algorithms since it is closely related to the algorithms themselves
%% and not the Bounding of DPOR. Provided that other algorithms were suggested, different challenges may had been arisen. 

\section{Naive-BPOR}

Οι πρώτη τεχνική περιορισμού που παρουσιάζεται είναι ο Naive-BPOR (Αλγόριθμος \ref{Vanilla}). Ο σκοπός αυτού του αλγορίθμου είναι να μπλοκάρει traces που ξεπερνούν κάποιο
όριο.

\begin{algorithm}
    \caption{Naive-BPOR}
    \label{Vanilla}
    \SetKwInOut{Input}{input}
    Explore($\langle \rangle$,$\emptyset$,$b$)\;
    \Fn{Explore($E$,$Sleep$,$b$)}{
        \If{$\exists p \in (enabled(s_{[E]}) \backslash Sleep)$ such that $B_v(E.p) \leq b)$ }{
            $backtrack(E) :={p}$ \;
            \While{$\exists p \in (backtrack(E) \backslash Sleep$ and $B_v(E.p) \leq b$}{
                \ForEach{$e \in dom(E)$ such that $e \lesssim_{E.p} next_{[E]}(p) $}{
                    \Let{$E' = pre(E,e)$}
                    \Let{$u = notdep(e,E).p$}
                    \If{$I_{E'}(u) \cap backtrack(E') = \emptyset$}{
                        add some $q' \in I_{[E']}(u) to backtrack(E') $ \;
                    }
                }
                \Let{$Sleep' := \{q \in Sleep \mid E \models p \diamondsuit q \} $}
                $Explore(E.p, Sleep, b)$ \;
                add $p$ to $Sleep$ \;

            }
        }
    }
\end{algorithm}

Ο Αλγόριθμος \ref{Vanilla} είναι σχεδόν πανομοιότυπος με τον  Source-DPOR(Algorithm \ref{Source}). Η μόνη διαφορά είναι η προσθήκη μια συνθήκης που συνδέεται με τη δρομολόγηση
διεργασίων. Όταν ένα βήμα μιας διεργασίας $p$ προστίθεται στο $E$ με αποτέλεσμα το  trace $E.p : B_v(E.p) > b$ τότε η διεργασία δεν επιτρέπεται να δρομολογηθεί.
Ο Αλγόριθμος δεν είναι sound δηλαδή, δεν εξετάζει όλα τα traces τα οποία ανήκουν έχουν bound count μικρότερο από μία τιμή.

Υποθέτουμε το παραάδειμα του ενός εγγραφέα και των δύο αναγνωστών για bound $b=0$. Ένα παράδειγμα αναζήτησης για bound $b=0$ δίνεται στο Figure \ref{Naive-BPOR for bound=$0$}. 

\label{Vanilla0}    
\trace{w2rvbound.pdf}{Naive-BPOR for bound=$0$}

Όπως μπορούμε να παρατηρήσουμε υπάρχουν 4 traces τα οποία δεν ξεπερνούν το όριο. Αυτά είναι:
$p.q.q.r.r$, $q.q.p.r.r$, $r.r.p.q.q$, $q.q.r.r.p$.
Παρολ᾽ αυτά ο Naive-BPOR δεν είναι σσε θέση να τα εξερευνήσει όλα; $r.r.p.q.q$ δεν εξερευνάται.
Όπως δείξαμε στη σύγκριση μεταξύ  persistent και source sets, το $r$ δεν θα γίνει ποτέ το πρώτο γεγονός στο trace
καθώς αυτό θα οδηγούσε σε sleep set blocked trace. Η διακλάδωση που θα οδηγούσε στο ισοδύναμο trace με το $r.r.p.q.q$ απορρίπτεται
καθώς θα έχει μεγαλύτερο bound count από το ζητούμενο. Αξιοπρόσεκτο είναι ότι ένας Naive-BPOR που βασιζόταν σε persistent sets θα εξερευνούσε ένα μεγαλύτερο state-space
και θα εξερευνούσε και το ζητούμενο trace $r.r.p.q.q$.

\section{BPOR}

Το επόμενο βήμα είναι να υλοποιήσουμε έναν preemption bounded Αλγόριθμο (BPOR) \cite{BPOR}. Αυτός ο Αλγόριθμος θα βασίζεται στα
persistent sets. Καθώς έχουμε ήδη ένα ορθά υλοποιημένο αλγόριθμο που χρησιμοποιεί persistent sets στον Nidhugg-DPOR μπορούμε εύκολα να υλοποιήσουμε έναν BPOR
αλγόριθμο. Η καινοτμία του BPOR είναι η εισαγωγή συντηρητικών διακλαδώσεων. Αυτές είναι διακλαδώσει που εισάγονται προκειμένου να εγγυηθούν την εξερεύνηση όλου του
state space.
Είναι αρκετά συνηθισμένο ένα trace να ξεπερνάει το  bound limit
αλλά ένα ισοδύναμό του να μην το ξεπερνάει. Οι συντηρητικές διακλαδώσεις χρησιμοποιούνται γι αυτό το σκοπό.

Για να κάνουμε την ιδέα πιο καθαρή εισάγουμε την έννοια Trace Block.
\begin{definition}{(Trace block)}
Για ένα trace $T$ η ακολουθία $B$ από συνεχόμενα γεγονότα είναι ένα Trace Block ανν όλα τα γεγονότα πραγματοποιούντια από την ίδια διεργασία δηλαδή,
όλα τα γεγονότα έχουν το ίδιο thread id.
\end{definition}

Η ιδέα πίσω από τις συντηρητικές διακλαδώσεις είναι αρκετά απλή. Μια συντηρητική διακλάδωση προστίθεται στην αρχή του αντίστοιχου
Block όταν μια διακλάδωση δημιουργείται.
Συνήθως τα ταυτόχρονα γεγονότα συμβαίνουν μέσα σε ένα block. Έτσι όταν επιλέγεται μια εναλλακτική διακλάδωση 
το preemption count θα αυξηθεί. Αν η διακλάδωση αυτή είχε εξερευνηθεί από την αρχή του block το preemption count δε θα είχε αυξηθεί.
Ο BPOR παρουσιάζεται με λεπτομέρια στον Αλγόριθμο \ref{BPOR}. Ο BPOR διαφέρει από τον Classic-BPOR στον διπλά εμφολευμένο βρόγχο που εισάγεται
στη γραμμή 3. Ο εσωτερικός βρόγχος εισάγεται προκειμένου ο BPOR να υπακούει στον ορισμό των preemption-bounded persistent sets που εισάγαμε στο Κεφάλαιο 2
Επιπλέον, όπως ισχύει και στον Naive-BPOR, η εξερεύνηση σταματάει όταν το trace ξεπερνάει το όριο που εμείς έχουμε θέσει.

\begin{algorithm}
    \caption{BPOR}
    \label{BPOR}
    \SetKwInOut{Input}{input}
    \SetKwInput{Initialization}{Explore($ \emptyset $)}
    \SetKwHangingKw{Let}{let}
    \Fn{Explore($E$)}{
        \Let{$s := last(E)$}
        \For{all process $p$}{
            \For{all process $q \neq p$}{
            \If{$\exists i = max(\{ i \in dom(E) \mid E_i$ is dependent and may be co-enabled with $next(s,p)$ and $ E_i.tid = q \} $}{
                \uIf{$p \in enabled(pre(E,i)))$}{
                    add $p$ to $backtrack(pre(E,i))$ \;
                }
                \Else{add $enabled(pre(E,i))$ to $backtrack(pre(E,i))$ \;}
                \uIf{$j = max(\{ j \in dom(E) \mid j = 0 $ or $ S_{j-1}.tid \neq S_j.tid $ and $ j<i \})$}{
                    \uIf{$p \in enabled(pre(E,i)))$}{
                        add $p$ to $backtrack(pre(E,i))$ \;
                    }
                    \Else{add $enabled(pre(E,i))$ to $backtrack(pre(E,i))$ \;}
                }
            }
            }
        }
        \If{$p \in enabled(s)$}{
            add $p$ to $backtrack(s)$ \;
        }
        \Else{
            add any $u \in enabled(s)$ to $backtrack(s)$ \;
        }
        \Let{$visited = \emptyset $}
        \While{$ \exists u \in (enabled(s) \cap backtrack(s) \backslash visited) $}{
            add $u$ to $visited$ \;
            \uIf{$(B_v(S.next(s, u)) \leq c)$}{
                Explore($S.next(s, u)$) \;
            }
        }
    }
\end{algorithm}

\section{Nidhugg-BPOR}

Προκειμένου να εκμεταλλευτούμε την υποδομή του Nidhuggs, μια παραλλαγή του αλγορίθμου χρησιμοποιείται.
Ο αλγόριθμος παρουσιάζεται εδώ \ref{Nidhugg BPOR}.

\begin{algorithm}
    \caption{Nidhugg-BPOR}
    \label{Nidhugg BPOR}
    \SetKwInOut{Input}{input}
    \SetKwHangingKw{Let}{let}
    Explore($\langle \rangle$,$\emptyset$,$b$)\;
    \Fn{Explore($E$,$Sleep$,$b$)}{
        \If{$\exists p \in ((enabled(s_{[E]}) \backslash Sleep)$ and $B_v(E.p) <=b$}{
            backtrack(E) $:={p}$ \;
            \While{$\exists p \in (backtrack(E) \backslash Sleep $ and $B_v(E.p) <=b$}{
                \ForEach{$e \in dom(E)$ such that $e \lesssim_{E.p} next_{[E]}(p)$}{
                    \Let{$E' = pre(E,e)$}
                    \Let{$u = notdep(e,E).p$}
                    \Let{$CI = \{ i \in I_{E'}(u) \mid i \rightarrow p \}$}
                    \If{$CI \cap backtrack(E') = \emptyset$}{
                        \If{$CI \neq \emptyset$}{
                            add some $q' \in CI$ to $backtrack(E') $ \;
                        }
                        \Else{
                             add some $q' \in I_{[E']}(u)$ to $backtrack(E') $ \;}
                        }
                    \Let{$E''= pre\_block(e,E)$}
                    \Let{$u = notdep(e,E).p$}
                    \Let{$CI = \{ i \in I_{E''}(u) \mid i \rightarrow p \}$}
                    \If{$CI \cap backtrack(E') = \emptyset$}{
                        \If{$CI \neq \emptyset$}{
                            add some $q' \in CI$ to $backtrack(E') $ \;
                        }
                        \Else{
                            add some $c(q') \in I_{[E'']}(u)$ to $backtrack(E'') $ \;
                        }
                    } 
                }
                \Let{ $Sleep' := \{q \in Sleep \mid E \models p \diamondsuit q \}$ } 
                $Explore(E.p, Sleep)$ \;
                \If{$p$ is not conservative}{
                    add $p$ to $Sleep$ \;
                }
            }
        }
    }
\end{algorithm}

Παρατηρούμε ότι τα persistent sets χρησιμοποιείται και για τα conservative και για τα non-conservative branches.

Μια σημαντική πρόκληση δημιουργείται όταν ο DPOR Αλγόριθμος χρησιμοποιείται μαζί με τα sleep sets. Αυτό προκύπται από το γεγονός ότι οι 
οι συντηρητικές διακλαδώσεις δεν δημιουργούνται προκειμένουν αν επιλύσουν races. Στον Αλγόριθμο των sleep set παρατηρούμε ότι αν ακολουθήσουμε 
την ίδια στρατηγική με τις μη συντηρητικές διακλαδώσεις πολλά traces θα είχαν μπλοκαριστεί.

Ας πάρουμε και πάλι το παράδειγμα του εγγραφέα και των δύο αναγνωστών, όπως φαίνεται στο Figure \ref{Usage of non-conservative branches}.

\trace{w2rpersistent.pdf}{Usage of non-conservative branches}

Παρατηρούμε ότι το τελευταίο trace θα μπολοκαριστεί από τα sleep set ενώ θα έπρεπε να εξεταστεί. Ο Αλγόριθμος δεν γνωρίζει ότι η διεργασία
$r$ πρέπει να αφαιρεθεί από το sleep set καθώς δεν υπάρχει και δεν πρόκειται να υπάρξει ποτέ conflict με την πρώτη εντολή της διεργασίας καθώς 
αυτή σχετίζεται με μια μη μοιραζόμενη μεταβλητή. Προκειμένου να αντιμετωπίσουμε αυτό το πρόβλημα επιλέχθηκε τα 
conservative branch να μην προστίθενται στο sleep set. Παρολ᾽ αυτά πρέπει να είμαστε σε θέση να καταγράφουμε όλες τις διακλαδώσεις που προστίθενται σε
ένα συγκεκριμένο σημείο προκειμένου καμία διακλάδωση να μην προστεθεί δύο φορές.
Χωρίς αυτό το σύνολο που κάνει αυτή την καταγραφή η εκτέλεση δε θα τελείωνε.
Η λύση είναι η εισαγωγή του συντηρητικού συνόλου (conservative sets) όπου καταγράφεται η προσθήκη των διακλαδώσεων.

Διαισθητικά, ο αλγόριθμος είναι ίδιος με τον Nidhugg-DPOR με την προσθήκη των συντηρητικών διακλαδώσεων.
Ο Αλγόριθμος βασίζεται στην των conservative sets.
Τα υπόλοιπα προβλήματα που αντιμετωπίστηκαν δίνονται στην ενότηταα σχετικά με την υλοποίηση.

\tracelong{w2rbpor.pdf}{Example of BPOR execution}

\section{Source-BPOR}

Έχοντας συζητήσει τον BPOR Αλγόριθμο και τον Nidhugg-BPOR, το επόμενο βήμα είναι να προσπαϑήσουμε να συνδιάσουμε τον αλγόριθμο με τα source sets.
Η πρώτη παρατήρηση που πρέπει να κάνουμε είναι ότι τα source sets και επομένως ο αλγόριθμος που τα δημιουργεί δεν είναι κατάλληλα στην 
προσθήκη συντηρητικών διακλαδώσεων. Μια γρήγορη επεξήγησ δίνεται στο επόμενο παράδειγμα με τον έναν εγγραφέα και τους δύο αναγνώστες.
Ας υποθέσουμε έναν αλγόριθμο που χρησιμποιεί source sets για την προσθήκη συντηρητικών διακλαδώσεων.
Το αποτέλεσμα φαίνεται στο Figure \ref{Following source sets for conservative branches}.

\trace{w2rsourceconservative.pdf}{Following source sets for conservative branches}

Είναι σαφές ότι κάποια traces δεν εξερευνόνται. Συγκεκριμένα, το trace που ξεκινά με τη r θα απορριφθεί. Ο λόγος είναι ότι μοιράζεται τα ίδια
initials με το r1 ακόμα και στην αρχή αυτού του block. Σαν αποτέλεσμα ο Αλγόριθμος πρέπει να δημιουργεί
persistent sets όταν συντηρητικές διακλαδώσεις προστίθενται.

Έχοντας υπόψιν την προηγούμενη παρατήρηση ο Αλγόριθμος Source-BPOR δίνεται εδώ \ref{SBPOR}.

\begin{algorithm}
    \caption{Source-BPOR}
    \label{SBPOR}
    \SetKwInOut{Input}{input}
    \SetKwHangingKw{Let}{let}
    Explore($\langle \rangle$,$\emptyset$,$b$)\;
    \Fn{Explore($E$,$Sleep$,$b$)}{
        \If{$\exists p \in ((enabled(s_{[E]}) \backslash Sleep)$ and $B_v(E.p) <=b$}{
            backtrack(E) $:={p}$ \;
            \While{$\exists p \in (backtrack(E) \backslash Sleep $ and $B_v(E.p) <=b$}{
                \ForEach{$e \in dom(E)$ such that $e \lesssim_{E.p} next_{[E]}(p)$}{
                    \Let{$E' = pre(E,e)$}
                    \Let{$u = notdep(e,E).p$}
                    \If{$I_{E'}(u) \cap backtrack(E') = \emptyset$}{
                        add some $q' \in I_{[E']}(u) \text{ to } backtrack(E') $ \;
                    }
                    \Let{$E''= pre\_block(e,E)$}
                    \Let{$u = notdep(e,E).p$}
                    \Let{$CI = \{ i \in I_{E''}(u) \mid i \rightarrow p \}$}
                    \If{$CI \cap backtrack(E') = \emptyset$}{
                        \If{$CI \neq \emptyset$}{
                            add some $q' \in CI$ to $backtrack(E') $ \;
                        }
                        \Else{
                            add some $c(q') \in I_{[E'']}(u)$ to $backtrack(E'') $ \;
                        }
                    } 
                }
                \Let{ $Sleep' := \{q \in Sleep \mid E \models p \diamondsuit q \}$ } 
                $Explore(E.p, Sleep)$ \;
                \If{$p$ is not conservative}{
                    add $p$ to $Sleep$ \;
                }
            }
        }
    }
\end{algorithm}

Παρατηρούμε ότι ο Αλγόριθμος \ref{SBPOR} χρησιμποιεί Source-DPOR (Alg. \ref{Source}) για τη μη συντηρητικές διακλαδώσεις και τον 
BPOR (Alg. \ref{Nidhugg BPOR}) για τις συντηρητικές διακλαδώσεις.

\section{Προκλήσεις από την προσθήκη Συντηρητικών Διακλαδώσεων}
%% I still believe that this discussion should be placed after results have been presented.

\subsection{Συντηρητικές Διακλαδώσεις}

Στην προηγούμενη ενότηταη είδαμε ότι τα conservative sets δεν μπορούν να χρησιμποιήσουν τα sleep sets. Αυτό συμβαίνει 
επειδή αυτές οι διακλαδώσεις δνε προκύπτουν από conflicts και σαν αποτέλεσμα είναι αδύνατο να ``wake up'' άλλες διεργασίες
των οποίων τα επόμενα βήματα είναι τοπικές λειτουργίες. Το πρόβλημα γίνεται ακόμα πιο πολύπλοκο όταν σκεφτούμε ότι όταν προσθέτουμε
συντηρητικές διακλαδώσεις ο αλγόριθμος ``ξενχά'' τί έχει λάβει χώρα προηγουμένως. Αυτή η έλλειψη μνήμης οδηγεί σε μια έκρηξη του 
χώρου καταστάσεων. Αυτή η έκρηξη είναι ακόμα πιο έντονη και από την εξερεύνηση όλου του state space
Προκειμένου να εξηγήοσυμε καλύτερα, ένα παράδειγμα που περιλαμβάνει έναν εγγραφέα και τρεις αναγνώστες δίνεται στο Figure \ref{writer-3-readers
explosion}. Συγκεκριμένα, η διεργασία εγγραφέας που γράφει τη μεταβλητή x ενώ οι διεργασίες αναγνώστες διαβάζουν αυτή τη μεταβλητή
αφού διαβάσουν μια άλλη μοναδική για την κάθε διεργασία μεταβλητή που μπορεί να θεωρηθεί τοπική λειτουργία.

\trace{w3rbpor.pdf}{writer-3-readers explosion}

Όπως μπορούμε να συμπεράνουμε από το παράδειγμα οι εξερευνημένες καταστάσεις είναι περισσότερες από τις αναμενόμενες. 
Ο αναμενόμενος αριθμός θα έπρεπε να είναι $8$ αλλά στην πραγματικότητα είναι $4!$. Παρατηρούμε ότι κάθε εγγραφέας γίνεται baccktracked 
στην πρώτηη εντολή καθώς έχει conflict με το $w(x)$. Αυτές οι διακλαδώσιιες θα πρέπει να γίνουν blocked απο την unbounded search. 
Παρολ᾽ αυτά, αυτές οι διακλαδώσεις είναι συντηρητικές.
Υπάρχουν  extra traces όπως το $r2.r1.w$ trace που ελέγχει ήδη το εξερευνημένο $r1.r2.w$ trace. 
Ο Αλγόριθμος δεν είναι σε θέση να γνωρίζει όταν προσθέτει συντηρητικές διακλαδώσεις 
αν ισοδύναμα traces έχουν ήδη εξερευνηθεί. Η κατάσταση γίνεται χειρότερ όταν ακόμα περισσότεροι εγγραφείς διαβάζουν
την ίδια μοιραζόμενη μεταβλητή καθώς αν θέσουμε ένα μεγάλο όριο τότε ο αριθμός των traces προσεγγίζει την τιμή $N!$ 
όπου $N$ ο συνολικός αριθμόες των διεργασιών. 

\subsection{Sleep Sets}
Τα αποτελέσματα των διαφόρων  bounding αλγορίθμων υπονοεί ότι ο αριθμός τον sleep set blocked traces είναι πολύ μικρός
σε σχέση με τον αριθμό των εξερευνημένων traces. Αυτό προκύπτει από τα conservative branches. Έχουμε ήδη δείξει όταν μία διεργασία
προστείθεται και ως συντηρητική διακλάδωση και ως μη συντηρητική διακλάδωση η συντηρητική φύση της διακλάδωσης υπερισχύει.
Traces που θα θεωρούνταν περιττά στην ubounded εκδοχή του αλγορίθμου δεν είναι περιττά στην bounded εκδοχή καθώς οι μη 
συντηρητικές διακλαδώσεις μπορεί να έχουν απορριφθεί.

Ένα ακόμα  ``πρόβλημα'' με τα sleep sets είναι ότι  ``ευνοούν'' τις διακλαδώσεις που αυξάνουν το bound count ενώ μπλοκάρουν
traces με μικρότερο bound count. Στο Figure \ref{Sleep set contradiction} φαίνεται πως μια διακλάδωση που δεν οδηγεί
σε αύξηση του bound count απορρίπτεται.

\trace{sleepsetproblem.pdf}{Sleep set contradiction}

Αυτή η συμπεριφορά είνα αρκετά λογική αν αναλογιστούμε την depth-first φύση του αλγορίθμου.
Αποτέλεσμα είναι να επιλέγει πρώτα διακλαδώσει που βρίσκονται χαμηλότερα στο trace 
όπου το bound count είναι μεγαλύτερο καθώς περισσότερα preemptive switches έχουν συμβεί. Ο σκοπός των sleep
sets είναι να μπλοκάρουν traces, δηλαδή traces που έχουν ήδη εξερευνηθεί. Έτσι traces με μικρότερο bound count απορρίπτονται
Μια μέθοδος που θα εξερευνούσε το state space με έναν breadth first τρόπο ίσως δε θα ήταν εφικτή καθώς θα απαιτούσε
πολλή μνήμη από ημιτελή traces.

\iffalse
\subsection{Source Sets - Optimal DPOR}
The source set technique which can lead many times to optimal coverage of the state space manages avoid the exploration
of redundant sets. However as it was discussed in the previous section it avoids the scheduling of the traces.
Unfortunately it cannot be used in the when conservative branches are added since these branches are not related to the
sleep sets. As a result the conservative branches alone will never lead to sleep set blocked traces.

However, in many test cases there are some sleep set blocked traces which are caused by conditional reads and writes.
When a technique which does not utilize the sleep sets is to be used these traces would have been easily eliminated.
Unfortunately, it was experimentally shown that explored traces outnumber the sleepset blocked traces and, consequently,
the implementation of such an algorithm would have a minor impact.

Moreover, we have shown that even if we maintain the source-set optimization for the non-conservative branches the
results will be equivalent with using persistent sets. The idea of keeping the rejected traces from Source-DPOR that
would have been added from DPOR (with persistent sets) was rejected since it harms the soundness of the algorithm.
\fi

\subsection {The Performance - Soundness Tradeoff}

Από την προηγούμενη συζήτηση είναι σαφέ ότι όλοι οι bounding algorithms έχουν να αντιμετωπίσουν ένα tradeoff. Κάποια αλγόριθμοι είναι γρηγορότεροι
καθώς εξερευνούν ένα μικϱό κομμάτι του state space, χωρίς να καλύπτουν όλο το state space μέσα στο bound (Αλγόριθμος \ref{Vanilla}), 
ενώ άλλοι εξερευνούν όλο το state space (Αλγόριθμος \ref{Nidhugg BPOR}) μέσα στο bound αλλά απαιτούν πιο πολλή ώρα.
