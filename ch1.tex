 \chapter{Introduction}

Nowadays, the necessity for concurrency is undeniable. 
Moore's Law \cite{4785860} dictates that the number of
transistors that can be placed inside a dense integrated circuit doubles roughly every two years.
However, during the past years, energy and heat restrictions have significantly thwarted the ability 
to accelerate a single chip. Multithreaded architectures
is the main way with which modern technology can keep up with Moore's Law and therefore, concurrent 
programming is essential for scaling software to modern hardware. Apart from that, through concurrency, it becomes possible for 
long-running tasks to not delay short running ones and consequently, concurrent programs offer an increased availability 
of services making them essential for numerous applications, such as web services. 

Nevertheless, due to the fact that concurrent processes and threads share resources and have to communicate with each other,
concurrent programming is a more challenging endeavor, compared to sequential programming. To make matters worse, 
processes are scheduled in a non-deterministic manner, which can lead to errors that may occur only on specific 
rare interleavings. Therefore, concurrency bugs can be extremely hard to detect and reproduce, making techniques such as
unit testing, ineffective in detecting such errors. Testing concurrent programs requires the systematic exploration
of all possible interleavings (or at least a sufficient subset of those).

Advances made in model checking have led to \textit{Stateless Model Checking} \cite{Godefroid:1997:MCP:263699.263717}, 
which systematically explores the state-space of a given program, without storing the global states, in order to verify
that each reachable state satisfies a given property.
Stateless model checking is a practical approach that does not have large memory requirements and has been implemented in
tools such as CHESS \cite{Musuvathi:2008:FRH:1855741.1855760} and Verisoft \cite{Godefroid:2005:SMC:1084665.1084674}.
Still, this technique suffers from combinatorial explosion. However, different interleavings that can be obtained from each other by
swapping adjacent and independent execution steps, can be considered equivalent.
Partial Order Reduction (POR) \cite{Godefroid1996, POR, 10.1007/3-540-53863-1_36} algorithms utilize this observation to successfully diminish
the size of the explored state-space. Dynamic Partial Order Reduction 
(DPOR) \cite{FlanaganDPOR, AbdullaAronisJohnssonSagonasDPOR2014} algorithms manage to achieve an even increased reduction,
by detecting dependencies more accurately. DPOR techniques have been successfully implemented in tools like Concuerror 
\cite{6569727, Gotovos:2011:TDC:2034654.2034664}, Nidhugg \cite{Abdulla:2015:SMC:2945565.2945622}, Inspect \cite{yang2008inspect}
and Eta \cite{simsa2011efficient}.

Parallelizing DPOR algorithms is essential in order to make them scalable to modern computer architectures, but also to potentially 
achieve significant speedups that will alleviate the effect of the exponential state-space explosion. The parallelization of
DPOR algorithms that use persistent sets \cite{FlanaganDPOR, Lei:2006:RTC:1248722.1248743, 10.1007/3-540-53863-1_36} has already been 
been examined and parallel versions have been implement for Inspect \cite{Yang:2007:DDP:1770532.1770541} and Eta 
\cite{Simsa2012ScalableDP}.

On this thesis, we are going to focus on the parallelization of Concuerror, a stateless model checker
used for testing concurrent Erlang programs. Specifically, we are going to examine the parallelization
of source-DPOR \cite{AbdullaAronisJohnssonSagonasDPOR2014} and optimal-DPOR \cite{AbdullaAronisJohnssonSagonasDPOR2014}, 
which are the main algorithms implemented in Concuerror.




