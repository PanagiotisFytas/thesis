\chapter{Unbounded DPOR}
\label{unbounded}

Σε αυτό το κεφάλαιο συζητόνται οι αλγόριθμοι Classic-DPOR και Source-DPOR. Αυτοί οι αλγόριθμοι διαφέρουν μεταξύ τους σε ό,τι αφορά τα επαρκή σύνολα που χρησιμοποιούν προκειμένου
να εξερευνήσουν το χώρο καταστάσεων.Ο DPOR βασίζεται στα persistent sets ενώ ο Source-DPOR στα source sets.    

\section{Source-DPOR}
Σε αυτή την ενότητα παρουσιάζεται ο Source-DPOR \cite{AbdullaAronisJohnssonSagonasDPOR2014} αλγόριθμος. Παρόλο που ο Source-DPOR είναι μεταγενέστερος του Classic-DPOR, όπως φαίνεται και 
από το όνομα, επιλέγουμε να τον παρουσιάσουμε πρώτο καθώς όλοι οι αλγόιρθμοι που παρουσιάζονται σε αυτή τη διπλωματική βασίζονται στον Source-DPOR.

\SetKwProg{Fn}{Function}{}{}
\SetKwHangingKw{Let}{let}
\begin{algorithm}
    \caption{Source-DPOR}
    \label{Source}
    Explore($\langle \rangle$,$\emptyset$)\;
    \Fn{Explore($E$,$Sleep$)}{
        \If{$\exists p \in (enabled(s_{[E]}) \backslash Sleep)$}{
            $backtrack(E) :={p}$ \;
            \While{$\exists p \in (backtrack(E) \backslash Sleep)$}{
                \ForEach{$e \in dom(E)$ such that $e \lesssim_{E.p} next_{[E]}(p)$}{
                    \Let{$E' = pre(E,e)$}
                    \Let{$u = notdep(e,E).p$}
                    \If{$I_{E'}(u) \cap backtrack(E') = \emptyset$}{
                        add some $q' \in I_{[E']}(u) \text{ to } backtrack(E') $ \;
                    }
                }
                \Let{$Sleep' := \{q \in Sleep \mid E \models p \diamondsuit q \} $}
                $Explore(E.p, Sleep')$ \;
                add $p$ to $Sleep$ \;

            }
        }
    }
\end{algorithm}

\noindent Αρχικά μια αυθαίρετη ενεργή διεργασία που δεν κοιμάται προστίθεται στο $backtrack(E)$. 

Σε κάθε βήμα ο αλγόριθμος αποτελείται από δύο διακριτά βήματα.
Κατα τη διάρκεια του πρώτου βήματος ο εντοπισμός συναγωνισμών (races) λαμβάνει χώρα.
Ο αλγόριθμος επιλέγει μια διεργασία $p$ η οποία μπορεί να εκτελέσει το επόμενό της βήμα, δηλαδή, $p \in enabled(s_{[E]})$, και την προσθέτει στο τέλος του υπο εξερεύνηση trace $E$.
Στη συνέχεια ο αλγόριθμος βρίσκει κάθε γεγονός $e$ το οποίο περιέχεται ήδη στο υπο εξερεύνηση trace ($e \in dom(E)$) και το αντιμεταθέτει με το επόμενο βήμα του $p$.

Αυτό συμβαίνει προκειμένου να εξερευνηθεί μια ακολουθία εκτέλεσης για την οποία το $p$ συμβαίνει πριν το $e$.
Έχουμε ότι η ακολουθία αποτελείται από τα:
\begin{itemize} 
\item $E' \equiv pre(E,e)$: την υπακολουθία $E'$ που αποτελείται από γεγονότα δρομολογημένα πριν το $e$ στο $E$.
\item $u \equiv notdep(e,E).p$: την συνένωνση $u$ από όλα τα γεγονότα που είναι δρομολογημένα μετά το  $e$ στο $E$ αλλά είναι ανεξάρτητα από τα $e$ και $p$.
\item $proc(e)$ που είναι το id της διεργασίά που  προκάλεσε το $e$.
\end{itemize}
Στη συνέχεια ο αλγόριθμος εξετάζει αν υπάρει κάποια διεργασία που ανήκει στο $I_{[E']}(u)$ και είναι ήδη στο $backtrack(E')$. 
Αν όχι, τότε μια διεργσία στο $I_{[E']}(u)$ προστίθεται στο backtrack.

Κατά τη φάση της εξερεύνησης, η εξερεύνηση ξεκινά από το $E.p$. Το σημαντικό κομμάτι είναι ο υπολογισμός του sleep set του επόμενου βήματος καθώς κάποιες διεργασίες ίσως πρέπει να 
ξυπνήσουν.
Αν το επόμενο βήμα μιας διεργασίας συγκρούεται με το $next(p)$ τότε αυτή η διεργασία πρέπει να ξυπνήσει. Σαν αποτέλεσμα το sleep set αποτελείται από τις είδη κοιμώμενες διεργασίες
των οποίων το επόμενο βήμα είναι ανεξάρτητο από το $next(p)$, δηλαδή ισχύει, $Sleep' := \{q \in Sleep \mid E \models p \diamondsuit q \} $ 
Μετά το τέλος της εξερεύνησης του $E.p$, το $p$ προστίθεται στο sleep set επειδή θέλουμε να αποτρέψουμε την εξερεύνηση ενός ισοδύναμου trace.

\section{Classic-DPOR}

Επειδή ο Source-DPOR χρησιμποιεί διανυσματικά ρολόγια για να παρακολουθεί τα γεγονότα θα χρησιμοποιήσουμε τον DPOR με χρήση Clock Vectors \cite{FlanaganDPOR} και μια παραλλαγή του που δίνεται στον Αλγόριθμο \ref{DPORV}.

\begin{algorithm}
    \caption{DPOR using Clock Vectors (Classic-DPOR)}
    \label{DPORV}
    \SetKwInOut{Input}{input}
    \SetKwInput{Initialization}{Explore($\emptyset, \lambda x. \bot$)}
    \SetKwHangingKw{Let}{let}
    \Fn{Explore($E$,$C$)}{
        \Let{$s := last(E)$}
        \For{all process $p$}{
            \If{$\exists i = max(\{ i \in dom(E) \mid E_i$ is dependent and may be co-enabled with $next(s,p)$ and $i \not \leq  C(p)(proc(E_i)) \} $}{
                \uIf{$p \in enabled(pre(E,i)))$}{
                    add $p$ to $backtrack(pre(E,i))$ \;
                }
                \Else{add $enabled(pre(E,i))$ to $backtrack(pre(E,i))$ \;}
            }
        }
        \If{$\exists p \in enabled(s)$}{
            $backtrack(s) := {p}$ \;
            \Let{$done = \emptyset$}
            \While{$\exists p \in (backtrack(s) \backslash done)$}{
               add $p$ to $done$ \;
               \Let{$t = next(s,p)$} 
                \Let{$E' = E.t$} 
                \Let{$cu = max \{ C(i) \mid i \in 1.. | S |$ and $E_i$ dependent with $t \}$} 
                \Let{$cu2 = cu [ p := | E' | ] $} 
               \Let{$C' = C [p:= cu2, |E'| := cu2 ]$} 
                $Explore(E',C')$ \;
            }
        }
    }
\end{algorithm}

Ένα clock vector $C(p)$ συντηρείται καθόλη την εκτέλεση του αλγορίθμου για κάθε
διεργασία $p$. Έτσι, το $C(p_i) = \langle c_1, c_2, ..., c_m \rangle$
αντιπροσωπεύει το clock vector μιας διεργασίας $p_i$.
όπου το $c_j$ είναι το  index της τελευταίας μετάβασης της διεργασίας $p_j$ για την οποία ισχύει ότι
$c_j \rightarrow_s p_i$. Διαισθητικά, το clock vector μιας διεργασίας $p_j$ δίνει πληροφορία στη διεργασία $p_j$ σχετικά με την εκτέλεση των βημάτων των άλλων διεργασιών που 
συμβαίνουν πριν από την εκτέλεση των βημάτων της $p_j$. Τα clock vectors βασίζονται στον αλγόριθμο του Lamport \cite{Lamport@CACM-89} και περισσότερες λεπτομέριες 
σχετικά με τη χρήση τους μπορούν να βρεθούν εδώ \cite{FlanaganDPOR}. 
Αντιπροσωπεύουμε την αρχική κατάσταση όλων των vector clocks ως $\bot = \langle 0, ..., 0 \rangle$. Με $C(p)(proc(E_i ))$ αντιπροσωπεύουμε την τιμή των 
clock vector μίας διεργασίας $p$ για τη διεργασία στο i-οστό βήμα του $E$.

Η αρχική κατάσταση του Classic-DPOR είναι μια κενή ακολουθία από γεγονότα και όλα τα vector clocks τίθενται σε $\bot$.

Ο Classic-DPOR αποτελείται από δύο φάσεις. Η πρώτη φάση είναι η ανίχνευση races. Κατα τη διάρκεια αυτής της φάσης η επόμενη μετάβαση από όλες τις διεργασίες $p$ λαμβάνεται υπόψιν.
Για κάθμε μια μετάβαση $next(s, p)$ (που μπορεί να είναι ενεργή ή όχι στο s), υπολογίζεται η τελευταία εξαρτόμενη με αυτή μετάβαση $i$ στο $E$.
Ο υπολογισμός λαβμάνει χώρα στη γραμμή 4.
Αν υπάρχει μια τέτοια μετάβαση $i$, ίσως υπάρχει και ένα race
condition ή μία εξάρτηση μεταξύ $i$ και $next(s, p)$, και επομένως, ίσως θα πϱέπει να εισάγουμε ένα “backtracking point” στη θέση του 
state $pre(S, i)$, δηλαδή στο state ακριβώς πριν την ετκέλεση της μετάβασης $i$. Αν $p$ είναι ενεργή διεργασία προστίθεται σαν backtrack point. Αλλιώς το σύνολο όλων των ενεργών 
μεταβάσεων γίνεται backtracked.

Κατά τη διάρκει της φάσης της εξερεύνησης η διεργασία $p$ που ανήκει στο $enabled(s)$ προστίθεται σαν backtrack point όπως και στον Source-DPOR. 
Στη συνέχεια τα vector clocks ενημερώνονται σύμφωνα με τον αλγόριθμο του Lamport.

\section{Σύγκριση μεταξύ Classic-DPOR και Source-DPOR}

Όπως μπορούμε εύκολα να διαπιστώσουμε και οι δύο αλγόριθμοι αποτελούνται από τις δύο ίδιες διακριτές φάσεις το race detection και τη φάση της εξερεύνησης. Επιπλέον και οι 
δύο αλγόριθμοι βασίζονται στα διανυσματικά ρολόγια, παρόλο που η χρήση τους απλώς υπονοείται στον Source-DPOR στις γραμμές 6 και 8 του Αλγορίθμου \ref{Source}.

Η βασική διαφορά έγκειται στη φάση της ανίχνευσης race. Κατά τη διάρκεια αυτής της φάσης στον Classic-DPOR όλες οι διεργασίες $p$ λαμβάνονται υποψιν πριν δρομολογηθούν
και έτσι πολλές από αυτές θα γίνουν backtrackede. Από την άλλη, στον Source-DPOR μόνο η τελευταία διεργασία που δρομολογήθηκε 
λαμϐάνεται υπόψιν. Επιπλέον οι δύο αλγόριθμοι διαφέρουν στον τρόπο που αντιμετωπίζουν την περίπτωση της απουσίας ενεργών διεργασιών κατα τη διάρκει προσθήκης backtrack point.
Σε αυτή την περίπτωση ο Classic-DPOR προσθέτει όλες της ενεργές διεργασίες ενώ καμία διεργασία δεν προστείθεται από τον Source-DPOR.
Συνολικά, ο Source-DPOR εκμεταλλέυεται τα Clock vectors περισσότερο από τον Classic-DPOR.

\section{Συνδιάζοντας Classic-DPOR και Source-DPOR}

Προκειμένου να εκμεταλλευτούμε την υποδομή του Nidhugg ο Αλγόριθμος \ref{NDPOR} θα χρησιμποιηθεί. Σαν αποτέλεσμα δεν υπάρχει ανάγκη να προσθέσουμε όλα τα διαθέσεμα
threads όταν το $p$ δεν είναι ενεργό. Αν δεν έχει βρεθεί κατάλληλος υποψήφιος τότε
ένας υποψήφιος που προτείνεται από τον αντίστοιχ Source-DPOR αλγόριθμο θα προστεθεί. Αυτός ο τρόπος υπολογισμού persistent sets θεωρείται πιο πολύπλοκος και επομένω
κακή επιλογή \cite{Gode05}. Παρολ᾽ αυτά η χρήση source sets είναι πιο κοντά σε αυτή την προσέγγιση.

\begin{algorithm}
    \caption{Nidhugg-DPOR}
    \label{NDPOR}
    \SetKwInOut{Input}{input}
    \SetKwHangingKw{Let}{let}
    Explore($\langle \rangle$,$\emptyset$)\;
    \Fn{Explore($E$,$Sleep$)}{
        \If{$\exists p \in (enabled(s_{[E]}) \backslash Sleep$}{
            backtrack(E) $:={p}$ \;
            \While{$\exists p \in (backtrack(E) \backslash Sleep)$}{
                \ForEach{$e \in dom(E)$ such that $e \lesssim_{E.p} next_{[E]}(p)$}{
                    \Let{$E' = pre(E,e)$}
                    \Let{$u = notdep(e,E).p$}
                    \Let{$CI = \{ i \in I_{E'}(u) \mid i \rightarrow p \}$}
                    \If{$CI \cap backtrack(E') = \emptyset$}{
                        \If{$CI \neq \emptyset$}{
                            add some $q' \in CI \text{ to } backtrack(E') $ \;
                        }
                        \Else{add some $q' I_{E'}(u) \text{ to } backtrack(E')$}
                    }
                }
                \Let{ $Sleep' := \{q \in Sleep \mid E \models p \diamondsuit q \} $ } 
                $Explore(E.p, Sleep)$ \;
                add $p$ to $Sleep$ \;
            }
        }
    }
\end{algorithm}

Ο Αλγόριθμος \ref{NDPOR} διαφέρει από τον Source-DPOR (Αλγ. \ref{Source}) στον υπολογισμό των initials.
Συγκεκριμένα ένα υποσύνολο των initials ($CI$) που συμβαίνει πριν από το $p$ χρησιμοποιείται. Διαισθητικά στην περίπτωηση του εγγραφέα και 
των δύο αναγνωστών και οι δύο αναγνώστες θα προστεθούν στο backtrack καθώς ο πρώτος αναγνώστης δε συμβαίνει πριν από τον δεύτερο.
Για να γενικέυσουμε την ιδέα: 
Το Nidhugg δεν μας επιτρέπει να δημιουργήσουμε διακλαδώσει για το $last(E)$, κατα τη διάρκεια της φάσης της δρομολόγησης. Οι διακλαδώσεις προστίθενται αργότερα
στον Source-DPOR (Αλγ. \ref{Source}). Όταν ένα race εξετάζεται συνήθως μόνο ένα thread που προκαλέι το race Θα προστεθεί καθώς το $CI$ περιέχει αυτό 
το thread μόνο.

Θα δώσουμε μια διαισθητική απόδειξη σχετικά με την ορθότητα του Nidhugg-DPOR στον να υπολογίζει persistent set και θα 
εξηγήσουμε γιατί όταν ο αλγόριθμος τελειώσει ένα persistent θα έχει υπολογιστεί σε κάθε βήμα. Το ενδιαφέρον κομμάτι είναι να δείξουμε ότι για καθε διακλάδωση που
είναι μια εντολή εγγραφής όλες οι διεργασίες που συγκρούονται με την εντολή και μπορούν να συμβούν ταυτόχρονα με αυτή θα προστεθούν στο σύνολο των διακλαδώσεων.

Ας υποθέσουμε δύο διεργασίες που βρίσκονται σε race στο $last(S)$.
\begin{itemize}

\item Περίπτωση 1: τουλάχιστον μία διεργασία περιέχει μια εντολή εγγραφής. Ξέρουμε ότι οι δύο διεργασίες θα πρέπει να αντιστραφούν σε κάποιο σημείο.
Από τη στιγμή που ο Nidhugg-DPOR αγνοεί τα weak initials θα προσθέσει και τις δύο διεργασίες. Στο Figure \ref{Construction of persistent
sets in Nidhugg when there is a write process} παρατηρούμε ότι οι διεργασίες $q$ και $r$ αντιστρέφονται. Στον Source-DPOR 
μόνο μία από τις διεργασίες πρέπει να χρησιμποιηθοεί για να γίνει διακλάδωση καθώς μοιράζονται τα ίδια initials. 
Παρολ᾽ αυτά στον Nidhugg-DPOR αυτό δεν ισχύει καθώς το $CI$ δεν περιέχει βήματα από τις άλλες διεργασίες.

\trace{nidhuggpersistent1.pdf}{Construction of persistent sets in Nidhugg when there is a write process}

\item Case 2: και οι δύο διεργασίες είναι αναγνώστες.
Αφού δεν υπολογίζουμε το $I$ αλλά το $CI$ η πρώτη ανάγνωση δε θα ληφθεί υπόψιν καθώς δε συμβαίνει πριν από την δεύτερη ανάγνωση και σαν αποτέλεσμα και οι δύο διεργασίες
θα προστεθούν στο $backtrack$. Στο Figure \ref{Construction of persistent sets in Nidhugg when both are read processes} παρατηρούμε ότι στον υπολογισμό του $CI$ 
όταν το race μεταξύ των $p$ και $r$ εντοπίζεται η $q$ αγνοείται και, επομένως, το $r$ θα προστεθεί για να γίνει διακλάδωση.

\trace{nidhuggpersistent.pdf}{Construction of persistent sets in Nidhugg when both are read processes}

\end{itemize}

Είναι σαφές ότι καμία διεργασίε που δεν ανήκει στο $backtrack(S)$ δεν έχει ανταγωνισμό με το τις διεργασίες στο $backtrack(S)$. 
