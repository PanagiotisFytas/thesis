\chapter{Επιπλέον Συζήτηση του Bounding Problem}
\label{Chapter 5}

Σε αυτό το κεφάλαιο συζητόνται εναλλακτικές προσεγγίσεις του preemption bounding problem για τον DPOR.
Προτείνεται μια καινούρια προσέγγιση η οποία δείχνουμε ότι είναι ισοδύναμη με την προσθήκη συντηρητικών διακλαδώσεων η οποία όμως δεν χρησιμοποιεί 
συντηρητικές διακλαδώσεις.

\section{Τεχνικές χωρίς την Προσθήκη Συντηρητικών Διακλαδώσεων}

Σε προηγούμενο κεφάλαιο συζητήσαμε τις προκλήσεις που προκύπτουν από τη σχεδίαση ενός Bounded DPOR αλγορίθμου. Είδαμε ότι δεν μπορούν να γίνουν
σημαντικές βελτιώσεις όταν προσθέτουμε συντηρητικές διακλαδώσεις και ότι πολλές βελτιστοποιήσεις που δουλεύουν για τις μη φραγμένες εκδοχές των
αλγορίθμων δεν δουλεύουν για τις φραγμένες εκδοχές τους.

\subsection{Κίνητρο}
Ο μόνος αλγόριθμος που δεν προσθέτει συντηρητικές διακλαδώσεις είναι ο Naive-BPOR. Για ένα επαρκές bound ένα trace που περιείχε 
σφάλμα θα εξετάζοταν από τον αλγόριθμο. Το ελλάτωμα αυτού του αλγορίθμου είναι ότι δεν είναι sound.
Σε αυτό τον αλγόριθμο μια συνάρτηση υπολογίζει τον αριμό των preemptive switches. Όμως πολλές switches που υπολογίζονται μπορούν να 
αποφευχθούν.

Ένα παράδειγμα δίνεται στο Figure \ref{An example of avoidable preemption-switch} που εξηγεί καλύτερα την ιδέα. Έστω 
ένα πρόγραμμα που αποτελείτα από τις διεργασίες $p$ και $q$. Η διεργασία $p$ γράφει μια μοιραζόμενη μεταβλητή $x$ και η διεργασία $q$ διαβάζει
μια μεταβλητή $y$ (η οποία δεν τροποιείται από κάποια άλλη διεργαία), και γράφει την μεταβλητή $x$. Υπάρχουν δύο δυνατά 
interleavings όπως φαίνεται και στην εικόνα. Αν υποθέσουμε ότι κάνουμε εξερεύνηση με bound $0$ τότε και μια συντηρητική διακλάδωσηη πρέπει να 
προστεθεί.

\trace{motivation.pdf}{An example of avoidable preemption-switch}

Στο Figure \ref{An example of avoidable preemption-switch} το preemptive switch που λαμβάνει χώρα θα μπορούσε εύκολα να αποφευχθεί αντιστρέφοντας
απλώς την πρώτη εντολή του $q$ με την πρώτη την $p$. Αλλά τί επιτρέπει αυτή την αλλαγή;

Η απάντηση βρίσκεται στα γεγονότα που αποτελείται το block. Στην περίπτωσή μας το block αποτελείται από 
ένα μόνο βήμα. Το πρώτο block διαβάζει μια μεταβλητή που δε χρησιμοποιείτα από άλλο block. Επομένως τα δύο block δεν έχουν happens-before
σχέση.

Η παρατήρηση αυτή οδηγεί στο επόμεον ερώτημα: Ποία preemption switches είναι υποχρεωτικά; Ή ισοδύναμα ποια traces δεν μπορούν να παραχθούν
χωρίς preemptive switch; Επιπλέον, είναι δυνατό για ένα trace να υπολογίσουμε τον ελάχιστο αριθμό από preemptive switches που θα απαιτούσε ένα
ισοδύναμο trace;

\subsection{Ένας Αλγόριθμος χωρίς Συντηρητκές Διακλαδώσεις}
Ένας αλγόριθμος που θα έκανε bounded search θα ήταν διαφορετικός από τον Naive-BPOR μόνο σε ότι αφορά τη συνάρτηση που υπολογίζει το
preemption count της δρομολόγησης. Αυτή η συνάρτηση $f$ θα ήταν αύξουσα δηλαδή θα ίσχυε για ένα πρόθεμα $E$, $f(E) \leq f(E.E')$ για κάθε $E'$.

Η γενική μορφή του αλγορίθμου δίνεται στον Αλγόριθμο \ref{NBBPOR}.

\begin{algorithm}[H]
    \SetAlgoLined
    \caption{General form of the BPOR without branch addition}
    \label{NBBPOR}
    \KwResult{Explore the whole state space within the bound}
    Explore($\emptyset$)\;
    \Fn{Explore($S$)}{
        T = Sufficient\_set($final(S)$)
     \For{all $t \in T$}{
         \If{$min\{B_v([S.t])\} \leq c$}{
            Explore($S.t$)
         }
        }
    }
\end{algorithm}

Συγκρίνοντας τον Αλγόρθμο \ref{NBBPOR} με τον \ref{GeneralDPOR} παρατηρούε ότι αντί να υπολογίζουμε το $B_v(S.t)$ δηλαδή το preemption
count ενός trace υπολογίζουμε το $min(B_v[S.t])$ δηλαδή το ελάχιστο $B_v$ για όλα τα ισοδύναμα traces.

\subsection{Υπολογισμός του Ελάχιστου Preemption Count}
To μόνο που μας μένει είναι η κατασκεύ της συνάρτησης $f$. Για ένα trace $E$ που αποτελείται από blocks πολλές σχέσεις 
happens-before ισχύουν. Κάθε ισοδύναμο trace πρέπει να συμφωνεί με αυτές τις σχέσεις. Επίσης σχέσεις happens-before ισχύουν
για τις εντολές εντός ενός block. Γι αυτή την ενότητα μόνο θα μας απασχολήσουν οι σχέσεις μεταξύ blocks.
Οι λόγοι που έγινε αυτή η επιλογή είναι οι εξής:

\begin{itemize}
    \item Οι αλγόριθμοι που παρουσιάζονται στη συνέχεια είναι πολύ πιο απλοί.
    \item Δεν μας ενδιαφέρει να σπάσουμε περαιτέρω κάθε block και επομένως μπορούμε να δούμε τοblock σαν μια ενότητα.
\end{itemize}

Αυτές οι σχέσεις happens-before σχηματίζουνε ένα γράφο. Αυτός ο γράφος αποτελείται από κόμβους που αντιστοιχούν σε blocks και 
ακμές που αντιστοιχούν στις σχέσεις μεταξύ τους. Προφανώς block που ανήκουν στο ίδιο thread έχουν happens-before σχέση.
Επίσης μπορούμε να κινηθούμε από το ένα block στο άλλο από τη στιγμή που αυτά είναι ταυτόχρονα.
Προσθέτουμε βάρη σε κάθε ακμή. Ακμές που συνδέουν blocks του ίδιου thread έχουν βάρος $0$. Οι ακμές που ξεκινάν από block
που είναι μπλοκαρισμένα έχουν επίσης βάρος $0$. Οι υπόλοιπες ακμές έχουν βάρος ένα.

Προκειμένου να βρούμε το ελάχιστο preemption count διασχίζουμε όλα τα block του γράφου χωρίς να παραβιάζουμε τις happens-before σχέσεις.
Επομένως το minimum bound count αντιστοιχεί στο ελάχιστο hamiltonian μονοπάτι το οποίο δεν παραβίαζει τις happens-before σχέσεις.

Επειδή ο υπολογισμός του ελάχιστου hamiltonian μονοπατιού είναι απαιτητικός κατασκευάζουμε ένα γράφο που περιορίζει όσο αυτό είναι
δυνατό πιθανές διασχίσεις που παραβιάζουν τις σχέσεις happens-before.

\noindent Ένας αλγόριθμος που προσθέτει τα blocks στο γράφο είναι δίνεται στον Αλγόριθμο \ref{Adding a new block to the
dependencies graph}. Ο αλγόριθμος λειτουργεί επαγωγικά. Αρχικά ο γράφος αποτελείται από το πρώτο block. Όταν block
του trace ολοκληρώνεται το προσθέτουμε στο γράφο. Προσθέτουμε κάθε block που συμβαίνει ταυτόχρονα με ένα άλλο block με διπλή ακμή
Επιπλέον συνδέουμ το πιο πρόσφατο block κάθε thread που συμβαίνει πριν από το καινούριο block με ακμες που καταλήγουν στο καινούριο block.

\SetKw{Return}{return \;}
\SetKw{Break}{break \;}
\begin{algorithm}[H]
    \caption{Adding a new block to the dependencies' graph}
    \label{Adding a new block to the dependencies graph}
    \Fn{AddBlock(block,graph)}{
        \If{previous block of the same thread was not blocked}{
            increase the weigh of the edges coming from the previous block to 1 \;
        }

        \For{each thread t}{
            list:= preceding blocks t\;
            \For{l in reversed(list)}{
                \If{$l \leftrightarrow block$}{
                    add edge from $block$ to $l$ with weight $0$ \;
                    \If{$l$ is not last}{
                        add edge from $l$ to $block$ with weight $1$ \;
                    }
                    \Else{
                        add edge from $l$ to $block$ with weight $0$ \; 
                    }
                }
                \If{$l \rightarrow block$}{
                    \If{$l$ is not last}{
                        add edge from $l$ to $block$ with weight $1$ \;
                    }
                    \Else{
                        add edge from $l$ to block with weight $0$ \; 
                    }
                    \Break
                }
            }
        }
    }
\end{algorithm}


\trace{compulsoryswitch.pdf}{Graph example}

Στο Figure \ref{Graph example} ένα απλό παράδειγμα ενός τέτοιου γράφου παρουσιάζεται. Για το trace παρατηρούμε ότι το $w(y)$ του thread 
$q$ είναι ταυτόχρονο με το $r(x)$ ενώ συμβαίνει πριν το $w(y)$ του $p$ thread. Κάθε μετάβαση κοστίζει 1 preemption switch και γι αυτό
το λόγο έχει βάρος 1.
Επιπλέον, μεταβάσεις μεταξύ των ίδιων thread κοστίζουν 0. Είναι σημαντικό να σημειώσουμε ότι αν παραβιάσουμε τις happens-before σχέσεις
δεν υπάρχει διάσχιση που να μπορεί να διασχίσει όλους τους κόμβους. Για παράδειγμα ξεκινώντας από το $r(x)$ και μεταβένοντας στο
$w(x)$ δεν υπάρχει τρόπος να διασχίσουμε όλους του κόμβους.

Μπορούμε να δούμε ότι το hamiltonian path με βάρος 1 για ένα trace. Αυτό είναι και το ελάχιστο hamiltonian μονοπάτι.

Έτσι έχουμε καταφέρει να ανάξουμε το πρόβλημα του minimum preemption coount στο πρόβλημα του minimum hamiltonian path.
Το πρόβλημα αυτό είναι γνωστό  NP-hard. Έτσι δεν μπρούμε να αναμένουμε ο αλγόριθμος που υπολογίζει το βάρος να είναι θεαματικά πιο γρήγορος
από μια εξερεύνηση DFS. 

Αυτό είναι πολύ ενδιαφέρον αποτελέσμα καθώς απεικονίζε τη δυσκολία του DPOR bounding problem καθώς η προσθήκη των συντηρητικών διακλαδώσεων
υπονοεί αυτή την DFS αναζήτηση.

Τώρα που έχουμε διαπιστώσει τη δυσκολία του προβλήματος ένα καινυριο ερώτημα δημιουργείται. Μπορούμε να προσεγγίσουμε το βάρος του ελάχιστου μονοπατιού;
Ένας τέτοιας αλγόριθμος δεν θα ήταν sound αλλά θα ήταν μια βελτίωση στον Naive-BPOR χωρίς να έχουμε την έκρηξη του χώρου καταστάσεων.

\subsection{Προσεγγίζοντας το Bound Count}
Δύο μέθοδη δοκιμάστηκαν για να προσεγγίσουμε την τιμή του preemption count. Η ιδέα και για τους δύο αλγορίθμους βασίζεται στην εξής παρατήρηση:
Ένα preemption switch είναι υποχρεωτικό αν δύο block της ίδιας διεργασίας A διακόπτονται από ένα block της διεργασίας B.
Δηλαδή αν ισχύει $e_1(A) \rightarrow e(B) \rightarrow e_2(A)$. Στην περίπτωση που ίσχυε $e_1(A) \not \rightarrow e(B)$ ή
$e(B) \not \rightarrow e_2(A)$ θα μπορούσαμε να αντιστρέψουμε τα blocks χωρίς να παραβιάζουμε τις happens-before σχέσεις.

Ο Αλγόριθμος παρουσιάζεται εδώ:\\

\SetKwProg{Fn}{Function}{}{}
\begin{algorithm}[H]
    \caption{First Approximation Algorithm}
    \label{First Approximation Algorithm}
    \Fn{BoundCount($E$,$current\_bound$)}{
        \For{$i=0 \text{ to } len(E)-1$}{
            \If{$E[i].pid = last(E).pid$}{
                $higher\_block = i$ \;
                \Break
            }
        }
        \For{$i = higher\_block+1 \text{ to } len(E)-1$}{
            \If{$ E[higher\_block] \rightarrow E[i] \rightarrow last(E)$}{
                current\_bound++ \;
                \Return
            }
        }
    }
\end{algorithm}

Στον Αλγόριθμο \ref{First Approximation Algorithm} εντοπίζουμε το πιο πρόσφατο block με το ίδιο pid με το τελευτάιο block. Στη συνέχεια προσπαθούμε να βρούμε αν υπάρχει κάποιο
γεγονός που συμβαίνει μετά το πρώτο γεγονός και πριν το τελευταίο.
Αν αυτό υπάρχει τότε αυξάνουμε τον μετρητή.
Προκειμένου να διαπιστώσουμε τις happens-before σχέσεις τα vector clocks μπορούν να χρησιμοποιηθούν.

Ο δεύτερος αλγόριθμος εξερευνεί μεγαλύτερο κομμάτι του state space.\\

\begin{algorithm}[H]
    \caption{Second Approximation Algorithm}
    \Fn{BoundCount($E$,$current\_bound$)}{
        \For{$i=len(E)-1 \text{ to } 0$}{
            \If{$E[i].pid = last(E).pid$}{
                $lower\_block = i$ \;
                \Break
            }
        }

        \For{$i = lower\_block+1 \text{ to } len(E)-1$}{
            \If{$ E[lower\_block] \rightarrow E[i] \rightarrow last(E)$}{
                current\_bound++\;
                \Return
            }
        }
    }
\end{algorithm}

\subsection{Αξιολόγηση των Αλγορίθμων Προσέγγισης}
Οι προηγούμενες προσεγγίσεις που συζητήθηκαν παρουσάζουν κάποια ενδιαφέροντα αποτελέσματα. Και οι δύο αλγόριθμοι φαίνεται να είναι πιο 
``sound'' και από τον BPOR και εξερευνούν traces που ξεπερνούν το bound. Αυτό προκύπτει από το γεγονός ότι τείνουν να
υποεκτιμούν το preemption count καθώς πιο πολύπλοκες σχέσεις δεν αποκαλύπτονται.
Παρατηρούμε ότι στο writer-N-readers ο αριθμός των traces είναι σταθερός για κάθε bound. Αυτό οφείλεται στο όιτ κάθε thread αποτελείται από μόνο 
μια εντολή.

\graph{img/wNrLB.png}{writer-N-readers bounded by the first estimation algorithm}
\smalltabular{tables/lazy1_bounded.tex}{Traces for the first estimation algorithm for various bound limits}

\subsection{Υλοποίηση του Lazy-BPOR}

Τα προηγούμενα testcases δείξαν ότι μπορύμε να αποφύγουμε το state space explosion καθώς δεν προσθέτουμε συντηρητικές διακλαδώσεις.
Το επόμενο βήμα είναι η υλοποίηση του Lazy-BPOR, ενός αλγορίθμου που υπολογίζει με ακρίβεια τα compulsory switches.
Η διαφορά του από τον Naive-BPOR είναι οτι διατηρεί ένα γράφο καθ᾽ όλη τη διάρκεια εκτέλεσης του DPOR
Η προσθήκη των κόμβων γίνεται όπως έχει ήδη περιγραφεί.
Το preemption count γίνεται με τον υπολογσιμό του minimum hamiltonian path.

\begin{algorithm}
    \caption{Lazy-BPOR}
    \label{Lazy-BPOR}
    \Let{$G =: \emptyset$}
    Explore($\langle \rangle$,$\emptyset$,$G$,$b$)\;
    \Fn{Explore($E$,$Sleep$,$G$,$b$)}{
        \If{$\exists p \in (enabled(s_{[E]}) \backslash Sleep)$ such that $B_v(E.p) \leq b$ }{
            backtrack(E) $:={p}$ \;
            \While{$\exists p \in (backtrack(E) \backslash Sleep $}{
                \ForEach{$e \in dom(E)$ such that $e \lesssim_{E.p} next_{[E]}(p) $}{
                    \Let{$E' = pre(E,e)$}
                    \Let{$u = notdep(e,E).p$}
                    \If{$I_{E'}(u) \cap backtrack(E') = \emptyset$}{
                        add some $q' \in I_{[E']}(u) to backtrack(E') $ \;
                    }
                }
                \Let{$Sleep' := \{q \in Sleep \mid E \models p \diamondsuit q \} $}
                \If{$p$ creates a new block}{
                    \Let{$block$ = $last\_block(E)$}
                    \Let{$G'$ = add\_block($block$,$G$)}
                }
                \If{$min \{ Ham\_path(G') \text{ which compensate with all happens-before relations of } E \} \leq b $}{
                    $Explore(E.p, Sleep,G',b)$ \;
                    add $p$ to $Sleep$ \;
                }
            }
        }
    }
\end{algorithm}



\subsection{Lazy-BPOR - Αξιολόγηση στο RCU}

Τα αποτελέσματα παρουσιάζονται στη συνέχεια. Συγκρίνουμ τον Lazy-BPOR με την buggy εκδοχή του BPOR. 

Στα Figures \ref{Comparison between DPOR and Lazy-BPOR} και \ref{Comparison between BPOR and Lazy-BPOR} παρουσιάζουμε τα αποτελέσματα για τα διάφορα testcases του RCU.


\landscapetabular{"tables/lazy_comp.tex"}{Comparison between DPOR and Lazy-BPOR}

\landscapetabular{"tables/lazy_buged_comp.tex"}{Comparison between DPOR and Lazy-BPOR without the bug}


Συγκρίνοντας τα αποτελέσματα παρατηρούμε ότι ο Lazy-BPOR εξετάζει λιγότερα traces αλλά απαιτεί περισσότερο χρόνο καθώς ο υπολογισμός του preemption count απαιτεί
αρκετό χρόνο.

\landscapetabular{"tables/hline_pandas_lazy_preep.tex"}{Comparison between BPOR and Lazy-BPOR}

\section{Συμπεράσματα}
Παρόλο που ο Lazy-BPOR δεν είναι πιο αποδοτικός από τους υπόλοιπους αλγορίθμους που παρουσιάστηκαν σε αυτή τη διπλωματική παρουσιάζει μερικά ενδιαφέρονται αποτελέσματα.

\begin{itemize}
    \item Μπορούμε να εξερευνήσουμε το preemption-bounded state space χωρίς την προσθήκη conservative branches.
    \item Καθώς δεν προσθέτει συντηρητικές διακλαδώσεις δίνει ένα άνω φράγμα στον αριθμό των traces πολύ μικρότερο από αυτό του BPOR (τον αριθμό των traces της unbounded εκδοχής)
    \item Το πιο ενδιαφέρον είναι ότι ανάγει την preemption-bounded search σε ένα γνωστό γραφοθεωρητικό πρόβλημα το οποίο μπορεί να επιδεχθεί ευρηστικές για την 
        επιτάχυνση του υπολογισμού του hamiltonian path.
\end{itemize}

