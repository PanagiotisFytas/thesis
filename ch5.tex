\chapter{Parallelizing DPOR Algorithms}
\label{paradpor}

In this chapter, we are going to present the parallel version of source-DPOR and optimal-DPOR algorithms. Let us first
discuss some existing work in parallelizing persistent-set based DPOR algorithms.

\section{Existing Work}

At a first glance, parallelizing DPOR algorithms may seem straightforward. Since the state space of a program contains no cycles, 
we should simply distribute that state space into multiple workers-schedulers. For example, in the case of Algorithm \ref{Source}, 
we would assign a prefix of the form $E.p$ to a scheduler. That scheduler would be responsible for the execution of $explore(E.p, Sleep)$.
However this approach leads to two main issues \cite{Yang:2007:DDP:1770532.1770541}.

Firstly, DPOR algorithms detect races and update the exploration frontier in a non-local manner. While calls 
to $explore(E.p, Sleep)$ may guarantee that for all maximal execution 
sequences of form $E.p.w$, the algorithm has explored some execution sequence $E'$ which is in $[E.p.w]_\simeq$, backtrack points
may also be inserted in the prefixes of $E.p$. 
For instance, lets assume that we assign to one scheduler the exploration of $E.p$ and to another scheduler
the exploration of $E.q$. Both of those explorations may lead to adding the same process $r$ to the backtrack at $E$. This would mean that
both those scheduler would end up calling $explore(E.r, Sleep)$. Therefore, different schedulers may end up fully exploring
identical interleavings, despite the use of sleep sets. In order to combat those redundant explorations, 
Yang et al. \cite{Yang:2007:DDP:1770532.1770541} suggest a heuristic
that simply modifies the lazy addition of backtrack entries to the exploration frontier \cite{FlanaganDPOR} to become more eager.
Adding backtrack entries more eagerly, i.e., earlier in the exploration phase, reduces the chances of two different workers
exploring identical interleavings. However, this is simply a heuristic, which means that
depending on the tested program, a significant amount of redundant computations
may still occur. Particularly, this heuristic fails to prevent redundant explorations, when branches in the code lead to different races.

Secondly, the size of different chunks of the state space cannot be known preemptively. This means that some form of load balancing
is essential to achieve linear speedup.
Yang et al. \cite{Yang:2007:DDP:1770532.1770541} suggest using a centralized load balancer to unload work from a scheduler, whose
frontier exceeds a certain limit, to idle schedulers.

A more appropriate way to solve these issues is to use a centralized $controller$, which keeps track of the current 
$execution$ $tree$ (a tree whose branches correspond to the current execution sequences $E$ of the schedulers) in order
to assure that no redundant explorations occur \cite{Simsa2012ScalableDP}. This method also suggests the use of
$time$ $slicing$ to achieve load balancing. 

\section{Parallel source-DPOR}

Here we are going to present how to efficiently parallelize the source-DPOR algorithm, by modifying the parallel algorithm 
presented by Simsa et al. \cite{Simsa2012ScalableDP}.

\subsection{Algorithm Presentation}

Normally DPOR algorithms perform a depth-first search of the state-space to check for erroneous interleavings. Instead, we
are going to use multiple depth-first searches (by partitioning the frontier of our search) to explore our state-space.
Also we are going to use a centralized $Controller$ to oversee this exploration so we avoid any redundant computations.

\SetKwProg{Fn}{Function}{}{}
\SetKwHangingKw{Let}{let}
\begin{algorithm}
    \caption{Controller Loop}
    \label{controllerloop}
    \Fn{controller\_loop($N$, $Budget$, $Schedulers$)}{
        $E_0 \leftarrow$ an arbitrary initial execution sequence\;
        $Frontier \leftarrow[E_0]$\;
        $T \leftarrow$ an execution tree rooted at $E_0$\;
        \While{ $size(Frontier) > 0$} {
            $partition(Frontier, N)$\;
            \While{ exists an idle scheduler $S$ and an unassigned execution sequence $E$ in $Frontier$}{
                $E_c \leftarrow$ a copy of $E$\;
                $spawn(S, explore\_loop(E_c, Budget))$\;
            }
            $wait\_scheduler\_response(Frontier, T)$\;
        }
           
    }

\end{algorithm}

The logic of the Controller is described by Algorithm \ref{controllerloop}. The Controller will maintain a $Frontier$, 
which is a set of execution sequences $E$, and an execution tree $T$, which contains
as branches the execution sequences of the Frontier. For as long as there exists an execution sequence at the frontier 
($size(Frontier)>0$), the controller will partition its Frontier to at most $N$ execution sequences. Then, the controller
will try to assign all of its unassigned execution sequences to any idle scheduler, by spawning $explore\_loop(E_c, Budget)$
functions. Finally, it will block until it receives a response from a scheduler.


\begin{algorithm}
    \caption{Frontier Partitioning}
    \label{partition}
    \Fn{partition($Frontier$, $N$)}{
        \For{all E $\in$ Frontier}{
            \If{$size(Frontier) = N$}{
                return\;
            }
            \While{ $total\_backtrack\_entries(E) > 1$ \textbf{ and } $size(Frontier) < N$}{
                $E' \leftarrow $ \textnormal{the smallest prefix of $E$
                that has a backtrack entry} \;
                $p \leftarrow \textnormal{ a process} \in backtrack(E') $\;
                $E_c' \leftarrow \textnormal{ a copy of } E'$\;
                \textnormal{remove $p$ from $ backtrack(E')$}\;
                \textnormal{add $p$ to $ sleep(E')$}\;
                \textnormal{add $ backtrack(E') $ to $ sleep(E_c')$}\;
                \textnormal{add $E_c'$ to $ Frontier$}\;
            }
        }
    }
\end{algorithm}

During the $partitioning$ phase (Algorithm \ref{partition}), we inspect the current Frontier to determine whether
we should create additional execution sequences. Every execution sequence that contains more than one backtrack entry,
is split into multiple sequences until either the frontier contains $N$ sequences or all sequences
have exactly one backtrack entry. It is vital to modify sleep sets appropriately, because if we were to simply
remove backtrack entries, our algorithm would have an increased amount of sleep-set blocked interleavings.

\begin{algorithm}
    \caption{Scheduler Exploration Loop}
    \label{explore_loop}
    \Fn{explore\_loop($E_0$, $Budget$)}{
        $StartTime \leftarrow get\_time()$\;
        $ E \leftarrow E_0$\;
        \Repeat{$CurrentTime - StartTime > Budget \textbf{ or  }size(E'') \leq size(E_0)$}{
            $ E' \leftarrow explore(E)$\;
            $ plan\_more\_interleavings(E') $\;
            $ E'' \leftarrow get\_next\_execution\_sequence(E')$\;
            $CurrentTime \leftarrow get\_time()$\;
        }
        \textnormal{\textbf{send}  $E''$ to controller} \;
    }
\end{algorithm}

Algorithm \ref{explore_loop} details how the schedulers explore their assigned state space.
A call to $explore\_loop(E_0, Budget)$ guarantees that for all maximal execution 
sequences of form $E_0.w$, the algorithm has explored some execution sequence $E_0'$ which is in $[E_0.w]_\simeq$ 
We use $explore(E)$ and 
$ plan\_more\_interleavings(E') $ as a high level way to describe the main phases (state exploration and 
race detection) of the sequential source-DPOR. The $ plan\_more\_interleavings(E') $ function
could add backtrack points in prefixes of $E_0$. This could lead to different schedulers exploring
identical interleavings. We avoid this by having 
the function $ get\_next\_execution\_sequence(E')$ return 
the largest prefix of $E'$ that has a non empty backtrack set. This leads to a depth-first exploration of the assigned
state space before considering interleavings outside of the state space. The exploration
continues until we encounter a prefix of $E_0$ ( $size(E'') \leq size(E_0)$ ). This is necessary to 
assert that the specific scheduler will not explore interleavings outside of its state space. When the exploration
terminates, the backtrack points added to the prefixes of $E_0$ will be reported back to the controller.

\begin{algorithm}
    \caption{Handling Scheduler Response}
    \label{response}
    \Fn{wait\_scheduler\_response(Frontier, T)}{
        \textnormal{\textbf{receive} $E$ from a scheduler}\;
        $E' \leftarrow update\_execution\_tree(E, T)$\;
        \textnormal{add $E'$ to $Frontier$}\;
    }
\end{algorithm}


When the controller receives a response (an execution sequence $E$) from a scheduler (Algorithm \ref{response}), it will 
try and report any new
backtrack entries in $E$ to the execution tree $T$. Those backtrack entries that are not found
in the execution tree, are added to it and they are not removed from the execution sequence. This means that 
this execution sequence is the first to $claim$ $ownership$ over those entries and the state-space
that exists under them. Any backtrack entries that already exist in $T$, are removed from the 
execution sequence $E$ (and added to the sleep sets at the appropriate prefixes of $E$),
because some other execution sequence has already claimed their ownership. This updated execution sequence is then 
added to the frontier of the controller. Through this we make sure that two schedulers cannot explore identical
interleavings.

When updating the execution tree, we also use the initial execution sequence that was assigned to a scheduler
(the one denoted as $E_0$ at Algorithm \ref{explore_loop}) to figure out which parts of the execution tree have
already been explored. Those parts are deleted from the execution tree. 
This is mandatory in order to keep the size of the execution tree proportionate to the size of our current frontier.


\subsection{Load Balancing}

In order to achieve decent speedups and scalability it is necessary to have load-balancing \cite{Simsa2012ScalableDP}.
This done through time-slicing \cite{wiki:timeslice} the exploration of execution sequences. This is the reason
behind the use of $Budget$ in Algorithm \ref{explore_loop}. By having schedulers return after a certain time-slice,
we can make sure that even if their assigned state-space was larger compared to that of other schedulers, they will
eventually exit and have their execution sequence and subsequently, their state space, partitioned. How effective
is this method is determined by two variables, the upper limit $N$ to the number of execution sequences in our
frontier and the $Budget$ of a scheduler.

Higher upper limit $N$ means that a larger work pool is available to the workers and therefore, the utilization of the schedulers
increases.
However a larger $N$ also means increased memory requirements, since more execution sequences are active at each time.
Setting this limit to double the amount of schedulers, produces the decent results for most test cases\cite{Simsa2012ScalableDP}.

Smaller Budget values lead to more balanced workload, since the work is distributed more frequently. However,
extremely low values may lead to an increased communication overhead between the Controller and the schedulers.
This can also cause the controller to become a significant bottleneck. The best way to deal with this,
is to pick an initial value $B$ of around 1-10 seconds. When a scheduler starts a new exploration,
the value of its budget will be dynamically assigned by the Controller depending on the amount of
idle schedulers. For instance, the first execution should have a budget of $B \over n$ were $n$ is the total amount of
schedulers, which are all idle. When half the schedulers are idle this value should be $B \over 2$, etc. This makes it possible to have
reduced communication (higher budget) during periods with many busy schedulers and a better balancing
(lower budget) during periods with many idle schedulers.

\subsection{A Simple Example}


\begin{figure*}
    \begin{minipage}{0.3\textwidth}
      \begin{lstlisting}[frame=none, numbers=none]
        p:
        write(x)
      \end{lstlisting}
    \end{minipage}
    \begin{minipage}{0.3\textwidth}
        \begin{lstlisting}[frame=none, numbers=none]
            q:
            read(x)
            write(x)
        \end{lstlisting}
      \end{minipage}
      \begin{minipage}{0.3\textwidth}
        \begin{lstlisting}[frame=none, numbers=none]
            r:
            write(x)
        \end{lstlisting}
      \end{minipage}
      \caption{Simple readers-writes example}
      \label{sourceexample}
  \end{figure*}

Let us consider the example in Figure \ref{sourceexample}. In this case we have 3 processes that write and read a shared variable x.
Figure \ref{Interleavings explored by the sequential source-DPOR} represents the traces explored during the sequential source-DPOR.
We use a bold rectangle to represent a new event and a faint rectangle to denote a replayed event. The red edges represent
the races that are detected and planned. The source set at a state is represented inside the brackets.

\tracelonglong{source1.png}{Interleavings explored by the sequential source-DPOR}

\tracelong{source2f.png}{Initial interleaving explored by the parallel algorithm}

Figure \ref{Initial interleaving explored by the parallel algorithm} depicts the initial step of the parallel source-DPOR. 
This initial execution sequence, along with detected races is partitioned into fragments which get assigned to different schedulers.
This image also contains the initial execution tree which represents the state space that exists in our exploration frontier
at this point.

\tracelong{source3f.png}{Exploration of the assigned traces by each scheduler}

In Figure \ref{Exploration of the assigned traces by each scheduler} we illustrate how each scheduler explores its assigned 
execution sequence. The first scheduler explores its first trace: $p.r.q.q$ (equivalent to the second trace 
at Figure \ref{Interleavings explored by the sequential source-DPOR}). After the race detection takes place,
there are no backtrack entries added below its assigned trace. The only execution sequence planned is the sequence $r$. However,
this sequence is not explored since it does not belong to the state space of the scheduler. The controller assigned to the
second scheduler the execution sequence $q$. After exploring its first trace, two more races are detected. It is important to notice
here that the first trace of the second scheduler is equivalent to the third trace of the sequential algorithm. However,
the sequential algorithm detects only one race. This happens because on the sequential algorithm the race between $q: read(x)$ and
$r:write(x)$ had already been detected at the second interleaving and so its planning gets skipped. 
On the contrary, on the parallel algorithm this interleaving
is explored by another scheduler and therefore, there is no knowledge of this race been already detected. This leads to
both our schedulers having detected the same race. Nevertheless, in both schedulers this new backtrack entry is outside of their
state space. 

This means that they will have to report their results back to the controller. The schedulers that reports first its results, 
will be the one to update the execution tree by inserting the new entry found. This
scheduler will add its execution sequence to the frontier, which will be again partitioned
(no need for a partition here since the unexplored frontier will only have one race). Then this execution sequence will be
assigned to an idle scheduler. The scheduler that reports second to the controller, will not be able to insert its 
backtrack entry into the execution tree, because that entry will already be there, and the backtrack entry gets removed
from its execution sequence. This execution sequence will be left with no more backtrack entries and as such it will not be
inserted into the frontier. This guarantees that we do not explore identical interleavings more than once.

Lets assume that Scheduler 1 was the one that managed to report first to the Controller. Then the execution tree will be the one
depicted in Figure \ref{Execution Tree if Scheduler 1 returned first}. Notice here that the states explored by the Scheduler 1
were deleted from the execution tree. This keeps the size of the execution tree proportionate to the size of 
the current exploration frontier. At this point Scheduler 1 will add its execution sequence to the frontier. Then the 
controller will assign this sequence to Scheduler 1 (since Scheduler 2 has not yet returned), and Scheduler 1 will explore
the last 2 traces (trace 5 and 6 from the sequential example). After both Scheduler 1 and 2 have returned the execution 
will have finished since there will no more traces left to explore.


\tracelong{source4f.png}{Execution Tree if Scheduler 1 returned first}

\section{Parallel optimal-DPOR}

\subsection{Basic Idea}

Parallelizing the optimal-DPOR algorithm is significantly more complicated. We do know that whenever a call to 
$Explore(E, Sleep, WuT)$ returns during Algorithm \ref{optimal}, then for all maximal execution sequences
of form $E.w$, the algorithm has explored some execution sequence $E'$ which is in $[E.w]_\simeq$ \cite{AbdullaAronisJohnssonSagonasDPOR2014}.
However, the complete $WuT$ at some execution sequence $E$ cannot be known until we have completed
exploring all execution sequences which are ordered before $E$, according to the total order of our state space
(Definition \ref{def:Ordered}). This happens because the $insert_{[E']}(v,wut(E'))$ function can add
entries to any wakeup tree of an execution sequence that is ordered after the current execution sequence.

Therefore, when assigning an incomplete wakeup tree to a scheduler there is no
guarantee that the scheduler will explore the complete assigned state space. This means that if a
scheduler inserts a fragment into a wakeup tree owned by a different scheduler, we cannot know if that 
fragment (or a different but equivalent fragment) was indeed explored, unless there is some form of
centralized race detection.

However, we do know that any leaf of $\langle B , \prec \rangle$ remains a leaf of $insert_{[E]}(w,\langle B , \prec \rangle)$
\cite{AbdullaAronisJohnssonSagonasDPOR2014}. This means that during the sequential algorithm, any fragment that
is inserted into a wakeup tree is a fragment that must be explored. Therefore, when we insert a fragment into a wakeup tree,
we can explore it out of order. We can take advantage of this to create an algorithm that can explore many interleavings in parallel
but race detects each explored interleaving sequentially.

\subsection{Algorithm Presentation}

Due to the fact that we need to have parallel exploration of interleavings but sequential planning, we need to decouple the
the normal exploration loop of a scheduler into two different parts: $state$ $exploration$ and $race$ $detection-planning$.
Our workers (the Schedulers) will be responsible for the first part. For the second part, we are going to use a centralized
planner. However, in order to be able to better distribute the available work to the schedulers when the planner is busy,
we are also going to use a Controller.

\begin{algorithm}
    \caption{Optimal Controller}
    \label{optcontrollerloop}
    \Fn{controller\_loop($Schedulers$)}{
        $E_0 \leftarrow$ an arbitrary initial execution sequence\;
        $Frontier \leftarrow[E_0]$\;
        $T \leftarrow$ an execution tree rooted at $E_0$\;
        $PlannerQueue \leftarrow empty$\;
        \While{ $size(Frontier) > 0$ \textnormal{and} $ size(PlannerQueue) > 0$} {
            $partition(Frontier)$\;
            \While{ exists an idle scheduler $S$ and an unassigned execution sequence $E$ in $Frontier$}{
                $E_c \leftarrow$ a copy of $E$\;
                $spawn(S, explore(E_c))$\;
            }
            \While{ the Planner is idle and PlannerQueue $\neq$ empty }{
                $E \leftarrow PlannerQueue.pop()$\;
                $update\_trace(E, T)$\;
                $spawn(Planner, plan(E))$\;
            }
            $wait\_response(Frontier, T, PlannerQueue)$\;
        }
           
    }

\end{algorithm}

Algorithm \ref{optcontrollerloop} describes the functionality of the Controller. Similarly to the source-DPOR parallel version,
the Controller is responsible for maintaining the current Frontier (as well as partitioning it) and the current Execution Tree
and for assigning execution sequences to schedulers, for as long we have idle schedulers and available work.

Apart from that, the Controller also maintains a queue of fully explored execution sequences that need to be race detected.
When the Planner is idle and the queue is not empty, the execution sequence is updated (through $update\_trace(E,T)$)
and then is sent to the Planner so its races can be detected. When updating the execution sequence from the execution tree,
the subtrees of the execution tree which are ordered after the execution sequence (according to the ordering of our state
space Definition \ref{def:Ordered}) are inserted into the execution tree as $not\_owned$ wakeup trees. This guarantees
that no redundant fragments are going to be inserted for future explorations and therefore, the algorithm remains optimal. 

The $plan(E)$ function race detects the fully explored execution sequence $E$ according to the logic of optimal-DPOR 
(Algorithm \ref{optimal}). When the planning of the sequence is finished the results are reported back to the Contoller.
The $explore(E)$ function explores the execution sequence $E$ until a maximal execution sequence
has been reached and reports back that execution sequence to the Controller.


\begin{algorithm}
    \caption{Optimal Frontier Partitioning}
    \label{optpartition}
    \Fn{partition($Frontier$)}{
        \For{all E $\in$ Frontier}{
            \While{ $wakeup\_tree\_leaves(E) > 1$ }{
                $E' \leftarrow $ \textnormal{a prefix of $E$
                with $wut(E') \neq \emptyset $} \;
                $v \leftarrow \textnormal{ a leaf} \in wut(E') $\;
                $E_c' \leftarrow \textnormal{ a copy of } E'$\;
                \textnormal{mark $v$ as $not\_owned$ at $ wut(E')$}\;
                $\{Prefix, v, Suffix\} \leftarrow split\_wut\_at(v, wut(E_c'))$\;
                \textnormal{add the first processes of $Prefix$ to $ sleep(E_c')$}\;
                \textnormal{mark $ Suffix $ as $not\_owned$ at $ wut(E_c')$}\;
                \textnormal{add $E_c'$ to $ Frontier$}\;
            }
        }
    }

\end{algorithm}

Partitioning the exploration frontier (Algorithm \ref{optpartition}) has two main differences, compared to the parallel source-DPOR.
Firstly, the frontier gets partitioned completely, so we can maximize the parallelization of the exploration phase. 
Secondly, the entries that are distributed from one execution sequence, are not simply removed from the backtrack and added to
the sleep set. It is vital here to maintain the correct ordering between the interleavings (Definition \ref{def:Ordered}). Therefore,
the given entry simply is marked as $not\_owned$ at the distributed sequence. The function  $split\_wut\_at(v, wut(E_c'))$
splits the copy of the wakeup tree to 3 parts: the $Prefix$ (the wakeup tree entries ordered before the branch $v$), the leaf $v$
and the $Suffix$ (the wakeup entries ordered after $v$). The first processes processes of the entries of
the $Prefix$ are added to the sleep set at the new execution sequence $E_c'$ (e.g. if $p.q.r$ is a leaf in $Prefix$, then $p$ is added to
$sleep(E_c')$). The $Suffix$ entries are marked as $not\_owned$ at $E_c'$.

\begin{algorithm}
    \caption{Handling Scheduler and Planner Response}
    \label{optresponse}
    \Fn{wait\_response(Frontier, T, PlannerQueue)}{
        \textbf{receive} a message M\;
        \uIf{ M is sent from a Scheduler}{  
            $E \leftarrow M$\;
            $PlannerQueue.push(E)$\;
        }
        \uElseIf{M is sent from the Planner}{
            $E \leftarrow M$\;
            $update\_execution\_tree(E, T)$\;
            \textnormal{add $E$ to $Frontier$}\;
        }
    }
\end{algorithm}

After assigning the available work to the available schedulers and the Planner, the Controller will wait for a response
either from a scheduler or the Planner (Algorithm \ref{optresponse}). When a response is received from a scheduler,
the fully explored received execution sequence will be added to the queue of the Planner and the Controller will continue
with its loop (Algorithm \ref{optcontrollerloop}). If a response is received from the Planner, the Controller will update
the execution tree $T$ by adding the new wakeup trees that were inserted by the planner and by deleting the suffix of
the execution sequence that was just explored and has no wakeup trees. We delete this part in order to have the 
execution tree only contain the part of the state space that is either currently getting explored or is planned to be explored.
If we were to not delete those suffixes the size of the execution tree would eventually be the size of our complete state space.

\subsection{Example}

\begin{figure*}
    \begin{minipage}{0.3\textwidth}
      \begin{lstlisting}[frame=none, numbers=none]
        p:
        i = N
        while x[i] != 0 and i > 0:
            i = i - 1
      \end{lstlisting}
    \end{minipage}
    \begin{minipage}{0.3\textwidth}
        \begin{lstlisting}[frame=none, numbers=none]
            q:
            R1 = x[0]
            R2 = x[0]
            assert(R1 == R2)
            x[1] = R2 + 1
        \end{lstlisting}
      \end{minipage}
      \begin{minipage}{0.3\textwidth}
        \begin{lstlisting}[frame=none, numbers=none]
            r:
            R3 = x[0]
            R4 = x[1]
            assert(R3 == R4)
            x[2] = R4 + 1
        \end{lstlisting}
      \end{minipage}
      \caption{$Lastzero$ 2 example}
      \label{optexample}
  \end{figure*}

In Figure \ref{optexample} we can se the pseudocode of $lastzero$ 2, were we have an array of 3 elements (initially all elements have
a zero value) and 3 processes. The first process ($p$) searches the array for the zero element with the highest index. 
The other two processes increase their assigned element by a value of 1.

\tracelonglong{opt1.png}{Interleavings explored by the sequential optimal-DPOR}

Figure \ref{Interleavings explored by the sequential optimal-DPOR} represents 
the traces explored during the sequential optimal-DPOR.
We use a black bold rectangle to represent a new event and a faint rectangle to denote a replayed event. 
The continuous red edges represent the races that are detected and planned. The red nodes represent the wakeup tree entries
at each trace.

\tracelonglong{opt3.png}{Initial interleaving explored by the parallel optimal-DPOR}

Figure \ref{Initial interleaving explored by the parallel optimal-DPOR} depicts the initial step of the parallel optimal-DPOR. 
An arbitrary execution sequence is explored initially and then its races are detected and planned in the form of wakeup trees.
The wakeup trees are distributed into different fragments and all unassigned fragments are assigned into the idle schedulers.
Also the execution tree is initialized with the current exploration frontier.

In this example, lets assume that Scheduler 2 finishes first the exploration of its assigned trace. The controller will
then receive the new explored trace and will add this trace to the queue of the Planner. Since the Planner is idle,
this trace will be sent to the Planner to be race detected. While race detecting this trace, no more interleavings will be 
planned. This trace is equivalent to the 3rd trace of the sequential execution 
(Figue \ref{Interleavings explored by the sequential optimal-DPOR}). Notice here that in the sequential algorithm
this trace had an additional wakeup tree. This wakeup tree was planned by the 2nd trace of the sequential algorithm,
which has yet to be race detected in our example. Therefore, traces 4 and 5 of the sequential algorithm cannot be
planned from the 3rd trace but only from the 2nd. This makes apparent the main issue of the parallelization of the optimal-DPOR:
the complete wakeup tree at some execution sequence $E$ cannot be known until we have completed
exploring all execution sequences which are ordered before $E$, according to the total order of our state space
(Definition \ref{def:Ordered})