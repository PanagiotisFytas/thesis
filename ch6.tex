\chapter{Performance Evaluation}
\label{perfresults}

In this chapter we are going to present the performance results of our parallel source-DPOR and optimal-DPOR algorithms,
implemented in Concuerror. We are going to evaluate our results on some standard and synthetic programs that are normally
used to test DPOR algorithms. Finally, we will try to explain the behavior of the parallel program as reflected in those
charts, drawing related conclusions whenever possible.

\section{Tests Overview}

First we are going to give a brief overview of the programs tested:

\begin{itemize}
    \item $indexer$ $N$: This test uses a Compare and Swap (CAS) primitive instruction to check if a specific element of
    a matrix is set to 0 and if so, set it to a new value. This implemented in erlang by using $ETS$ tables and specifically
    the $insert\_new/2$ function. This function return false if the key of the inserted tuple exists (the entry is set to 0)
    or it inserts the tuple if the key is not found.
    \item $readers$ $N$: This benchmark uses a writer process that writes a variable and N reader processes that read that variable.
    \item $writers$ $N$: This is a modification of the readers test. Here we have a reader process that reads a variable and 
    N writer processes that write that variable. The fact that we have many writers creates more races and therefore more work
    for our program.
    \item $ lastzero$ $N$: In this test we have N+1 processes that read and write on an array of N+1 size, which has all its 
    values initialized with zero. The first process reads the array in order to find the zero element with the highest
    index. The other N processes read an array element and updated the next one.
\end{itemize}

Table \ref{Sequential Source-DPOR vs Optimal-DPOR for our benchmarks} contains information about the traces explored and duration of those
explorations for the sequential versions of Source-DPOR and Optimal-DPOR.

\smalltabular{"tables/synthetic_unbounded.tex"}{Sequential Source-DPOR vs Optimal-DPOR for our benchmarks}

The benchmarks were performanced on a multiprocessor with 64 AMD Opteron 6276(2.3 GHz) cores, 126 GB of memory, running
Linux 4.9.0-8amd64 and running the later Erlang version (Erlang/OTP 21.1). While running our tests, we are using the
--$keep\_going$ flag to continue exploring our state space, even after an error is found. We do this so we can evaluate
how fast the complete state space gets explored.

\section{Source DPOR}

\subsection{Preformance Results}

Here we are going to present graphs depicting the execution time and the speedup ($T_{serial} 
\over T_{parallel}$) of the source DPOR algorithm
for different numbers of schedulers and for various test cases. Also we are going combine the 
speedups for those test cases in a comparative graph (Figure \ref{Speedup comparison for source-DPOR}).

(Note: more benchmarks will be added)

\tracelonglong{scripts/indexerCombined.png}{Performance for indexer 15}
\tracelonglong{scripts/readers15combined.png}{Performance for readers 15}
\tracelonglong{scripts/lastzero11combined.png}{Performance for lastzero 11}
\mediumGraph{scripts/allspeedup.png}{Speedup comparison for source-DPOR}
\iffalse
\mediumGraph{scripts/indexer15_time.png}{Execution time for indexer 15}
\mediumGraph{scripts/readers15_time.png}{Execution time for readers 15}
\mediumGraph{scripts/writers11_time.png}{Execution time for writers 11}

\mediumGraph{scripts/indexer15_speedup.png}{Speedup for indexer 15}
\mediumGraph{scripts/readers15_speedup.png}{Speedup time for readers 15}
\mediumGraph{scripts/writers11_speedup.png}{Speedup time for writers 11}
\fi

\subsection{Performance Analysis}

As we can see in our charts, our parallel implementation of source-DPOR in Concuerror significantly reduces the testing
time of our test cases. Namely, in the cases of both lastzero and indexer we have a speedup by a factor of 1.9 for 2 schedulers and
of 3.9 for 4 schedulers. This makes parallel source DPOR highly usable in personal computers.

Additionally, we are getting decent scalability in test cases like lastzero11 and indexer15. Particularly, with
32 schedulers we manage as speedup by a factor of 16 in the case of lastzero. However, through the scalability charts 
we can notice that with more schedulers the scalability of those test cases starts to decline. This happens because we
are using a centralize controller which becomes a bottleneck, since the more the schedulers the
higher the chance the controller will be busy and therefore, unable to distribute work. The most important reason 
behind this drop in scalability though, is the fact that with more available schedulers the assignment of the state
space of the program becomes more fine grained. A more fine grained assignment leads to the schedulers exploring their
assigned state space faster and having more races found outside their state space (since the state space is smaller).
Consequently, the communication with the controller is more frequent. This situation is problematic for two reasons:
the communication between a scheduler and the controller has a non-negligible overhead and when the amount of 
communications between the schedulers and the controller increases, the controller becomes an even bigger bottleneck.
As such, we come to the conclusion that a test case cannot scale beyond a certain point based on the number of its interleavings.

From Figure \ref{Speedup comparison for source-DPOR} we can notice the scalability of readers15 breaks significantly faster
compared to that of the other test cases, despite the fact that readers15 has more explored traces than indexer15
(Table \ref{Sequential Source-DPOR vs Optimal-DPOR for our benchmarks}). This means that number of the traces explored
is not the only scalability factor and different test cases with equivalent sizes can have varied results.

\section{Optimal DPOR}

\subsection{Preformance Results}

Our parallel optimal-DPOR fails to achieve any type of speedup, as depicted in Figure 
\ref{Execution time for readers 10 by optimal-DPOR}. Table \ref{Parallel Optimal-DPOR performance}
holds information about how our implementation runs on various test cases. The exploration and planning
time row show what percentage of the time of the sequential algorithm is spend on exploring and planning interleavings.
Those measurements will help us explain the reason for this lack of speedup.

\mediumGraph{scripts/readers10_timedd.png}{Execution time for readers 10 by optimal-DPOR}


\smalltabular{"tables/opttab.tex"}{Parallel Optimal-DPOR performance}

\subsection{Performance Analysis}

This lack of performance is completely justified through the exploration and planning percentages of the sequential algorithm,
presented at Table \ref{Parallel Optimal-DPOR performance}. While the percentage of exploration time in the case of a small test
case like readers10 is around 40$\%$, we have observed that for larger test cases this figure varies between 10$\%$-30$\%$. 
This means that even on an ideal setup, with zero overhead and infinite schedulers, our speedup could never exceed a factor
of 1.429 (on a test case with 70$\%$ planning time, if we consider that the exploration phase happens instantly the 
speedup would be 1.429 = $100\over 70$, since the planning phase would still have to take place sequentially). 
This reason, by itself, makes our algorithm impossible to achieve good performance and scalability.

To make matters worse, our algorithm also has a significant overhead. As mentioned before, communication between
the controller and the workers can be a substantial bottleneck. The parallel source-DPOR deals with it by assigning
a state space to the workers, minimizing the need for communication. However, in the parallel optimal-DPOR algorithm
the need for central planning leads to the scheduler having to report a trace back to the controller every sigle
time an exploration reaches its end. Also, in this case, before the centralized planner can race detect a trace,
we have to update this trace with the wakeup trees that ordered after the trace
(Definition \ref{def:Ordered}) and have been inserted into the execution tree
since the last time this trace was planned (Algorithm \ref{optcontrollerloop}).

\section{Final Comments}

The fact that optimal-DPOR lacks speedup does not mean that we have not benefited from our parallelization. 
The good resutls produced by the parallel source-DPOR help the source-DPOR algorithm outperform the sequential optimal-DPOR.

When there are no sleep-set blocked interleavings the sequential source-DPOR
can run faster than the optimal-DPOR (Table \ref{Sequential Source-DPOR vs Optimal-DPOR for our benchmarks}). 
In the case of readers15, for instance, the sequential source-DPOR is already faster by the optimal algorithm by
around 13 minutes. If we combine this with speedup of the parallel version, the benefit of our parallelization becomes clear.

Still, even when we have a significant amount of sleep-set blocked interleavings, like in the case of lastzero11, our parallel
source-DPOR can fairly easily catch up with the optimal-DPOR. Specifically, the optimal algorithm has an execution time of 23m32.843s
and explores 7168 interleavings, while the source-Algorithm has an execution time of 50m39.201s and explores 60073 interleavings.
By using two schedulers, the runtime of source-Algorithm drops to 26m33.051s and with four schedulers to 13m19,016s. As we add
schedulers source-DPOR will outperform the optimal algorithm even more.

