\chapter{Parallelizing Source-DPOR Algorithm}
\label{paradpor}

Concuerror utilizes primarily DPOR algorithms to systematically test concurrent Erlang programs. 
Therefore, parallelizing Concuerror entails designing parallel versions for its DPOR algorithms. 
Since Concuerror is written in Erlang, which is a functional language that is based on message passing to share 
data between processes, we are going to follow a message passing approach, while developing our algorithms.

In this chapter, we are going to present the parallel version of the source-DPOR algorithm. Let us first
discuss some existing work in parallelizing persistent-set based DPOR algorithms.

\section{Existing Work}

When parallelizing DPOR algorithms, ideally we would like to develop parallel algorithms that explore
exactly the same number of interleavings as their sequential versions. In other words, we want to retain
the soundness of our algorithms, while simultaneously not exploring more interleavings than necessary

At a first glance, parallelizing DPOR algorithms may seem straightforward. Since the state space of a program contains no cycles, 
we should simply distribute that state space into multiple workers-schedulers. For example, in the case of Algorithm \ref{Source}, 
we would assign a prefix of the form $E.p$ to a scheduler. That scheduler would be responsible for the execution of $explore(E.p, Sleep)$.
However this approach leads to two main issues \cite{Yang:2007:DDP:1770532.1770541}.

Firstly, DPOR algorithms detect races and update the exploration frontier in a non-local manner. While calls 
to $explore(E.p, Sleep)$ may guarantee that for all maximal execution 
sequences of form $E.w$, the algorithm has explored some execution sequence $E'$ which is in
$[E.w]_\simeq$, backtrack points may also be inserted in the prefixes of $E.p$. 
For instance, lets assume that we assign to one scheduler the exploration of $E.p$ and to another scheduler
the exploration of $E.q$. Since we are using message passing
as our programming model, our schedulers would use different copies
of the prefix $E$. Both of those explorations may lead to adding the same process $r$ to the backtrack at $E$. This would mean that
both those schedulers would end up calling $explore(E.r, Sleep)$. Therefore, different schedulers may end up fully exploring
identical interleavings. In order to combat those redundant explorations, 
Yang et al. \cite{Yang:2007:DDP:1770532.1770541} suggest a heuristic
that simply modifies the lazy addition of backtrack entries to the exploration frontier \cite{FlanaganDPOR} to become more eager.
Adding backtrack entries more eagerly, i.e., earlier in the exploration phase, reduces the chances of two different workers
exploring identical interleavings. In the above scenario, this strategy could have led to $r$ being added to the backtrack at $E$ before
assigning the exploration of $E.p$ and $E.q$ to different schedulers. Therefore, both of our schedulers would have know that $r$ exists in
the backtrack of $E$ and none of them would have added it,
avoiding the duplicate exploration of $E.r$. However, this is simply a heuristic, which means that
depending on the tested program, a significant amount of redundant computations
may still occur. Particularly, this heuristic fails to prevent redundant explorations, when branches in the code lead to different races.

Secondly, the size of different chunks of the state space cannot be known a priori. This means that some form of load balancing
is essential to achieve linear speedup.
Yang et al. \cite{Yang:2007:DDP:1770532.1770541} suggest using a centralized load balancer to unload work from a scheduler.
Specifically, a scheduler calls the load balancer when the total
number of backtrack entries in the execution sequence of the scheduler
exceed a threshold. However, for different programs and different number of workers, different threshold
values should be used \cite{Simsa2012ScalableDP}. Still, Yang et al. provide no insight into the problem
of selecting an appropriate threshold.

Simsa et al. \cite{Simsa2012ScalableDP} provide
a more appropriate way to solve these issues. By using a centralized $Controller$, which keeps track of the current 
$execution$ $tree$ (a tree whose branches correspond to the current execution sequences $E$ of the schedulers), they assure that no redundant explorations occur. They also suggests the use of
$time$ $slicing$ to achieve load balancing.

\section{Parallel source-DPOR}

Here we are going to present how to efficiently parallelize the source-DPOR algorithm, by modifying the parallel algorithm 
presented by Simsa et al. \cite{Simsa2012ScalableDP}.

\subsection{Basic Idea}

Normally, DPOR algorithms perform a depth-first search of the state space to check for erroneous interleavings. Instead, we
are going to use multiple depth-first searches (by partitioning the frontier of our search) to explore our state space.

We are going to use a centralized $Controller$ who will be
responsible for distributing the exploration
frontier to different worker-schedulers. The Controller is
also going to oversee the parallel exploration, so we avoid exploring more interleavings than the sequential version.

In order to avoid redundant explorations, we are going to use the
concept of $ownership$ \cite{Simsa2012ScalableDP} of a node of a state space. A scheduler exclusively owns a node of the state space if it is
either contained  as a backtrack entry within the part of the frontier that was assigned 
to that specific scheduler, or if it is a descendant of a node
that the scheduler owns. All other nodes, are considered to have a
$disputed$ ownership. 

The scheduler, when conducting its depth-first search, is going to be
allowed to explore only nodes that it owns. When encountering disputed
nodes, the scheduler will report back to the Controller, which will 
keep track of the complete active frontier. If that disputed node
does not exist within the complete frontier, then no other 
scheduler has explored that node and therefore, the scheduler
can $claim$ $ownership$ over the disputed node and continue with 
exploring it. Otherwise, the ownership of the disputed node
has been claimed by some other scheduler and therefore, that node
can be discarded from the frontier of our scheduler.

For example, if a scheduler is responsible
for exploring an execution sequence $E$ that has a single backtrack 
entry of $p$ at $E$, then that scheduler owns every node
that is a descendant of $E.p$ i.e., it owns every execution sequence
that begins with the prefix $E.p$ ($[E.p]_\simeq$). Now lets assume that during the exploration of  $[E.p]_\simeq$, $r$ has been added at
the backtrack at $E$. Since this node is disputed, the scheduler
will not explore this backtrack entry. Instead, it will report
back to the Controller, in order to determine whether some other entry
has already claimed ownership over $E.r$. 


\subsection{Algorithm}

\SetKwProg{Fn}{Function}{}{}
\SetKwHangingKw{Let}{let}
\begin{algorithm}
    \caption{Controller Loop}
    \label{controllerloop}
    \Fn{controller\_loop($N$, $Budget$, $Schedulers$)}{
        $E_0 \leftarrow$ an arbitrary initial execution sequence\;
        $Frontier \leftarrow[E_0]$\;
        $T \leftarrow$ an execution tree rooted at $E_0$\;
        \While{$Frontier \neq \emptyset$} {
            $Frontier \leftarrow partition(Frontier, N)$\;
            \While{exists an idle scheduler $S$ and an unassigned execution sequence $E$ in $Frontier$}{
                $E_c \leftarrow$ a copy of $E$\;
                mark $E$ as assigned in $Frontier$\;
                $spawn(S, explore\_loop(E_c, Budget))$\;
            }
            $Frontier,T \leftarrow wait\_scheduler\_response(Frontier, T)$\;
        }
           
    }

\end{algorithm}

The logic of the Controller is shown in Algorithm \ref{controllerloop}. The Controller maintains a $Frontier$, 
which is a set of execution sequences $E$, and an execution tree $T$, which contains
as branches the execution sequences of the $Frontier$. For as long as there exists an execution sequence at the $Frontier$ 
($Frontier \neq \emptyset$), the Controller will partition its $Frontier$ to at most $N$ execution sequences. Then, the Controller
will try to assign all of its unassigned execution sequences to any idle scheduler, by spawning $explore\_loop(E_c, Budget)$
functions. Finally, it will block until it receives a response from a scheduler.


\begin{algorithm}
    \caption{Frontier Partitioning}
    \label{partition}
    \Fn{partition($Frontier$, $N$)}{
        \For{all E $\in$ Frontier}{
            \While{ $total\_backtrack\_entries(E) > 1$ \textbf{ and } $size(Frontier) < N$}{
                $E' \leftarrow $ \textnormal{the smallest prefix of $E$
                that has a backtrack entry} \;
                $p \leftarrow \textnormal{ a process} \in backtrack(E') $\;
                $E_c' \leftarrow \textnormal{ a copy of } E'$\;
                \textnormal{remove $p$ from $ backtrack(E')$}\;
                \textnormal{add $p$ to $ sleep(E')$}\;
                \textnormal{add $ backtrack(E') $ to $ sleep(E_c')$}\;
                \textnormal{add $E_c'$ to $ Frontier$}\;
            }
        }
        return $Frontier$\;
    }
\end{algorithm}

During the $partitioning$ phase (Algorithm \ref{partition}), we inspect the current $Frontier$ to determine whether
we should create additional execution sequences. Every execution sequence that contains more than one backtrack entry
is split into multiple sequences until either the $Frontier$ contains $N$ sequences or all sequences
have exactly one backtrack entry. It is vital to modify sleep sets appropriately because, if we were to simply
remove backtrack entries, our algorithm would have an increased amount of sleep-set blocked interleavings. In addition, we would also 
end up potentially re-adding the same backtrack entries, which would lead
to exploring duplicate interleavings.

\begin{algorithm}
    \caption{Scheduler Exploration Loop}
    \label{explore_loop}
    \Fn{explore\_loop($E_0$, $Budget$)}{
        $StartTime \leftarrow get\_time()$\;
        $ E \leftarrow E_0$\;
        \Repeat{$CurrentTime - StartTime > Budget \textbf{ or  }size(E'') \leq size(E_0)$}{
            $ E' \leftarrow explore(E)$\;
            $ plan\_more\_interleavings(E') $\;
            $ E'' \leftarrow get\_next\_execution\_sequence(E')$\;
            $CurrentTime \leftarrow get\_time()$\;
        }
        \textnormal{\textbf{send}  $E''$ to Controller} \;
    }
\end{algorithm}

Algorithm \ref{explore_loop} details how the schedulers explore their assigned state space.
A call to $explore\_loop(E_0, Budget)$ guarantees that for all maximal execution 
sequences of form $E_0.w$, the algorithm has explored some execution sequence $E_0'$ which is in $[E_0.w]_\simeq$ 
We use $explore(E)$ and 
$ plan\_more\_interleavings(E') $ as a high level way to describe the main phases (state exploration and 
race detection) of the sequential source-DPOR. The $ plan\_more\_interleavings(E') $ function
could add backtrack points in prefixes of $E_0$. This could lead to different schedulers exploring
identical interleavings. We avoid this by having 
the function $ get\_next\_execution\_sequence(E')$ return 
the largest prefix of $E'$ that has a non-empty backtrack set. This leads to a depth-first exploration of the assigned
state space before considering interleavings outside of the state space. The exploration
continues until we encounter a prefix of $E_0$ ( $size(E'') \leq size(E_0)$ ). This is necessary to 
assert that the specific scheduler will not explore interleavings outside of its state space. When the exploration
terminates, the backtrack points added to the prefixes of $E_0$ will be reported back to the Controller.

\begin{algorithm}
    \caption{Handling Scheduler Response}
    \label{response}
    \Fn{wait\_scheduler\_response(Frontier, T)}{
        \textnormal{\textbf{receive} $E$ from a scheduler}\;
        remove $E$ from $Frontier$\;
        $E',T \leftarrow update\_execution\_tree(E, T)$\;
        \If{$E'$ has at least one backtrack entry}{
                \textnormal{add $E'$ to $Frontier$}\;
        }
        return $Frontier,T$;
    }
\end{algorithm}


When the Controller receives a response (an execution sequence $E$) from a scheduler (Algorithm \ref{response}), it will 
try and report any new
backtrack entries in $E$ to the execution tree $T$. This is done
through the $update\_execution\_tree(E, T)$ function. This function
iterates over the execution sequence and the execution tree simultaneously and those backtrack entries of $E$ that are not found
in the execution tree, are added to it and they are not removed from the execution sequence. This means that 
this execution sequence is the first to $claim$ $ownership$ over those entries and the state space
that exists under them. Any backtrack entries that already exist in $T$ are removed from the 
execution sequence $E$ (and added to the sleep sets at the appropriate prefixes of $E$),
because some other execution sequence has already claimed their ownership. This updated execution sequence is then 
added to the $Frontier$ of the Controller. Through this we make sure that two schedulers cannot explore identical
interleavings.

When updating the execution tree, we also use the initial execution sequence that was assigned to a scheduler
(the one denoted as $E_0$ at Algorithm \ref{explore_loop}) to figure out which parts of the execution tree have
already been explored. Those parts are deleted from the execution tree. 
This is mandatory in order to keep the size of the execution tree proportionate to the size of our current $Frontier$.


\subsection{Load Balancing}

In order to achieve decent speedups and scalability it is necessary to have load-balancing \cite{Simsa2012ScalableDP}.
This done through time-slicing the exploration of execution sequences. This is the reason
behind the use of $Budget$ in Algorithm \ref{explore_loop}. By having schedulers return after a certain time-slice,
we can make sure that even if their assigned state space was larger compared to that of other schedulers, they will
eventually exit and have their execution sequence and subsequently, their state space, partitioned. How effective
is this method is determined by two variables, the upper limit $N$ to the number of execution sequences in our
$Frontier$ and the $Budget$ of a scheduler.

Higher upper limit $N$ means that a larger work pool is available to the workers and they do not always have to wait
for the $update\_execution\_tree(E, T)$ function to terminate. Therefore, the utilization of the schedulers
increases.
However, a larger $N$ means increased memory requirements, since more execution sequences are active at each time. More importantly, 
it also means that the state space splits into smaller fragments. This increases how frequently a scheduler discovers disputed nodes, which
leads to an increase in the communication between the schedulers and the controller.
Setting this limit to double the amount of schedulers, produces decent results for most test cases\cite{Simsa2012ScalableDP}.

Smaller $Budget$ values lead to more balanced workload, since the work is distributed more frequently. However,
extremely low values may lead to an increased communication overhead between the Controller and the schedulers.
This can also cause the Controller to become a significant bottleneck. The best way to deal with this,
is to pick an initial value $Budget$ of around 1-10 seconds. When a scheduler starts a new exploration,
the value of its budget will be dynamically assigned by the Controller depending on the amount of
idle schedulers. For instance, the first execution should have a budget of $Budget \over n$ were $n$ is the total amount of
schedulers, which are all idle. When half the schedulers are idle, this value should be $Budget \over 2$, etc. This makes it possible to have
reduced communication (higher budget) during periods with many busy schedulers and a better balancing
(lower budget) during periods with many idle schedulers.

\subsection{A Simple Example}


\begin{figure*}
    \begin{minipage}{0.3\textwidth}
      \begin{lstlisting}[frame=none, numbers=none]
        p:
        write(x)
      \end{lstlisting}
    \end{minipage}
    \begin{minipage}{0.3\textwidth}
        \begin{lstlisting}[frame=none, numbers=none]
            q:
            read(x)
            write(x)
        \end{lstlisting}
      \end{minipage}
      \begin{minipage}{0.3\textwidth}
        \begin{lstlisting}[frame=none, numbers=none]
            r:
            write(x)
        \end{lstlisting}
      \end{minipage}
      \caption{Simple readers-writes example}
      \label{sourceexample}
  \end{figure*}

Let us consider the example in Figure \ref{sourceexample}. In this case we have 3 processes that write and read a shared variable x.
Figure \ref{Interleavings explored by the sequential source-DPOR} represents the traces explored during the sequential source-DPOR.
We use a bold rectangle to represent a new event and a faint rectangle to denote a replayed event. The red edges represent
the races that are detected and planned. The source set at a state is represented inside the brackets.

\tracelonglong{source1.png}{Interleavings explored by the sequential source-DPOR}

\tracelong{source2f.png}{Initial interleaving explored by the parallel algorithm}

Figure \ref{Initial interleaving explored by the parallel algorithm} depicts the initial step of the parallel source-DPOR. 
This initial execution sequence, along with detected races is partitioned into fragments which get assigned to different schedulers.
This image also contains the initial execution tree which represents the state space that exists in our exploration frontier
at this point.

\tracelong{source3f.png}{Exploration of the assigned traces by each scheduler}

In Figure \ref{Exploration of the assigned traces by each scheduler} we illustrate how each scheduler explores its assigned 
execution sequence. The first scheduler explores its first trace: $p.r.q.q$ (equivalent to the second trace 
at Figure \ref{Interleavings explored by the sequential source-DPOR}). After the race detection takes place,
there are no backtrack entries added below its assigned trace. The only execution sequence planned is the sequence $r$. However,
this sequence is not explored since it does not belong to the state space of the scheduler. The Controller assigned to the
second scheduler the execution sequence $q$. After exploring its first trace, two more races are detected. It is important to notice
here that the first trace of the second scheduler is equivalent to the third trace of the sequential algorithm. However,
the sequential algorithm detects only one race. This happens because on the sequential algorithm the race between $q: read(x)$ and
$r:write(x)$ had already been detected at the second interleaving and so its planning gets skipped. 
On the contrary, on the parallel algorithm this interleaving
is explored by another scheduler and therefore, there is no knowledge of this race been already detected. This leads to
both our schedulers having detected the same race. Nevertheless, in both schedulers this new backtrack entry is outside of their
state space. 

This means that they will have to report their results back to the Controller. The schedulers that reports first its results, 
will be the one to update the execution tree by inserting the new entry found. This
scheduler will add its execution sequence to the frontier, which will be again partitioned
(no need for a partition here since the unexplored frontier will only have one race). Then this execution sequence will be
assigned to an idle scheduler. The scheduler that reports second to the Controller, will not be able to insert its 
backtrack entry into the execution tree, because that entry will already be there, and the backtrack entry gets removed
from its execution sequence. This execution sequence will be left with no more backtrack entries and as such it will not be
inserted into the frontier. This guarantees that we do not explore identical interleavings more than once.

Lets assume that Scheduler 1 was the one that managed to report first to the Controller. Then the execution tree will be the one
depicted in Figure \ref{Execution Tree if Scheduler 1 returned first}. Notice here that the states explored by the Scheduler 1
were deleted from the execution tree. This keeps the size of the execution tree proportional to the size of 
the current exploration frontier. At this point Scheduler 1 will add its execution sequence to the frontier. Then the 
Controller will assign this sequence to Scheduler 1 (since Scheduler 2 has not yet returned), and Scheduler 1 will explore
the last 2 traces (trace 5 and 6 from the sequential example). After both Scheduler 1 and 2 have returned the execution 
will have finished since there will no more traces left to explore.


\tracelong{source4f.png}{Execution Tree if Scheduler 1 returned first}
